\documentclass[a4paper]{article}
\usepackage[utf8]{inputenc}
\usepackage[T1]{fontenc}
\usepackage{graphicx}
\usepackage{geometry}
\geometry{a4paper, margin=1in}
\usepackage{hyperref}

\title{PDF to LaTeX Conversion (via OCR)}
\author{Generated by Python and Gemini}
\date{\today}

\begin{document}

\maketitle

\newpage
\section*{Page 1}

\textbf{GYORGY KEPES}\par

\vspace*{2cm}

\Huge\textbf{LINGUAGEM DA VISÃO}\par

\vspace*{0.5cm}

\textit{Pintura, Fotografia, Publicidade -- Design}\par

\vfill

\hfill \textit{paul theobald e companhia}\par

\newpage
\section*{Page 2}

{\centering
\scshape DÉCIMA TERCEIRA GRANDE IMPRESSÃO \\[1em]
\itshape linguagem da visão \\[1em]
\scshape por GYORGY KEPES \\[1em]
\itshape com ensaios introdutórios \\[1em]
\scshape por S. GIEDION e S. I. HAYAKAWA
\par
}

\bigskip

A {\scshape LINGUAGEM DA VISÃO} é um livro oportuno e corajoso, sem paralelo em sua área. Ele trata dos problemas atuais da expressão visual de um ponto de vista realista e humano, e se esforça para escapar dos confins estreitos do laboratório, preservando o contato mais próximo possível com nossas experiências do dia a dia. Faz uma análise extensa da estrutura e função da imagem gráfica na pintura, fotografia e design publicitário. Ele investiga profundamente as leis da organização visual e avalia em termos contemporâneos os vários dispositivos de representação concebidos por artistas de todas as idades, como -- tamanho, localização vertical, sobreposição, transparência, perspectiva, interpenetração, luz e cor, movimento, etc.

318 ilustrações complementam o texto. Há uma grande variedade de assuntos, técnicas e mídias. Elas incluem fotogramas, fotomontagens e colagens; diagramas analíticos, tipografia, caligrafia e letras; arte pré-histórica, pinturas e desenhos de crianças; trabalho significativo de mestres antigos e artistas contemporâneos, incluindo Arp, Braque, Duchamp, Degas, Juan Gris, Helion, Klee, Kandinsky, Leger, Malevich, Matisse, Moholy-Nagy, Miro, Mondrian, Picasso, Seurat e muitos outros. Design publicitário por Bayer, Binder, Beall, Burtin, Cassandre, Carlu, Doesburg, Lissitzky, Man Ray, McKnight Kauffer, Sutnar, Tschichold, etc. etc.

A {\scshape LINGUAGEM DA VISÃO}, com seu conceito completamente novo de avaliação e reavaliação, é um livro de importância vital para todos que sentem a necessidade urgente de uma compreensão mais clara da estrutura e função da arte em nossa sociedade.

O leigo encontrará na {\scshape LINGUAGEM DA VISÃO}, com suas inúmeras ilustrações, uma revelação iluminadora e rica em interesse humano e artístico.

O professor progressista e o historiador de arte apreciarão a abordagem nova e fresca da {\scshape LINGUAGEM DA VISÃO}; sua análise penetrante e avaliação de períodos passados, bem como movimentos contemporâneos, provarão ser um manual indispensável.

O jovem artista e estudante reconhecerá na {\scshape LINGUAGEM DA VISÃO} um guia simpático através das tendências contraditórias atuais e possíveis confusões pós-guerra. Ele visa fornecer direção, disciplina e uma atitude de pensamento alinhada ao presente.

O designer comercial criativo encontrará na {\scshape LINGUAGEM DA VISÃO} uma fonte de inspiração, de novos horizontes e de potencialidades ilimitadas.

A {\scshape LINGUAGEM DA VISÃO} é um exemplo notável de boa produção de livros. 224 páginas, 8 1/2 x 11, encadernado em tecido completo. O design tipográfico é do autor.

\vspace{2em} % More space before the publisher info

{\centering
\itshape paul theobald e companhia \\
5 North Wabash Ave., Chicago, Ill. 60602
\par
}

\newpage
\section*{Page 3}

%% Error processing this page with the Vision API. See console for details. %% 

\newpage
\section*{Page 4}

\vspace*{2in}
Para Juilska

\newpage
\section*{Page 5}

Gyorgy Kepes

linguagem da visão

Com ensaios introdutórios de S. GIEDION e S. I. HAYAKAWA

paul theobald e companhia 1969

\newpage
\section*{Page 6}

\begin{center}
\normalfont\large\bfseries Agradecimentos
\vspace{0.5em}
\rule{1.5in}{0.4pt}
\vspace{1em}
\end{center}

\raggedright
Em primeiro lugar, o autor deseja expressar seu reconhecimento aos psicólogos da Gestalt. Muitas das ideias inspiradoras e ilustrações concretas de Max Wertheimer, K. Koffka e W. Kohler foram usadas na primeira parte do livro para explicar as leis da organização visual.

O autor também expressa sua gratidão ao seu editor, seus alunos, seus colegas e seus amigos, pelo encorajamento e generosa ajuda na resolução dos muitos problemas criativos e técnicos relacionados a este livro.

Adeline Cross, Britton Harris, Ann Horn, Eva Manzardo, R. B. Tague e Mollie Thwaites, e particularmente Katinka Loeser e Helen Van de Woestyne prestaram inestimável ajuda no trabalho minucioso de revisão e reformulação do texto. O autor também deseja agradecer aos Professores Charles Morris e S. I. Hayakawa, que leram o manuscrito e fizeram críticas úteis.

Seria uma tarefa árdua, se não impossível, ilustrar este livro adequadamente sem a ajuda dos alunos do autor, seus colegas e os membros das equipes dos principais museus. A estes, o autor e o editor desejam expressar sua sincera gratidão. Nossos agradecimentos especiais são devidos à Srta. Frances Pernas, do Museu de Arte Moderna, por suas infalíveis cortesias; a Frank Levstik, Jr., e Hans Richter, por seus esforços; a Carl O. Schniewind e Walter J. Sherwood, do Art Institute of Chicago; a Peggy Guggenheim, do Art of this Century; à Baronesa Rebay, da Fundação Solomon R. Guggenheim; à J. B. Lippincott \& Co., a Charles Downs, da Abbott Laboratories, pelo empréstimo de diversas gravuras; a Egbert Jacobson, da Container Corporation of America, Harry Collins, da Collins, Miller, and Hutchings, R. D. Middleton e ao Dr. R. L. Leslie, pela ajuda com as ilustrações e pelas suas contribuições de gravuras.

\vspace{2em}
\footnotesize
DIREITOS AUTORAIS 1944 POR PAUL THEOBALD AND CO., CHICAGO 60602 \\
TODOS OS DIREITOS RESERVADOS \hfill DÉCIMA TERCEIRA GRANDE IMPRESSÃO \\
IMPRESSO E ENCADERNADO NOS ESTADOS UNIDOS DA AMÉRICA \\
HILLISON \& ETTEN COMPANY \hfill CHICAGO

\newpage
\section*{Page 7}

\noindent\textit{Sumário}\par
\medskip
\hrule
\medskip
\noindent A arte significa realidade por S. Giedion \hfill 6\par
\noindent A revisão da visão por S. I. Hayakawa \hfill 8\par
\noindent A linguagem da visão por Gyorgy Kepes \hfill 11\par
\noindent \quad I. Organização plástica \hfill 15\par
\noindent \quad II. Representação visual \hfill 65\par
\noindent \quad III. Rumo a uma iconografia dinâmica \hfill 200\par

\newpage
\section*{Page 8}

\begin{center}
\textit{Arte Significa Realidade}
\end{center}

Este livro, escrito por um jovem artista, testemunha que uma terceira geração
está em marcha, disposta a continuar e a assegurar a tradição moderna
que se desenvolveu ao longo deste século; ou, como Gyorgy Kepes
afirma: ``Colocar as demandas anteriores em termos concretos e em um plano
social ainda mais amplo.''

Não era a regra no século XIX que as gerações mais jovens
continuassem conscientemente o trabalho de seus predecessores. Fazer isso é novo;
significa que estamos em um período de consolidação.

O público, incluindo aqueles que o governam e administram, ainda carece
do treinamento artístico, ou seja, emocional, correspondente ao nosso período. Ambos
são atormentados pela cisão que existe entre métodos avançados de pensamento
e um pano de fundo emocional que não acompanhou esses métodos.
A demanda por continuidade se tornará cada vez mais a palavra-chave deste
período. \lq A cada dia algo novo\rq{} é a herança do impulso
desastroso do século passado. Isso ainda persiste de muitas maneiras. Continuidade não significa
estagnação ou reação. Continuidade significa desenvolvimento. Cada período muda,
assim como o corpo, de um dia para o outro. Gótico, Renascença, Barroco foram, em
todas as suas fases, em constante desenvolvimento. Mas essas mudanças precisam estar enraizadas
em considerações que não sejam puramente materialistas. Elas precisam crescer de
outras fontes: o Reino de Deus medieval, o absolutismo do século XVII,
uma fé política, ou mesmo um credo artístico.

\lq A cada dia algo novo\rq{} revela a desamparo combinado com a falta de
convicção interna, e sempre ansioso para lisonjear os piores instintos do público.
Significa mudança por mudança, mudança em nome da venda de alta pressão.
Significa desmoralização.

\par
\centering 6

\newpage
\section*{Page 9}

O gosto público hoje é formado principalmente pela publicidade e pelos artigos de uso diário. Por estes, ele pode ser educado ou corrompido. Responsáveis são os diretores de arte na indústria e nas empresas de publicidade e os compradores para as lojas de departamento 5 \& 10 e drogarias, que agem como censores e nivelam os designs dos artistas à sua própria concepção do gosto público. Eles devem alimentar a linha de montagem da maneira mais rápida e, como salvaguarda, julgam o gosto público como inferior ao que realmente é. Sua responsabilidade educacional parece não ter razão de existir.

Quem ainda acredita que a arte, a arte moderna, deve ser definida como um mero luxo ou algo distante, remoto da vida real, indigno do respeito de um \textquoteleft fazedor,\textquoteright\, faria melhor em não tocar neste livro. Gyorgy Kepes, como todos nós, considera a arte indispensável para uma vida plena. Seu principal objetivo é demonstrar como a revolução óptica---por volta de 1910---formou nossa concepção atual de espaço e a abordagem visual da realidade. Ele mostra como esse desenvolvimento foi diferenciado de muitas maneiras de expressão, do cubismo ao surrealismo, formando juntos a imagem multifacetada deste período. Ele mostra por que os artistas modernos tiveram que rejeitar uma obediência servil à representação de objetos, por que odiavam o ``\textit{trompe-l'oeil}.''

Os diferentes movimentos têm um denominador comum: uma nova concepção espacial. Eles não estão obsoletos quando se calam. Cada um deles vive em nós. Passo a passo, Kepes acompanha a libertação dos elementos plásticos: linhas, planos e cores, e a criação de um mundo de formas próprio. A concepção espacial interconecta os fragmentos de significado e os une, assim como a perspectiva de outro período fez quando usou um único ponto de vista para representação naturalista. Devemos notar o grande cuidado com que Gyorgy Kepes mostra o contato da arte moderna com a realidade, e como pinturas que, à primeira vista, parecem remotas da vida, são extraídas de sua própria essência.

Este livro parece ser dirigido à jovem geração que deve reconstruir a América. Esta reconstrução será realizada apenas em anos futuros. Mas o livro poderia ter uma influência imediata se aqueles que comandam o gosto público nas muitas áreas da vida atual dedicassem tempo, num fim de semana tranquilo, para ler suas páginas e refletir sobre elas.

Nova York, 12 de junho de 1944. \hfill S. Giedion

\newpage
\section*{Page 10}

\centering
\large A Revisão da Visão
\par
\vspace{1em}

Qualquer que seja a linguagem que se herde, ela é ao mesmo tempo uma ferramenta e uma armadilha. É uma ferramenta porque com ela ordenamos nossa experiência, combinando os dados abstraídos do fluxo ao nosso redor com unidades linguísticas: palavras, frases, sentenças. O que é verdadeiro para as linguagens verbais também é verdadeiro para as "linguagens" visuais: combinamos os dados do fluxo da experiência visual com clichês de imagem, com estereótipos de um tipo ou de outro, de acordo com a forma como fomos ensinados a ver.

E, tendo combinado os dados da experiência com nossas abstrações, visuais ou verbais, manipulamos essas abstrações, com ou sem referência posterior aos dados, e fazemos sistemas com elas. Esses sistemas de abstrações, artefatos da mente, quando verbais, chamamos de "explicações" ou "filosofias"; quando visuais, chamamos de nossa "imagem do mundo".
Com esses pequenos sistemas em nossas cabeças, observamos o dinamismo dos eventos ao nosso redor, e encontramos, ou nos persuadimos de que encontramos, correspondências entre as imagens dentro de nossas cabeças e o mundo exterior. Acreditando que essas correspondências são reais, nos sentimos à vontade naquilo que consideramos um mundo "conhecido".

Ao dizer por que nossas abstrações, verbais ou visuais, são uma ferramenta, já insinuei por que elas também são uma armadilha. Se as abstrações, as palavras, as frases, as sentenças, os clichês visuais, os estereótipos interpretativos, que herdamos de nosso ambiente cultural, são adequados à sua tarefa, nenhum problema é apresentado. Mas, como outros instrumentos, as linguagens selecionam, e ao selecionar o que selecionam, elas deixam de fora o que não selecionam. O termômetro, que fala um tipo de linguagem limitada, nada sabe sobre peso. Se apenas a temperatura importa e o peso não, o que o termômetro "diz" é adequado. Mas se o peso, a cor, o odor ou outros fatores além da temperatura importam, então esses fatores dos quais o termômetro não pode falar são os dentes da armadilha. Toda linguagem, assim como a linguagem do termômetro, deixa trabalho por fazer para outras linguagens.

Não é um acidente histórico que o Sr. Kepes e outros artistas de similar

\vfill
\centerline{8}

\newpage
\section*{Page 11}

A orientação fala da ``nova visão'' e da ``linguagem da visão''. São necessárias revisões da linguagem. Todos os dias, todos nós, como pessoas, como grupos, como sociedades, estamos presos nas garras daquilo que as linguagens mais antigas ignoram completamente. Falamos de um mundo novo, encolhido e interdependente nos sinais de fumaça primitivos de ``nacionalidade'', ``raça'' e ``soberania''. Falamos dos problemas de uma era de cartéis internacionais e monopólios de patentes na linguagem econômica infantil do Almanaque do Pobre Ricardo. Tentamos visualizar a dinâmica de um universo que é um pleno eletrodinâmico nos clichês representacionais evoluídos em uma época em que ``objetos'' concebidos estaticamente e isoláveis eram considerados como ocupando posições em um ``espaço'' vazio e absoluto. Visualmente, a maioria de nós ainda é ``orientada para objetos'' e não ``orientada para relações''. Somos prisioneiros de antigas orientações embutidas nas linguagens que herdamos.

A linguagem da visão determina, talvez de forma ainda mais sutil e completa do que a linguagem verbal, a estrutura de nossa consciência. Ver em modos limitados de visão é não ver nada --- é estar limitado aos mais estreitos paroquialismos do sentimento.

O que o Sr. Kepes nos faria, então, por sua tentativa de reeducação visual, é nos compelir a levar em consideração a ``refração'' de nossos modos de visão herdados. Ele faz isso mostrando-nos o que entra em nossa experiência. Ele nos dá a ``gramática'' e a ``sintaxe'' da visão: quais interações de quais forças no sistema nervoso humano e no mundo exterior produzem quais tensões visuais e resoluções de tensões; quais combinações de elementos visuais resultam em novas organizações de sentimento; quais ``declarações visuais'', à parte do conteúdo ``literário'' ou representacional, podem ser feitas com linha, cor, forma, textura e arranjo.

Privando-nos propositalmente do fácil conforto de todos os estereótipos estéticos e clichês interpretativos, o Sr. Kepes nos faria experimentar a visão como visão. O esforço do Sr. Kepes talvez possa ser melhor caracterizado pela seguinte analogia. Para um estudioso chinês, o prazer derivado de uma inscrição se deve apenas parcialmente aos sentimentos que ela pode expressar. Ele pode se deleitar com a caligrafia mesmo quando a inscrição não tem sentido para ele como texto. Suponha agora que existisse um estudioso chinês singularmente obtuso, que estivesse unicamente preocupado com o conteúdo literário ou moral das inscrições, e totalmente cego à sua caligrafia. Como alguém o faria ver a

\newpage
\section*{Page 12}

Qualidades caligráficas de uma inscrição se ele persistisse, cada vez que a inscrição fosse apresentada para exame, em discutir seu conteúdo literário, sua exatidão ou inexatidão como declaração de fato, sua aprovação ou desaprovação de suas injunções morais?

É precisamente um problema como este que o artista contemporâneo enfrenta, confrontado com um público para quem o conteúdo literário, sentimental, moral, etc., da arte é arte --- para quem a experiência visual como tal é uma dimensão quase completamente ignorada. A vasta maioria de nós --- e por nós não me refiro apenas àqueles que professam saber algo sobre ``arte'', mas também ao público em geral que se deleita com capas de revistas e calendários de seguradoras e gravuras de caça e quadros de veleiros --- somos sofisticados por nosso ambiente cultural além do ponto em que é fácil entender o que pessoas como o Sr. Kepes estão querendo dizer. Todos fomos ensinados, ao olhar para imagens, a procurar por demais. Algo da qualidade do deleite de uma criança ao brincar com cores e formas precisa ser restaurado em nós antes que aprendamos a ver novamente, antes que desaprendamos os termos nos quais ordinariamente vemos.

Essa restauração da visão, então, é o que a ``gramática'' da visão do Sr. Kepes realizaria para nós.

A revisão da linguagem da visão não é, claro, um fim artístico em si mesma para o Sr. Kepes. Como lidamos com a realidade é determinado no momento do impacto pela forma como a apreendemos. A visão compartilha com a fala a distinção de ser o mais importante dos meios pelos quais apreendemos a realidade.

Cessar de olhar para as coisas atomicamente na experiência visual e ver a conectividade significa, entre outras coisas, perder em nossa experiência social, como argumenta o Sr. Kepes, a importância própria ilusória do ``individualismo'' absoluto em favor da conectividade social e da interdependência. Quando estruturarmos os impactos primários da experiência de forma diferente, estruturaremos o mundo de forma diferente.

A reorganização de nossos hábitos visuais para que percebamos não ``coisas'' isoladas no ``espaço'', mas estrutura, ordem e a conectividade de eventos no espaço-tempo, é talvez o tipo mais profundo de revolução possível --- uma revolução que está há muito atrasada não apenas na arte, mas em toda a nossa experiência.

\raggedleft
S. I. Hayakawa \\
Instituto de Tecnologia de Illinois

\newpage
\section*{Page 13}

\vspace*{\stretch{1}}
\centering
{\itshape A Linguagem da Visão}

\vspace*{\stretch{2}}
\raggedleft 11

\newpage
\section*{Page 14}

Hoje experimentamos o caos. O desperdício de recursos humanos e materiais e a canalização de quase todo o esforço criativo para becos sem saída atestam o fato de que nossa vida comum perdeu sua coerência. No foco deste eclipse de uma existência humana saudável está o indivíduo, dilacerado pelos fragmentos despedaçados de seu mundo informe, incapaz de organizar suas necessidades físicas e psicológicas.

Essa informe tragicidade é o resultado de uma contradição em nossa existência social. Ela indica nossa falha na organização daquele novo equipamento com o qual devemos funcionar se quisermos manter nosso equilíbrio em um mundo dinâmico.

Avanços na ciência e tecnologia criaram uma nova dimensão. Hoje, todas as pessoas do mundo são vizinhas, e os recursos naturais em uma escala até então inimaginável estão ao alcance de todos. A estrutura herdada de um mundo menor e superado, no entanto, impede a integração de nossas vidas nos termos da atual dimensão mais ampla. A agressão totalitária, como o aspecto mais destrutivo da resistência do passado, buscou direcionar o presente e o futuro para uma organização obsoleta, de uma necessidade empregando força para realizar o que era diametralmente oposto aos princípios de crescimento e desenvolvimento. Essas forças destrutivas, por outro lado, inevitavelmente preparam o caminho para a reconstrução. Quanto mais insuportáveis as tensões e pressões causadas pela contradição entre as potencialidades do presente e as formas ultrapassadas do passado, mais forte é a compulsão para equilibrar nossa vida na dimensão contemporânea. Para alcançar uma vida habitável hoje, então, devemos nos reorientar e criar formas em termos das condições históricas presentes. Em vez de permitir tanto uma nova acumulação aleatória de descobertas científicas quanto uma expansão tecnológica sem plano, nossa tarefa é estabelecer uma interconexão orgânica das novas fronteiras do conhecimento. Integração, planejamento e forma são as palavras-chave de todos os esforços progressivos hoje; o objetivo é uma nova ordem-estrutura vital, uma nova forma no plano social, na qual todo o conhecimento presente e as posses tecnológicas possam funcionar sem impedimentos como um todo.

Essa nova ordem-estrutura só pode ser alcançada se o homem de hoje se tornar verdadeiramente contemporâneo e totalmente capaz de usar suas capacidades. Ser contemporâneo em um sentido verdadeiro exige um conhecimento muito avançado dos fatos que governam a vida de hoje. A compreensão dos aspectos vitais de nossa vida, no entanto, para a maioria de nós ainda está no mesmo estágio de cem anos atrás. No passado, raios, pragas e fome eram considerados visitas da Providência, mas hoje, através do conhecimento e da compreensão, somos capazes de controlá-los. De igual modo, se fizermos uma aplicação social do conhecimento científico, os obstáculos presentes a uma existência humana contemporânea seriam eliminados. Devemos dissipar a crença de que guerra, crises econômicas ou desintegração psicológica são inevitáveis e devido a forças cegas e hostis da natureza. Os esforços coletivos dos cientistas nos deram uma vida mais rica e segura nos domínios biológicos e físicos; devemos enfrentá-los em
\vfill
\begin{center}
12
\end{center}

\newpage
\section*{Page 15}

domínios socioeconômicos e psicológicos. A educação em uma escala sem precedentes é imperativa se o homem, que agora vive em um mundo mais amplo, deseja ser verdadeiramente contemporâneo.

Mas esse novo conhecimento só pode ser a fibra viva da integração se o homem o experimentar com a totalidade do seu ser. As faculdades humanas, no entanto, foram embotadas e desintegraram-se em um clima de frustração. A experiência tendeu a se tornar apenas um trampolim para a exploração da natureza e do homem. As experiências são compartimentos isolados; elas exibem apenas aspectos únicos dos seres humanos. Para funcionar em sua plenitude, o homem deve restaurar a unidade de suas experiências para que possa registrar as dimensões sensoriais, emocionais e intelectuais do presente em um todo indivisível.

\textit{A linguagem da visão}, comunicação óptica, é um dos meios potenciais mais fortes tanto para reunir o homem e seu conhecimento quanto para reformar o homem em um ser integrado. A linguagem visual é capaz de disseminar o conhecimento de forma mais eficaz do que quase qualquer outro veículo de comunicação. Com ela, o homem pode expressar e transmitir suas experiências em forma de objeto. A comunicação visual é universal e internacional: não conhece limites de língua, vocabulário ou gramática, e pode ser percebida tanto pelo iletrado quanto pelo letrado. A linguagem visual pode transmitir fatos e ideias em um alcance mais amplo e profundo do que quase qualquer outro meio de comunicação. Ela pode reforçar o conceito verbal estático com a vitalidade sensorial de imagens dinâmicas. Pode interpretar a nova compreensão do mundo físico e dos eventos sociais porque as inter-relações dinâmicas e a interpenetração, que são significativas de toda compreensão científica avançada de hoje, são expressões intrínsecas dos veículos contemporâneos de comunicação visual: fotografia, cinema e televisão.

Mas a linguagem da visão tem uma tarefa contemporânea mais sutil e, até certo ponto, ainda mais importante. Perceber uma imagem visual implica a participação do observador em um processo de organização. A experiência de uma imagem é, portanto, um ato criativo de integração. Sua característica essencial é que, pelo poder plástico, uma experiência é formada em um todo orgânico. Aqui está uma disciplina básica de formação, isto é, pensar em termos de estrutura, uma disciplina de suma importância no caos do nosso mundo informe. As artes plásticas, as formas ótimas da linguagem da visão, são, portanto, um meio educacional inestimável.

A linguagem visual deve ser reajustada, no entanto, para enfrentar seu desafio histórico de educar o homem para um padrão contemporâneo, e de ajudá-lo a pensar em termos de forma.

As descobertas tecnológicas estenderam e remodelaram o ambiente físico. Elas mudaram nosso entorno visual em parte pela reconstrução real do ambiente físico, e em parte pela apresentação de ferramentas visuais que auxiliam nosso discernimento daquelas fases do mundo visível que antes eram muito pequenas, muito rápidas, muito grandes ou muito lentas para compreendermos. A visão é principalmente um dispositivo de orientação; um meio para medir--

\newpage
\section*{Page 16}

e organizar eventos espaciais. O domínio da natureza está intimamente conectado ao domínio do espaço; esta é a orientação visual. Cada novo ambiente visual exige uma reorientação, uma nova forma de medição. Ver relações espaciais em terreno plano é uma experiência diferente de vê-las em uma região montanhosa, onde uma forma intercepta a outra. Orientar-se ao caminhar exige uma medição espacial diferente da exigida ao andar de carro ou de avião. Compreender as relações espaciais e orientar-se em uma metrópole de hoje, entre as dimensões intrincadas de ruas, metrôs, elevados e arranha-céus, exige uma nova forma de ver. A ampliação dos horizontes e as novas dimensões do ambiente visual necessitam de novos idiomas de medição espacial e comunicação do espaço. A imagem visual de hoje deve lidar com tudo isso: deve evoluir uma linguagem do espaço que esteja ajustada aos novos padrões de experiência. Essa nova linguagem pode e permitirá que a sensibilidade humana perceba relações espaço-tempo nunca antes reconhecidas. A visão não é apenas orientação em esferas físicas, mas também orientação em esferas humanas. O homem não pode suportar mais o caos em sua vida emocional e intelectual do que pode suportá-lo em sua existência biológica. Em cada era da história humana, o homem foi compelido a buscar um equilíbrio temporário em seus conflitos com a natureza e em suas relações com outros homens, e assim criou, através de uma organização de imagens visuais, uma ordem simbólica de suas experiências psicológicas e intelectuais. Essas formas de sua imaginação criativa o direcionaram e inspiraram a materializar a ordem potencial inerente a cada estágio da história. Mas até hoje, a organização simbólica de conflitos psicológicos e intelectuais tem sido limitada em seu poder porque estava presa a um sistema estático de conceitos de objeto. Hoje, a dinâmica dos eventos sociais e as novas perspectivas de um mundo móvel e físico nos compeliram a trocar uma iconografia estática por uma dinâmica. A linguagem visual deve, portanto, absorver os idiomas dinâmicos da imagética visual para mobilizar a imaginação criativa para a ação social positiva e direcioná-la para metas sociais positivas.

Hoje, artistas criativos têm três tarefas a cumprir se a linguagem da visão for se tornar um fator potente na remodelação de nossas vidas. Eles devem aprender e aplicar as leis da \textbf{organização plástica} necessárias para o restabelecimento da imagem criada em uma base saudável. Eles devem lidar com as experiências espaciais contemporâneas para aprender a utilizar a \textbf{representação visual} de eventos espaço-temporais contemporâneos. Finalmente, eles devem liberar as reservas da imaginação criativa e organizá-las em idiomas dinâmicos, ou seja, desenvolver uma \textbf{iconografia dinâmica} contemporânea.

\newpage
\section*{Page 17}

\section*{I. Organização Plástica}

\subsection*{\textit{A imagem criada}}

Vivemos em meio a um turbilhão de qualidades de luz. Dessa confusão vertiginosa construímos entidades unificadas, aquelas formas de experiência chamadas imagens visuais.

Perceber uma imagem é participar de um processo de formação; é um ato criativo. Da forma mais simples de orientação à mais abrangente unidade plástica de uma obra de arte, há uma base significativa comum: o acompanhamento das qualidades sensoriais do campo visual e a sua organização. Independente do que se ``vê'', toda experiência de uma imagem visual é uma formação; um processo dinâmico de integração, uma experiência ``plástica''. A palavra ``plástica'' é, portanto, usada aqui para designar a qualidade formativa, a modelagem das impressões sensoriais em totalidades unificadas e orgânicas.

\par\vspace{\baselineskip}%
{\small%
\noindent\hangindent=\parindent\hangafter=1\kern-\parindent$\bullet$\enspace Ao longo desta discussão e do que se segue, deve-se entender que todos os termos utilizados são arbitrários e não devem ser considerados como cientificamente estabelecidos. O uso de tais termos tornou-se necessário pela falta de uma terminologia adequada no campo da experiência visual considerada como uma atividade criativa.%
}%
\normalsize%

\vfill
\null\hfill 15

\newpage
\section*{Page 18}

A experiência de uma imagem plástica é uma forma evoluída através de um processo de organização. A imagem plástica possui todas as características de um organismo vivo. Ela existe através de forças em interação que atuam em seus respectivos campos, e são condicionadas por esses campos. Possui uma unidade orgânica e espacial; ou seja, é um todo cujo comportamento não é determinado pelo de seus componentes individuais, mas onde as partes são determinadas pela natureza intrínseca do todo. É, portanto, um sistema fechado que atinge sua unidade dinâmica por vários níveis de integração; por equilíbrio, ritmo e harmonia.

A experiência de cada imagem é o resultado de uma interação entre forças físicas externas e forças internas do indivíduo à medida que ele assimila, ordena e molda as forças externas à sua própria medida. As forças externas são agentes luminosos bombardeando o olho e produzindo alterações na retina. As forças internas constituem a tendência dinâmica do indivíduo de restaurar o equilíbrio após cada perturbação externa, e assim manter seu sistema em relativa estabilidade.

Toda força age em um meio, existe em um campo. Qualquer processo induzido por forças só faz sentido em referência ao ambiente, como uma interação entre a força e o meio em que atua. Andamos contra a resistência da terra, a extensão espacial do mundo objetivo. Voamos, impulsionados pela resistência do ar. O quadro de referência ou campo em que uma força atua condiciona o alcance e o caminho da ação induzida. O peso e a forma de um material, assim como a natureza do meio resistente, definirão as manifestações da força da gravidade. Uma pedra solta no ar se comporta diferentemente de uma solta na água, neve, mercúrio ou lama.

Forças ópticas e as respostas fisiológicas e psicológicas que elas induzem também só são significativas em seus respectivos campos. As forças ópticas externas que fornecem as bases físicas da experiência que chamamos de imagem plástica, e as forças internas---a tendência dinâmica de integrar os impactos do ambiente---atuam dentro de seus respectivos quadros de referência. Deve-se ter em mente, no entanto, que é o sistema nervoso que organiza os impactos do exterior. Portanto, a distinção entre quadros de referência externos e internos é, em certo sentido, artificial e utilizada apenas por conveniência, uma vez que em cada experiência o quadro de referência externo é transformado em parte do interno.

\section*{Forças Externas}

A imagem plástica como uma experiência dinâmica começa com a energia luminosa fluindo através do olho do espectador para seu sistema nervoso. Por exemplo, essa energia luminosa é articulada em uma superfície pictórica em diferentes extensões por diferentes pigmentos. A natureza dos pigmentos fornece a base para sensações de luz e cor; ou seja, brilho, matiz e saturação. A demarcação geométrica dessas qualidades fornece a base física para a percepção de áreas e suas formas. Juntos, esses fatores constituem o vocabulário da linguagem da visão e atuam como as forças ópticas de atração.

\vfill
16

\newpage
\section*{Page 19}

\begin{center}
\textit{Ilusão visual de tamanho e direção}
\end{center}

\vspace{1em}

\textit{O campo visual, o campo retiniano}

As forças de atração visual---um ponto, uma linha, uma área---existem em um fundo óptico e atuam sobre o campo óptico. Este campo óptico é projetado na superfície retiniana dos olhos como um fundo inseparável para as distintas unidades visuais. Não se pode, portanto, perceber as unidades visuais como entidades isoladas, mas como relações. "Como as chamadas ilusões ópticas mostram, não vemos frações individuais de uma coisa; em vez disso, o modo de aparecimento de cada parte depende não apenas da estimulação que surge naquele ponto, mas também das condições que prevalecem em outros pontos."\textbullet

Cor e valor dependem sempre das superfícies circundantes imediatas. Um valor de brilho pode ser amplificado ou anulado pelos outros valores. Uma cor pode ser intensificada ou neutralizada da mesma forma.

O mesmo se aplica às qualidades de textura. Tamanhos e formas também são percebidos apenas em unidade polar com um fundo e sua qualidade óptica específica é devida aos seus respectivos referenciais. Uma forma ligeiramente irregular parece fortemente irregular em um referencial de quadrados geometricamente perfeitos, mas a mesma forma parece perfeitamente regular em referência a unidades extremamente irregulares. Em geral, todas as unidades ópticas em uma superfície de imagem derivam suas qualidades em relação aos seus respectivos fundos, variando da superfície circundante imediata ao campo óptico como um todo.

\textbullet\ W. Kohler, Physikal Gestalten 1920

\newpage
\section*{Page 20}

\textit{Ilusão visual de valores}

Não pode haver, portanto, nem uma qualidade absoluta de cor, brilho, saturação, nem uma medida absoluta de tamanho, comprimento e forma no campo óptico, porque cada unidade visual ganha seu modo único de aparência em uma inter-relação dinâmica com seu ambiente óptico. Eis um ponto importante. A gama de matiz, valor, saturação e a escala de medida geométrica é incomparavelmente mais estreita na superfície da imagem do que no ambiente visível de uma pessoa, e somente por um uso criativo da relatividade das diferenças ópticas pode-se criar uma imagem óptica na superfície que se iguale à vitalidade do mundo visível.

``O céu em uma paisagem pode ser milhares de vezes mais brilhante que uma sombra profunda ou um buraco no chão. Uma nuvem cumulus no céu pode ser centenas de milhares de vezes mais brilhante que a sombra mais escura. No entanto, o artista deve representar uma paisagem por meio de uma paleta cujo branco é apenas cerca de trinta vezes mais brilhante que o preto.''

\textit{M. Luckiesh, Ilusões visuais.}

18

\newpage
\section*{Page 21}

\textbf{\textit{O campo tridimensional}}
Ao observar uma paisagem, pessoas na rua, ou qualquer objeto individual, como o campo visual não possui limites definidos, só se pode fazer uma interpretação espacial das coisas que se veem---sua localização, extensão---com base em sua própria posição espacial. Ele julga a posição, direção e intervalo das coisas vistas relacionando-as a si mesmo. Ele mede e organiza para cima, para baixo, para a esquerda, para a direita, avanço e recessão em um único sistema físico do qual seu corpo é o centro e é identificado com as principais direções no espaço. O eixo horizontal e vertical egocêntrico é o fundo latente, e as diferenças ópticas são interpretadas contra esse fundo. Se o espectador move a cabeça, o olho ou o corpo---consequentemente mudando sua posição e consequentemente mudando o campo retiniano da posição vertical natural---, ele imediatamente transfere para os objetos mais próximos a ele o papel original do corpo humano e as principais direções do espaço permanecem válidas.

\textbf{\textit{O campo da imagem}}
O campo visual de uma imagem é menos difuso. Ele é limitado aos contornos de um plano de imagem e às duas dimensões desta superfície.

O quadro de referência muda da direção espacial mais geral do espectador para o novo fundo do campo da imagem---para as quatro bordas e as duas dimensões. É criado um quadro de referência inteiramente novo, um mundo com novas leis formadas a partir das novas relações.

As quatro bordas do plano da imagem geralmente assumem as principais direções do espaço, e cada unidade óptica distinta na superfície recebe sua avaliação espacial, sua posição, direção e intervalo devido à sua relação com as margens consideradas os eixos horizontal e vertical do mundo recém-criado. O plano de imagem bidimensional assume o centro do campo espacial e cada unidade óptica parece avançar ou recuar dele. Um ponto, uma linha ou uma forma na superfície da imagem é vista como possuindo qualidades espaciais. Se um ponto ou uma linha for colocado em uma ou outra posição na superfície, a posição das respectivas unidades ópticas em relação à margem da imagem relacionará diferentes significados espaciais como uma forma dinâmica de movimento. Os elementos parecem estar se movendo para a esquerda, para a direita, para cima, para baixo, e recuando ou avançando, dependendo de sua respectiva posição no plano da imagem. As unidades ópticas criam uma interpretação da superfície como um mundo espacial; elas têm força e direção, elas se tornam forças espaciais.

\vspace*{\fill}
\begin{center}
19
\end{center}

\newpage
\section*{Page 22}

\begin{center}
\itshape As forças espaciais
\end{center}

Uma pedra, uma árvore ou um peixe tem seu próprio tipo particular de existência. A pedra é estática com o movimento perpendicular latente de seu peso. A árvore pode se expandir em qualquer direção, mas não pode mudar sua posição. O peixe pode se mover e assumir qualquer posição. Cada um se comporta de acordo com sua natureza específica. Da mesma forma, qualquer unidade visível colocada em um plano-imagem gera uma vida própria.

Posições, direções e diferenças de tamanho, forma, brilho, cor e textura são medidas e assimiladas pelo olho. O olho confere o caráter de sua experiência neuromuscular à sua fonte. Uma vez que cada forma, cor, valor, textura, direção e posição produz uma qualidade diferente de experiência, deve surgir uma contradição inerente de estarem na mesma superfície plana. Essa contradição só pode ser resolvida à medida que eles têm a aparência de movimento no plano-imagem. Esses movimentos virtuais de qualidades ópticas moldarão e formarão o espaço da imagem, agindo assim como forças espaciais. A qualidade espacial deriva apenas incidentalmente do fato de que os sinais ópticos se assemelham a objetos conhecidos empiricamente. Experimenta-se o espaço ao olhar para uma superfície bidimensional articulada principalmente porque se tenta inconscientemente organizar e perceber as diferentes sensações induzidas pelas qualidades e medidas opcionais como um todo, e ao fazê-lo é forçado, pelas várias qualidades em suas relações entre si e com a superfície da imagem, a atribuir significado espacial a essas relações.

No diagrama retirado de Kopferman, os quadrados pretos em um contorno retangular que indica os limites do plano-imagem demonstram as modificações da mesma forma sob várias condições. Onde quer que seja visto como um quadrado, em parte porque é paralelo às bordas do plano-imagem, e em parte porque está realmente em uma posição horizontal-vertical em relação ao próximo quadro de referência --- a página. É, portanto, dependente do fundo em que aparece. Se o fundo tem uma correspondência definida com o eixo horizontal-vertical, no entanto, a figura quadrada em posição diagonal não só perde sua estabilidade, mas sofre uma modificação. É vista, não como um quadrado, mas como um losango. Um estudo do diagrama torna óbvio que a relação da unidade com a borda da imagem gera sua expressão espacial. Em um caso, ela aparece estática e suspensa; em outro, estática, mas com forte resistência --- quase com uma qualidade de solidez; em um terceiro caso, ela muda de forma e perde sua concretude; finalmente, sugere um movimento potencial e flutua entre a forma de quadrado e a forma de losango.

20

\newpage
\section*{Page 23}

Quer queiramos ou não, qualquer diferenciação óptica de uma superfície de imagem gera uma sensação de espaço. Um desenho tipográfico, rabiscos no papel, manchas de cor numa tela, uma manipulação simples e aleatória de luz ou uma pintura com uma mensagem emocional explosiva --- tudo isso são expressões espaciais em virtude do processo pelo qual o olho organiza suas diferenças visíveis em um todo.

Picasso. Mulher Chorando 1936\\
Reprodução Cortesia\\
The Art Institute of Chicago

\newpage
\section*{Page 24}

\vspace*{8cm} % Espaço para a primeira imagem

Kandinsky. 9 Pontos em Ascensão

\vspace*{8cm} % Espaço para a segunda imagem

Malevich. Sensação de Voo 1914-15

\vfill
\centering
22

\newpage
\section*{Page 25}

\noindent El Lissitzky. \textit{Ilustração} 1923

\bigskip

Antes de se começar a usar a linguagem visual para a comunicação de uma mensagem concreta, deve-se aprender a maior variedade possível de sensações espaciais inerentes às relações das forças que atuam na superfície da imagem. O acúmulo de uma experiência tão variada é a parte mais importante do treinamento para a expressão visual. Aquilo que se chama educação técnica, o domínio de uma habilidade particular ou de um hábito particular de representação visual, deve ser postergado enquanto se aprende a base objetiva da linguagem da visão. Uma manipulação lúdica de cada elemento: pontos, formas, linhas\textemdash variando-os em posição, cor, valor e textura\textemdash é o caminho mais curto para a compreensão de suas inter-relações. Assim como as letras do alfabeto podem ser combinadas de inúmeras maneiras para formar palavras que transmitem significados, as medidas e qualidades ópticas podem ser reunidas de inúmeras maneiras, e cada relação particular gera uma sensação diferente de espaço. As variações a serem alcançadas são infinitas. Pois, embora os sinais elementares da língua inglesa sejam apenas vinte e seis, o número de forças elementares com as quais o maquinário da visão é provido é prodigioso.

\vfill\null\hfill 23\null

\newpage
\section*{Page 26}

Um ponto de cor gera diferentes experiências de espaço dependendo se é colocado no meio do plano da imagem, à esquerda ou direita, ou na parte superior ou inferior. Cada interrelação única produz uma sensação espacial única. A introdução de mais de um ponto aumenta a sensação de espaço. Os pontos se afastam um do outro ou se aproximam, recuando ou avançando, e parecem ter peso ou uma direção centrípeta ou centrífuga. Um evento espacial ainda mais vital é criado quando essas áreas de superfície são articuladas em tamanho, cor. Linhas retas e curvas em uma relação horizontal, vertical ou diagonal com a margem da imagem forçam o olho a se orientar e explorar a superfície de uma maneira diferente e originam outra variedade de sensação espacial. Uma expressão espacial ainda mais rica pode ser criada manipulando várias formas na superfície da imagem. Seu valor, cor, textura e posição relativa induzem experiências espaciais de maior intensidade e variedade.

24

\newpage
\section*{Page 27}

%% Error processing this page with the Vision API. See console for details. %% 

\newpage
\section*{Page 28}

Adeline Cross. Estudo das qualidades espaciais de avanço e recuo dos valores tonais.\\
\textit{Trabalho feito para o curso da autora em Fundamentos Visuais. Escola de Design em Chicago.}

26

\newpage
\section*{Page 29}

\noindent\itshape Estudo das qualidades de avanço e recuo das cores

\null
\vspace*{\fill}
\centering 27

\newpage
\section*{Page 30}

\textit{Campo magnético}

\textit{H. L. Carpenter.}\\
\textit{A atração e repulsão das}\\
\textit{forças espaciais.}\\
\textit{Trabalho realizado para o curso do autor}\\
\textit{em Fundamentos Visuais.}\\
\textit{Escola de Design em Chicago}

\vfill
\centering 28

\newpage
\section*{Page 31}

\subsection*{Campos de forças espaciais}

Um ser humano é mais do que seu próprio corpo; ele implica as ações que
atingem e transformam seu ambiente. Uma barra de aço magnetizada é mais do que sua
própria massa; seu campo elétrico pertence a ela tanto quanto sua substância, sua
forma e seu peso. A superfície da imagem torna-se um mundo espacial vital, não apenas
no sentido de que as forças espaciais estão agindo sobre ela --- movendo, caindo e circulando
--- mas também no sentido de que entre esses movimentos o próprio campo é carregado com
ação. Os elementos visuais reais são apenas os pontos focais deste campo; eles são
a energia concentrada. Cor, valor, textura, ponto, linha e área irradiam diferentes
quantidades de energia, e assim cada elemento ou qualidade pode abranger um diferente
raio da superfície da imagem. Esses campos se estendem em todas as dimensões e cada
campo tem sua própria forma única.

\textit{Os campos de forças podem ser interrompidos, ou
podem colidir uns com os outros. Um campo interceptando outro campo, o atrai ou o repele; o
reforça ou interfere nele. Essa interação
de um campo com outro causa tensões e
esforços. Quando duas linhas se cruzam, por exemplo, os
campos de forças se confrontam e as energias espaciais são
concentradas no ângulo de reflexão.}

\vspace*{\fill}
\raggedleft 29

\newpage
\section*{Page 32}

\centering
Forças internas

Todo organismo vivo -- seja uma planta, um animal, um ser humano, ou uma estrutura social -- é uma forma relativamente constante. Assim como as rodas de uma bicicleta se mantêm eretas apenas através do movimento perpétuo, o organismo mantém sua forma através do movimento constante. Uma planta, por exemplo, pelo processo perpétuo de metabolismo, utiliza a luz solar, água, solo, retendo apenas o que necessita para manter seu organismo relativamente estável. Para manter a mesma estrutura constante, todo organismo vivo deve alcançar uma unidade dinâmica. A imagem plástica não é exceção. Somente pela ordem dinâmica ela pode se tornar uma forma viva da experiência humana. "O olho exige especialmente a completude", diz Goethe. Forças internas estão agindo para restaurar o equilíbrio após cada perturbação externa.

As mudanças produzidas pela luz na retina são equilibradas por reações fisiológicas. Quando a energia luminosa induz perturbações fisiológicas na retina, consumindo certas substâncias químicas, etc., o organismo em geral e a retina em particular restauram as substâncias consumidas. Quando certas ações musculares impostas pela distribuição de luz no campo visual induzem fadiga, o organismo responde com um movimento complementar, acionando novas unidades neuromusculares. Assim como qualquer trabalhador altera, de tempos em tempos, o ritmo de seu trabalho, o olho, ou melhor, o aparelho neuromuscular, tende a encontrar o mesmo alívio da fadiga, uma experiência de equilíbrio. Cada impacto externo no olho é neutralizado por um movimento recíproco interno. Se o olho for atingido por um feixe de luz intenso e súbito, ele se fecha automaticamente. Mas o olho também possui reações mais positivas e sutis à estimulação luminosa. Se raios de luz vermelha o atingem por um período prolongado, ele reage, ao ser desviado da superfície vermelha, vendo imediatamente uma pós-imagem verde. O organismo biológico age para restaurar a sensibilidade da superfície que foi mantida em ação, dando-lhe descanso completo.

\vfill
\centering
30

\newpage
\section*{Page 33}

A tendência dinâmica para o equilíbrio não se restringe a um nível biológico. A visão é mais do que pura sensação, pois os raios de luz que atingem o olho não possuem uma ordem intrínseca como tal. Eles são apenas um panorama aleatório e caótico de acontecimentos de luz móveis e independentes. Assim que atingem a retina, a mente os organiza e molda em unidades espaciais significativas. Não podemos suportar o caos---a perturbação do equilíbrio no campo da experiência. Consequentemente, devemos imediatamente transformar os impactos de luz em formas e figuras. Exposto a um campo visual que em sua qualidade de luz é minimamente heterogêneo, organiza-se esse campo de uma vez em dois elementos opostos; em uma figura contra um fundo. Fala-se de branco com uma referência implícita inevitável ao preto, cinza ou outras cores. Para transmitir o significado de ``sim'', implica-se uma compreensão latente de ``não''. Um todo unificado é assim criado. Toda imagem é baseada nesse dualismo dinâmico, na unidade dos opostos. Certos impulsos são unidos em um todo visual estável, enquanto outros impulsos são deixados em seu estado fluido desorganizado e servem apenas como fundo e são percebidos como intervalos. Essa organização de figuras e fundos é repetida progressivamente até que todo o campo visual seja percebido como uma unidade formada e ordenada---a imagem plástica.

\par
\textit{Fluctuação da figura e do fundo.}

\par
\textit{Mi Yujen. Paisagem. \\ The Cleveland Museum of Art}

\newpage
\section*{Page 34}

Em cada conceito claro da natureza da visão e em cada abordagem saudável ao mundo espacial, esta unidade dinâmica de figura e fundo tem sido claramente compreendida. Lao Tse demonstrou tal compreensão ao dizer: ``Um vaso é útil apenas através de seu vazio. É o espaço aberto em uma parede que serve como janela. Assim, é o não-existente nas coisas que as torna utilizáveis.'' A cultura visual oriental tem uma profunda compreensão do papel do espaço vazio na imagem. Pintores chineses e japoneses têm a admirável coragem de deixar grandes trechos vazios em sua superfície pictórica, de modo que a superfície é dividida em intervalos desiguais que, através de seu espaçamento, forçam o olho do espectador a movimentos de velocidade variável ao seguir as relações, e assim criam a unidade pela maior variação possível da superfície. A caligrafia chinesa e japonesa também tem um profundo respeito pelo intervalo branco. Os caracteres são escritos em quadrados imaginários, e as áreas em branco recebem tanta consideração quanto as unidades gráficas, os traços. A comunicação escrita ou impressa é viva ou morta dependendo da organização de seus espaços em branco. Um único caractere ganha clareza e significado pela relação ordenada do fundo espacial que o cerca. Quanto maior a variedade e a distinção entre as respectivas unidades de fundo, mais clara se torna a compreensão de um caractere como uma expressão ou signo individual.

\par

O significado do intervalo espacial também está sendo compreendido na arquitetura. Por um tempo, a ideia de integração de estruturas espaciais, forma orgânica, na qual figura e fundo são considerados em uma unidade de interdependência mútua, foi perdida na pressa selvagem do progresso tecnológico. Cada nova invenção, cada nova descoberta científica, cada novo produto, era considerada sem referência às suas implicações para a vida humana. Mas estamos testemunhando agora uma reorientação para uma vida mais integrada, alcançada através do reconhecimento progressivo da interconexão entre figura e fundo. Arquitetos contemporâneos estão se afastando da ênfase unilateral na fachada de um edifício, e os melhores exemplos da arquitetura contemporânea mostram uma integração perfeita do edifício real, do ``envelope'' ativo, das divisões criadas pelos materiais e dos espaços de convivência entre esses materiais. Telas leves, cortinas, paredes de vidro são empregadas para ampliar essa integração opticamente e para criar um espaço vivo, fluído, articulado interna e externamente: uma única unidade viva.

\par

A mesma tendência prevalece na ciência. Diz Erwin Schrodinger:
``Não temos mais medo de grandes espaços vazios em nossos móveis ou em nossas paredes. Não temos o que os alemães chamam de 'platz angst'---o medo dos espaços vazios---mais. \ldots Agora, há algo semelhante em nossa ciência. \ldots Não queremos acessórios ornamentais. Assim como não temos mais medo de superfícies nuas em nossos móveis e salas de estar, em nossa imagem científica do mundo externo não tentamos preencher os espaços vazios.''$\bullet$

\par
{\leftskip=\parindent\relax
$\bullet$\textit{Erwin Schrodinger, Physical Science and the Temper of the Age.}
}

\par
\centering 32

\newpage
\section*{Page 35}

Caligrafia em estilo Ts'Ao (cursivo\\
corrido) por Su\\
Tung-P'o.\\
De um decalque na Coleção Berthold\\
Laufer, Field Museum\\
de História Natural, Chicago,\\
Cortesia da Srta. Lucy Driscoll

Cortesia de R. Hunter Middleton\\
Ludlow Typographic Company

33

\newpage
\section*{Page 36}

\raggedright

\textit{Os campos das forças internas}

A tendência dinâmica de integrar os impactos ópticos em um todo equilibrado e unificado age dentro do campo da constituição fisiológica e psicológica do homem. As forças que impulsionam a restauração do equilíbrio no organismo humano são forças nervosas, e o sistema nervoso, assim como o plano da imagem, é limitado. Assim como as limitações da superfície da imagem servem como o referencial necessário na transformação de impactos ópticos em forças espaciais, as características dos mecanismos fisiológicos e psicológicos servem como fatores condicionantes para experimentar forças de integração, ou seja, transformar forças espaciais em forças plásticas.

\vspace{1em}

\textit{O campo fisiológico}

O equilíbrio nas dimensões fisiológicas é condicionado pelas limitações do olho.

"Pois o olho possui todos os defeitos possíveis que podem ser encontrados em um instrumento óptico, e até mesmo alguns que lhe são peculiares; mas todos são tão contrariados, que a imprecisão da imagem resultante de sua presença excede muito pouco, sob condições ordinárias de iluminação, os limites estabelecidos para a delicadeza da sensação pelas dimensões dos cones da retina. Mas assim que fazemos nossa observação sob condições um tanto alteradas, tomamos consciência da aberração cromática, do astigmatismo, do ponto cego, das sombras venosas, da transparência imperfeita dos meios e de todos os outros defeitos dos quais falamos."

\textbullet \ H. W. Helmholtz, Óptica Fisiológica

O olho é construído de tal forma que pode focar imagens em apenas uma área muito pequena do campo retiniano. Para obter dados focados suficientes, ele deve girar na área das coisas a serem percebidas. Essa limitação é a base de numerosos movimentos musculares delicados que são registrados como sensações e interpretados como sinais espaciais. O olho, focando a luz vermelha de uma distância infinita na retina, focará, ao mesmo tempo, os raios violeta de uma distância de dois pés. Se diferentes superfícies coloridas estiverem à mesma distância dos olhos, como em um plano de imagem bidimensional, um ajuste muscular é necessário para trazer os diferentes raios sucessivamente para o foco. Esses ajustes são registrados como sensações diferentes e, associando suas qualidades às respectivas cores, criam sensações espaciais.

34

\newpage
\section*{Page 37}

\section*{Balanço de cores}

Assim como todas as outras partes do corpo humano, o olho tem um limite natural de capacidade de trabalho. Após um certo desempenho, ele se fadiga. Mas, como já foi salientado, as forças fisiológicas visam a um equilíbrio para manter os sistemas essencialmente constantes. ``Toda cor definida exerce uma certa violência sobre o olho, e força o órgão à oposição,'' disse Goethe. Após cada fadiga do aparelho ocular, após cada destruição de certas substâncias fotoquímicas na retina, tornando-a insensível a uma ou outra cor, há uma tendência dinâmica para recuperar a ordem original, a sensibilidade plena, um balanço fisiológico.

``Se olharmos fixamente para um objeto de cores brilhantes por alguns segundos e depois desviarmos o olhar para uma superfície iluminada, de preferência branca, veremos uma mancha de cor diferente, que flutua diante de nossos olhos e se move conforme os giramos. A explicação é que os receptores na retina não estão igualmente fatigados. Se estivéssemos olhando para o vermelho, o receptor vermelho estaria mais fatigado do que o outro, de modo que, quando a luz branca é lançada no olho, os outros dois receptores são mais plenamente estimulados e um verde-azulado aparece.''

``Se permitirmos a entrada de luz de um céu nublado através de uma pequena abertura em um quarto escuro, de modo que ela incida lateralmente sobre uma folha de papel branco na horizontal, enquanto a luz de vela incide do outro lado, e se então segurarmos um lápis verticalmente sobre o papel, ele, naturalmente, projetará duas sombras: a feita pela luz do dia será laranja, e assim parece; a outra feita pela luz de vela é realmente branca, mas aparece azul por contraste. O azul e o laranja das duas sombras são ambas cores que chamamos de branco, quando as vemos à luz do dia e à luz de vela, respectivamente. Vistas juntas, elas aparecem como duas cores muito diferentes e toleravelmente saturadas, mas não hesitamos um momento em reconhecer o papel branco à luz de vela como branco, e muito diferente do laranja.''

Essa ``pós-imagem,'' sombras coloridas, e vários outros fenômenos visuais, contraste de borda e contraste simultâneo, apontam para as características mais significativas das superfícies coloridas como forças plásticas. Através de uma interação dinâmica das superfícies no plano da imagem, as cores tendem a uma relação equilibrada em termos de sensibilidade retiniana plena. Cada tonalidade induz simultaneamente ou em sucessão sua parte complementar respectiva, retornando à sua origem, que é a luz branca. Grosso modo, uma superfície vermelha induzirá um verde-azulado, um certo azul-violeta criará um amarelo complementar, o laranja realça o azul-turquesa. A harmonia de cores complementares---a lei mais universalmente aceita do balanço plástico---tem aqui sua fundação. Embora haja pequenas disparidades na interpretação do que são tonalidades opostas, o automatismo das cores em pinturas infantis, as expressões de cores de quase todas as culturas, a pesquisa científica dos grandes pensadores---Leonardo, Goethe, Schopenhauer, Chevreul, Ostwald---testificam a validade universal da harmonia das cores como conformidade à lei do organismo humano, um equilíbrio na experiência visual.

Superfícies de cores adjacentes modificam-se mutuamente em matiz e brilho.

\begin{itemize}
    \item H. Helmholtz, \textit{Progresso Recente da Teoria da Visão}, 1868
    \item Sir W. Bragg, \textit{O Universo da Luz}
\end{itemize}

\vspace{\baselineskip}
\hfill 35

\newpage
\section*{Page 38}

\subsection*{{\itshape Tensão espacial; equilíbrio dinâmico}}

Uma superfície bidimensional sem qualquer articulação é uma experiência morta. A base de todo processo vivo é uma contradição interna. A qualidade vital de uma imagem é gerada pela tensão entre as forças espaciais; isto é, pela luta entre a atração e a repulsão dos campos dessas forças.

A experiência do espaço, como já vimos, baseia-se no movimento virtual das diferentes unidades ópticas do plano da imagem. Esses movimentos só podem ser percebidos se o referencial, o plano da imagem bidimensional, for evidente; não se podem ver coisas em movimento sem um fundo. Considerada isoladamente, cada força óptica interrompe automaticamente a qualidade bidimensional da superfície da imagem e completa seu movimento virtual induzindo seu próprio campo. Torna-se impossível, portanto, perceber o fator dinâmico desse movimento. Aqui está um ponto importante. Assim como qualquer força só pode ser manifestada através da resistência a uma força oposta, as forças espaciais só podem ser percebidas ao encontrarem forças espaciais opostas. Um posicionamento aleatório de forças espaciais, ponto, linha, área, abrirá o plano da imagem, mas como essas forças estão tão dispostas aleatoriamente, elas não alcançarão uma constelação equilibrada em que sejam iguais em força e opostas em direção. A superfície da imagem é feita oca; o fundo bidimensional, o referencial em que os movimentos espaciais podem ser medidos, está ausente. A vitalidade espacial não pode atingir sua plena maturidade.

Se as forças e seus campos induzidos forem de igual qualidade óptica e força espacial, um equilíbrio será alcançado, mas será sem tensão, estático e sem vida. Se, no entanto, soubermos estimar as forças e seus campos de energia, seremos capazes de usar esses campos opostos de modo que cada um equilibre o outro no plano da imagem. Uma linha ou forma em uma certa cor e posição terá um campo aberto e avançando em direção ao espectador; outra unidade criará um campo na direção de recuo; outra ativará um campo tendendo para cima na superfície; e ainda outra, para baixo. Esses movimentos podem ser diferentes em termos de suas medidas e qualidades ópticas --- isto é, opostos em direção, peso, intensidade --- mas, se forem iguais em força em termos de seus campos espaciais, um equilíbrio dinâmico será alcançado na superfície da imagem.

Se alguém vê numa balança uma libra de ferro equilibrando uma libra de penas, fica-se envolvido na experiência devido à aparente contradição óptica na lógica. É-se forçado a pensar sobre a natureza dos materiais opostos e a compreender suas relações futuras. A visão de um adulto equilibrado por uma criança pequena numa gangorra, devido às suas diferentes distâncias do centro de gravidade, induz uma experiência similar. Esse equilíbrio dinâmico também pode ser simplesmente ilustrado amassando um pedaço de papel na mão. Deitado, o papel está inerte, morto. Ao ser comprimido, adquire uma qualidade cinética. Mão e papel exercem campos de força fortes que se opõem um ao outro, mas que, no entanto, estão em equilíbrio.

\newpage
\section*{Page 39}

O ponto da retina no eixo óptico do olho proporciona acuidade visual. O centro geométrico do campo da imagem é costumeiramente identificado com o eixo óptico e é assumido como o centro de gravidade de todas as forças agindo na superfície da imagem.

\vfill
\hfill 37

\newpage
\section*{Page 40}

\textit{Harold Walter. Colagem 1939}

\textit{Kasimir Malevich. Composição Suprematista}

\textit{Ruth Robbins. Estudo de equilíbrio dinâmico em valores de textura. Trabalho realizado para o curso do autor em Fundamentos Visuais. Escola de Design em Chicago.}

38

\newpage
\section*{Page 41}

39

\newpage
\section*{Page 42}

Jean Helion. Linogravura 1932.

\newpage
\section*{Page 43}

El Lissitzky. \textit{Proun} 1923

amigo no tribunal

Lester Beall. \textit{Design de Publicidade}

\newpage
\section*{Page 44}

\noindent
\textit{Piet Mondrian. Composição com Vermelho 1936}\\
\textit{Cortesia do Museu de Arte da Filadélfia, Coleção A. E. Gallatin}

\vfill

\centering
42
\par

\newpage
\section*{Page 45}

Jan Tschichold. Design Tipográfico

\vspace*{10em} % Espaçamento vertical para empurrar o bloco de conteúdo para baixo

\begin{center}
\textit{Jan Tschichold:} \\ % Fonte similar a manuscrita simulada com itálico
\vspace{1em} % Espaço entre as duas linhas no bloco central
\textbf{\Large Design Tipográfico} % Título principal, em negrito e grande
\end{center}

\vspace*{\fill} % Empurra o conteúdo seguinte para o final da página

\raggedright
Benno Schwabe \& Cia. Basileia 1935%
\hfill
43
\par

\newpage
\section*{Page 46}

\centering
\textbf{\large O campo psicológico}
\par
\noindent Forças subjetivas tendendo ao equilíbrio se manifestam nas faculdades emocionais e intelectuais, bem como no nível fisiológico. A imagem plástica é um organismo que se estende às dimensões da compreensão para além do raio sensorial. O fascínio de um pôr do sol ou um nascer do sol, o interesse irresistível despertado pelas formas e cores sempre em mudança das chamas, ou os padrões rítmicos e reflexos das ondas na água têm um significado revelador. Nunca nos cansamos dessas transformações ópticas que, apesar de todas as variações, mantêm sua unidade. Seguimos as inúmeras mutações sem qualquer sentimento de compulsão. O aspecto significativo de tais experiências é que elas mobilizam respostas, pensamentos e sentimentos mais amplos, não diretamente conectados com a imagem real vista. Assim como um menino é capaz de mover um grande sino de igreja pela adição rítmica de um puxão a outro, a ordem rítmica de chamas ou ondas pode induzir dimensões de experiência cada vez maiores e mais amplas. Da percepção de padrões sensoriais, passa-se a estruturas correspondentes nos domínios emocionais e intelectuais. A experiência torna-se completa. Para alcançar o equilíbrio nesta dimensão mais ampla, as bases dinâmicas, o alcance espaço-temporal da experiência plástica devem ser assegurados.

A tendência dinâmica de organizar as forças ópticas em um todo unificado atua dentro do campo psicológico em um contexto de prontidão para perceber um campo de atenção. A atenção, contudo, sofre de duas limitações: primeiro, sua limitação no número de unidades ópticas que pode abranger; e segundo, sua duração limitada no tempo de foco em uma situação óptica. E assim como as limitações do campo bidimensional da imagem servem como um quadro de referência necessário para as localizações virtuais das unidades ópticas, as limitações do campo psicológico servem como condição necessária às leis da organização plástica.

\vspace{2em}

\centering
\textbf{\large A extensão espacial da organização plástica}
\par
\noindent Para ser vista de uma forma puramente sensorial, uma unidade óptica deve incidir na pequena região retiniana de visão nítida; para ser compreendida, a imagem deve incidir dentro do campo limitado do ato atencional. ``O processo de organização visual pode ser considerado como uma figura contra o fundo do campo da consciência. No campo geral embaçado, forma-se uma área de clareza e intensidade --- o campo de atenção.'' Dentro deste campo de atenção, pode-se ver claramente, e de uma só vez, apenas um número limitado de unidades visuais. A extensão desta vividez e clareza é determinada pela energia do ato atencional. Esta energia atencional é suficiente para apreender e relacionar apenas um número limitado de unidades ópticas. De fato, apenas cinco ou seis elementos opticamente distintos podem ser vistos juntos claramente em suas características e relações individuais.

\vfill
\raggedright
44

\newpage
\section*{Page 47}

Confrontado com um campo óptico complexo, um o reduzirá a inter-relações básicas. Assim como na natureza há uma tendência a encontrar a unidade de superfície mais econômica em cada formação, assim na organização visual há uma tendência a encontrar a unidade espacial mais econômica na ordenação das diferenças ópticas. Diante da agitação dos impactos ópticos, a primeira reação é formar no menor intervalo de tempo a maior extensão espacial possível.

Certas características ópticas tendem a ser vistas juntas como uma configuração espacial. Ao olharmos para uma tela de meio-tom bastante ampliada, o que realmente vemos são diferentes tamanhos de pontos pretos e diferentes intervalos brancos. Mas instantaneamente organizamos e agrupamos essas diferenças visíveis. Algumas unidades de pontos pretos são vistas de uma forma; outras de outra. Alguns elementos são vistos juntos porque estão próximos uns dos outros; outros estão ligados porque são semelhantes em tamanho, direção, forma. Somente após esta organização instantânea ser alcançada é que se pode ver a semelhança da imagem com um olho humano.

Esta organização de pertencimento óptico é mais básica do que o reconhecimento dos próprios objetos. Os numerosos dispositivos ópticos que a natureza emprega no mundo animal para esconder animais de seus inimigos revelam o funcionamento desta lei de organização visual. Uma camuflagem de cobra. É uma agregação de pequenas unidades de cor-forma. Como a afinidade de qualidades visuais elementares é mais fundamental para a construção da imagem do que as relações da experiência empírica, os padrões em seu corpo são mais facilmente vistos junto com os padrões correspondentes em seu fundo do que sua forma --- conhecimento que é adquirido em outras experiências. A cobra desaparece em seu fundo.

\newpage
\section*{Page 48}

\subsection*{Proximidade}

Proximidade é a condição mais simples de organização. Ouvimos palavras em coerência verbal, principalmente devido à proximidade temporal de seus elementos sonoros. Lemos palavras como totalidades segregadas porque suas letras estão mais próximas umas das outras do que a última e a primeira letra de duas palavras. Formamos as estrelas do firmamento em uma variedade de padrões, principalmente devido à sua proximidade relativa umas das outras. De modo geral, a distância relativamente mais próxima entre as unidades sensoriais oferece a menor resistência à sua interconexão, e assim possibilita o início da cristalização em uma forma estável. No campo da experiência visual, a proximidade das unidades ópticas também é a condição mais simples para a cristalização de "totalidades" visuais unificadas. Articulamos uma pintura, um design tipográfico, antes de tudo pela lei da proximidade. Unidades ópticas próximas umas das outras em um plano pictórico tendem a ser vistas juntas e, consequentemente, podem ser estabilizadas em figuras coerentes.

Uma ilustração tirada de K. Koffka pode elucidar a lei da proximidade. Duas linhas paralelas são percebidas como uma unidade se estiverem próximas o suficiente. Como o espaço entre elas é fechado, ele parece separado do espaço circundante. Se adicionarmos mais duas paralelas fora das duas primeiras, a figura formada pelo intervalo entre elas perde sua qualidade de todo coerente e serve apenas como pano de fundo para as duas novas unidades.

\section*{organização espacial é o fator vital em uma mensagem óptica}

\section*{sp atialor gani sationist hevital fa ctorin an optical message}

\section*{organização espacial é o fator vital em uma mensagem óptica}

46

\newpage
\section*{Page 49}

\footnotesize
\textit{Análise da tensão visual criada por direções organizacionais opostas.}

\vspace{1em}

\textit{Detalhe de Composição de Piet Mondrian 1915.}
\normalsize

\vspace{2em}

\raggedright

\textbf{\textit{Similaridade ou igualdade}}

\vspace{1em}

A proximidade, no entanto, pode ceder a outros fatores de organização. Também ligamos elementos em relações estáveis se eles possuem qualidades comuns. Tamanhos iguais, formas semelhantes, direções, cores correspondentes, valores, texturas também produzem a tendência dinâmica de serem vistos juntos. Proximidade e similaridade, como fatores na criação da estrutura espacial, devem ser consideradas em conjunto. Pois unidades formadas por proximidade podem ser desfeitas pela similaridade de seus elementos com outros elementos à distância, e unidades formadas por similaridade podem ser desfeitas pela proximidade extrema de elementos externos. Essa competição é importante para o organismo plástico, pois a direção oposta das organizações pode trazer uma tensão vital à experiência plástica. Considerando-os em conjunto, ``grupos de similaridade'' parecem ser mais unificados do que ``grupos de proximidade''.

\vfill
\raggedleft
47

\newpage
\section*{Page 50}

Theo Van Doesburg. \textit{Pintura}\\
Cortesia do Museu de Arte Moderna

\newpage
\section*{Page 51}

\section*{Continuidade}

\hfill R. J. Wolff. Pintura 1941

Toda unidade linear possui inércia cinética. Ela tende a ser continuada na mesma direção e com o mesmo movimento. Uma linha reta tende a ser vista em sua continuação como uma linha reta; uma linha curva como uma linha curva; uma linha ondulada com repetição contínua de seu ritmo original. Tal continuação linear ajuda a formar a imagem criando grupos de uma ordem simples. É o dispositivo mais potente para unir elementos heterogêneos e, assim, reduzir a imagem-quadro ao número de unidades que podem ser totalmente compreendidas em um único ato de atenção.

A lei da continuidade também é válida para a gradação ou progressão de matiz, valor e croma. O olho se move ao longo de uma direção de gradação de matiz ou valor de forma semelhante à maneira como se move ao longo de uma linha.

\vfill
\raggedleft 49

\newpage
\section*{Page 52}

\noindent\textit{Picasso. Desenho para uma Crucificação 1932}\par
\noindent\textit{Reprodução Cortesia}\par
\noindent\textit{O Instituto de Arte de Chicago}\par
\par
\par
\noindent\textit{L. Moholy Nagy. Escadaria 1936}\par
\par
\par
\par
\noindent\textit{L. Moholy Nagy. Fotograma 1941}\par
\par
\par
\par
\centering 50

\newpage
\section*{Page 53}

\section*{Fechamento}

Forças de organização impulsionando em direção à ordem espacial, à estabilidade, tendem a moldar unidades ópticas em totalidades compactas fechadas. Confrontado com uma situação óptica complexa, o observador busca a forma com a unidade mais estável, ou com as relações menos perturbadas com o ambiente. Goethe observou que a imagem posterior de um quadrado nítido gradualmente se torna arredondada em uma forma circular. Assim como uma gota de água tende a adaptar sua forma à superfície mais econômica, uma unidade óptica tende a formar o fechamento mais econômico, segregando-se o mais completamente possível de seus arredores. Uma área fechada parece mais formada, mais estável, do que uma que é aberta e sem limites. Um preenchimento psicológico dos intervalos entre as unidades ocorre, e conexões latentes são construídas. Este fator de fechamento pode atuar na dimensão plana, gerando a partir de unidades lineares abertas a experiência de uma forma fechada, mas também pode unificar outras dimensões. Certas interconexões latentes de pontos, linhas, formas, cores e valores são fechadas psicologicamente em totalidades bidimensionais ou tridimensionais. O fator de fechamento pode ser mais significativo do que a proximidade ou a similaridade.

Devido às leis da organização visual, nenhuma unidade visual pode existir em si mesma no plano da imagem. Cada unidade vai além de si mesma e implica um todo maior. Assim, as unidades não apenas vivem no plano da imagem; elas também crescem. Elas se fundem em totalidades com uma função comum. Três tons musicais têm cada um seu comprimento de onda particular, sua qualidade tonal individual; mas quando os três são tocados juntos, suas características individuais recuam e algo inteiramente novo aparece --- o acorde. Da mesma forma, as unidades ópticas organizadas em configurações espaciais tornam-se mais do que a soma total de suas partes componentes. Essas totalidades maiores formam, com outros grupos, uma unidade ainda mais abrangente, e esse processo continua até que todas as relações possíveis sejam esgotadas; ou seja, até que o limite de atenção seja atingido. Essa lei de organização implica, então, que o aumento numérico de elementos não leva necessariamente a uma perda de ordem do todo da imagem. Uma superfície de imagem uniforme é plana. Um aumento gradual dos elementos nessa superfície mostra claramente que, a cada adição ao número ou qualidade das unidades, uma unidade espacial pode ser mantida. Atingindo o limite numérico da organização, unidades anteriormente separadas, em uma espécie de salto revolucionário, formam uma figura comum --- e, assim, uma nova condição para a organização de um todo mais abrangente. O número de unidades pode ser aumentado desde que não interfiram, formando unidades adicionais. Mas quando este ponto de saturação é atingido, não há mais oportunidade para a organização plástica. Uma uniformidade de superfície é produzida em um novo nível.

\newpage
\section*{Page 54}

\begin{center}
\large\textit{A duração da imagem plástica}
\end{center}

As limitações do nosso sistema nervoso definem não apenas o número e a extensão das unidades ópticas individuais que podem ser percebidas como um todo, ou seja, a extensão espacial, mas também a duração da experiência visual. Não se pode olhar para uma relação estática por muito tempo sem perder o interesse, da mesma forma que não se pode sobreviver por muito tempo em um quarto selado onde o suprimento de oxigênio se esgota rapidamente. A imagem como uma experiência viva não pode subsistir por muito tempo em uma estrutura congelada. Para que a imagem permaneça um organismo vivo, as relações dentro dela devem estar em constante mudança. O olho e a mente devem ser alimentados com relações visuais em constante mudança. Apenas essa variedade em constante mudança pode fornecer a estimulação necessária para manter a atenção na superfície da imagem. A mudança implica movimento. A imagem plástica deve ser também articulada, portanto, na dimensão do tempo. O objetivo último da organização plástica é uma estrutura de movimento que dita a direção e a progressão em direção a relações espaciais sempre novas até que a experiência atinja sua saturação espacial mais completa. À medida que novas relações se desdobram progressivamente, a integração espacial da imagem ganha impulso até encontrar sua clarificação final na imagem plástica como um todo. Tal movimento é definido e condicionado por limitações fisiológicas e psicológicas. Como o movimento é basicamente um movimento ocular, a compreensão do papel condicionante da estrutura neuromuscular é de grande importância. No entanto, a direção do interesse é o que liga uma unidade a outra. O alcance último de uma imagem criada é definido pelas energias disponíveis de atenção.

\vspace*{\fill}
\centering 52

\newpage
\section*{Page 55}

\noindent\textit{\textbf{Organização da sequência óptica. Ritmo}}

A visão é o processo de trabalho do olho. Organizar a imagem significa medir e relacionar diferenças visíveis---matiz, valor, saturação, textura, posição, forma, direção, intervalo, tamanho---pela ação neuromuscular do olho. A atividade ininterrupta esgota as energias nervosas. O olho, ao trabalhar, precisa tanto de ação quanto de repouso. O equilíbrio dos componentes complementares deve ser reconhecido. Sem ele, o gasto de energia, medido incorretamente, leva à fadiga. Com o esforço correto, com um ritmo reconhecido, o trabalho é feito economicamente e com maior resistência. Cortar lenha, martelar, nadar, remar, andar, correr, dançar são atividades familiares nas quais o ritmo facilita o trabalho e, ao mesmo tempo, o dota de uma sensação de prazer. A proporção de ação e repouso---isto é, o ritmo---depende da natureza do trabalho. A repetição ordenada ou alternância regular de similaridades ou igualdades ópticas dita o ritmo da organização plástica. Ao reconhecer tal ordem, aprende-se quando a próxima ação ocular é devida e qual ajuste neuromuscular particular será necessário para apreender a próxima unidade. Para conservar as energias atencionais da visão, portanto, a superfície da imagem deve ter uma estrutura temporal de organização---deve ser ritmicamente articulada de uma forma que corresponda, para o olho, ao ritmo de qualquer processo de trabalho.

Mas o significado do ritmo vai muito além do material bruto do prazer na economia de energia. O ritmo não pode ser compreendido como uma sensação visual isolada. Seu próprio significado reside no fato de que é uma ordem de um todo temporal maior. A economia de energias mentais ao julgar as medidas fisiológicas necessárias só faz sentido em referência a todo o processo de construção da imagem.

Uma vez que um ritmo de acento e pausa é reconhecido, uma unidade dinâmica é formada, uma ordem que liga o tempo. Cima e baixo, esquerda e direita, reto e curvo, claro e escuro, pequeno e grande, condensação e rarefação, e outras características ópticas são ligadas por uma medida comum em uma sucessão orgânica que sustenta a atenção em um fluxo contínuo até que todas as relações evoluam para uma unidade. Plotino expressou essa verdade corretamente há muito tempo. Ele escreveu:

\begin{quote}
``O que é que, quando você olha para algo, te impressiona, te atrai, te cativa e te enche de alegria? Estamos todos de acordo, posso dizer, que é a inter-relação das partes umas com as outras e com o todo, com o elemento adicional de beleza na cor, que constitui a beleza tal como percebida pelo olho, em outras palavras, que a beleza nas coisas visíveis, como em tudo o mais, consiste em simetria e proporção.''
\end{quote}

\vspace*{\fill}
\raggedleft
53

\newpage
\section*{Page 56}

\raggedright

1. \textit{Estrela de Pitágoras}
2. \textit{A Proporção Divina}
3. \textit{Estrela de Pitágoras em}
   \textit{Pentagrama Regular}

\textit{Tecido Peruano.} \\
\textit{Coleção do Museu de Arte Fogg} \\
\textit{Cortesia de} \\
\textit{A Escola de Design de Rhode Island}

O padrão rítmico da superfície da imagem pode existir em tantos níveis quanto as diferenciações do campo visual. Se uma superfície permite qualquer subdivisão que repete sua própria forma ou tamanho em uma forma menor, uma ordem geométrica simples é alcançada. Essa subdivisão implica tamanhos, posições, direções e intervalos. Nesse nível pode haver ritmo através de formas, posições, comprimentos, ângulos, curvas, direções, intervalos. Quando a medida ordenada das unidades ópticas está relacionada ao seu movimento virtual de e para o plano da imagem, um nível superior de ritmo é alcançado. Temos então um ritmo das forças plásticas, uma mudança regular de sensação de movimentos espaciais de cores e valores; avançando, recuando, expandindo, contraindo, movendo para cima, para baixo, para a esquerda e para a direita. Finalmente, poderíamos ter mudanças ordenadas ou repetição de configurações mais complexas de experiência visual; ordem rítmica de tensão e repouso, concentração e rarefação, harmonia e discórdia. O ritmo pode ser simples, restrito a um ou outro metro das diferenças ópticas. Também pode ser composto, como dois ou mais metros legalmente variáveis existindo simultaneamente. Os ritmos podem corresponder e se amplificar, ou podem se opor, causando um nível superior de configuração rítmica.

Quase não há uma cultura na qual o ritmo visual não tenha sido concebido de uma forma ou de outra. No passado, no entanto, o principal interesse tem se concentrado em uma escala estática de proporção geométrica. O ritmo não era entendido como um resultado orgânico da organização sensorial dinâmica, mas era considerado a representação de certos metros absolutos observados na natureza visível ou derivados por especulações matemáticas. Certas proporções observadas no corpo humano, em formações cristalinas, em folhas, foram emprestadas e usadas, através de subdivisões correspondentes da superfície da imagem. Ritmos de crescimento e função alheios ao crescimento e função da organização visual foram "congelados" na superfície da imagem. O ritmo, em termos de medida geométrica apenas, negligenciou a qualidade dinâmica da experiência visual, os movimentos das forças plásticas.

54

\newpage
\section*{Page 57}

Houve, entretanto, exceções. A tapeçaria peruana antiga foi concebida com um saudável respeito pelo ritmo intrínseco ao processo dinâmico de organização visual. Por meio de uma cuidadosa troca de linhas, formas, cores, o ritmo do tempo é traduzido em espaço. Esses desenhos são descritos por Franz Boas, que diz:

\begin{quote}
``Em muitos tecidos encontramos padrões que consistem em um arranjo diagonal de quadrados ou retângulos. Em cada diagonal o mesmo desenho é repetido, enquanto a diagonal seguinte tem outro tipo. Em cada linha diagonal o desenho é exibido em posições variadas. Se um estiver voltado para a direita, o próximo está voltado para a esquerda. Ao mesmo tempo, há uma alternância de cores, de modo que mesmo quando a forma é a mesma, os tons e os valores das cores não serão os mesmos.''
\end{quote}

\vspace*{\fill}
\centering 55

\newpage
\section*{Page 58}

\begin{center}
Seurat. \textit{Le Chahut} 1890 \\
Cortesia da Galeria de Arte Albright, Buffalo
\end{center}

Após a abordagem estéril e estática à ordenação rítmica da superfície da imagem, Seurat, no século passado, trouxe o ritmo de volta a um nível dinâmico. Ele uniu formas, direções, cores e tamanhos em uma unidade rítmica através de uma interação cuidadosamente planejada de direções horizontais, verticais, linhas retas e curvas, e os movimentos de avanço e recuo das cores.

\vspace{\fill}
\raggedright
56

\newpage
\section*{Page 59}

\textit{Seurat. Detalhe de Side Show}

Cortesia do Museu de Arte Moderna

\newpage
\section*{Page 60}

Theo Van Doesburg\\
Composição\\
Cortesia de\\
Art of This Century

Sophie Taeuber-Arp. Composição 1931

Mondrian e Doesburg levaram este princípio dinâmico de ritmo à purificação final e máxima intensidade. Ao reduzir a superfície da imagem aos opostos básicos --- cores puras, formas elementares e direções horizontais e verticais --- eliminando qualquer semelhança com o mundo dos objetos familiar, como Mondrian escreve, a arte hoje conseguiu estabelecer uma expressão plástica, "a clara realização do ritmo libertado e universal distorcido e oculto no ritmo individual da forma limitante."

A invenção do cinema abriu caminho para um escopo e flexibilidade de organização rítmica até então inimagináveis. As novas possibilidades de sincronização da estrutura temporal e espacial da visão são, no entanto, ainda pouco exploradas. Dentre os poucos pioneiros que abordaram os problemas, Viking Eggeling e Hans Richter fizeram as primeiras e mais importantes clarificações práticas e teóricas. Eggeling apontou para o cerne de toda a organização visual quando escreveu: "O que deve ser apreendido e dado forma são coisas em fluxo."

58

\newpage
\section*{Page 61}

\subsection*{Organização da progressão espacial. O espaço equívoco}

A organização rítmica, embora uma condição essencial para manter a atenção e assim prolongar a vida útil da imagem, não é em si mesma totalmente suficiente para assegurar a máxima duração da atenção necessária para a integração de uma forma plástica. Conhece-se bem a sensação irritante produzida pela repetição regular do som de um tambor. Sabe-se quase instintivamente que um padrão rítmico simples possui uma regularidade que logo se torna monotonia. Se a imagem deve permanecer um organismo vivo, as relações dentro dela devem ter aspectos progressivamente mutáveis. Não se pode olhar por muito tempo para a mesma relação visual sem esgotar as energias nervosas da atenção. O poder do ritmo em manter a atenção prolongada é condicionado pela necessidade de alimentar a atenção através da mudança progressiva do material óptico. Mudança implica movimento. A tarefa final da organização plástica é, então, a criação de uma estrutura óptica de movimento que ditará a direção e a progressão das relações plásticas até que a experiência atinja a integração total. A característica mais evidente do movimento é a sua unidade, a sua continuidade dinâmica. O movimento, porém, implica também o oposto da unidade: variedade de localizações. O próprio significado do movimento reside nesta contradição interna da unidade dinâmica e da descontinuidade estática. Experimentar o movimento, então, significa revelar os seus aspectos contraditórios, estabelecer as suas relações mútuas, seguir a contradição através de todas as etapas. O campo da imagem é uma superfície bidimensional e as mudanças ópticas, portanto, devem necessariamente estar dentro da circulação da visão no plano. A base cinética da organização plástica---os caminhos lineares do olho no plano da imagem---é a medida comum que liga numa unidade as relações plásticas em mudança. O olho segue o caminho dado, e a sensação cinética do movimento ocular carrega a linha com a sua própria qualidade de experiência e estabelece uma continuidade dinâmica, uma unidade da superfície.

A função do caminho linear cinético na organização plástica pode ser comparada com a função da melodia na composição musical, e as seguintes observações de músicos devem ser úteis para proporcionar maior clarificação.

\begin{quote}
``A música, teoricamente considerada, consiste inteiramente em linhas de tom. Assemelha-se mais a um quadro ou a um desenho arquitetônico do que a qualquer outra criação artística; a diferença é que num desenho as linhas são visíveis e constantes, enquanto na música são audíveis e em movimento. Os tons separados são os pontos através dos quais as linhas são desenhadas; e a impressão que se pretende, e que é apreendida pelo ouvinte inteligente, não é a de tons únicos, mas de linhas contínuas de tons, descrevendo movimentos, curvas e ângulos, subindo, descendo, equilibrando-se---diretamente análogas às impressões lineares transmitidas por um quadro ou desenho.''
\end{quote}

\textbullet{} P. Goetschius, \textit{Elementary Counterpoint}

\raggedleft 59

\newpage
\section*{Page 62}

\textit{Diagrama linear de uma}

\textit{Pintura de Juan Gris}

Essa unidade linear pode abranger todos os possíveis opostos ópticos em todos os níveis do espaço e pode ser gerada por qualquer fator de organização visual. Quando as forças plásticas falham em criar a experiência de profundidade, o movimento linear organizará formas planas. Não só cada forma tem a sua própria individualidade, mas simultaneamente os contornos das formas possuem poder dinâmico para guiar o olho de uma para outra. Uma série de formas contíguas colocadas no plano da imagem são automaticamente conectadas pelo movimento dos seus contornos contínuos. Essas linhas movem-se primeiro de uma figura para outra, formando grupos, e depois de um grupo para outro, criando uma organização progressiva de todos os elementos na superfície da imagem. A direção linear dinâmica comum tem, portanto, um significado equívoco. Cada contorno de forma compartilha a direção do fluxo visual. Este conteúdo espacial ambíguo tem a sua vitalidade aumentada quando novas qualidades são adicionadas às formas. Se forem introduzidas diferenças de valor onde anteriormente havia uniformidade de valor, se uma forma for feita preta e outra branca, as restantes de valores intermediários, o caráter duplo do espaço será mais evidente. Uma forma parecerá avançar em direção ao espectador, outra a recuar, e assim por diante, mas a continuação evidente ou latente do contorno continua a mover-se na superfície plana e a contradição e identidade entre as dimensões de profundidade e o plano da imagem ganham vida. Essa característica dupla pode ser ainda mais ampliada pela adição de cor, textura e outras qualidades espaciais e pela indicação ilusória de forma ou ação.

As direções do fluxo visual na superfície também podem ser indicadas de maneiras mais sutis. Um tipo de preenchimento psicológico dos intervalos ópticos fornecerá linhas latentes capazes de desempenhar o mesmo papel de organização que as linhas reais para formas que não possuem linhas comuns. De acordo com a lei do fechamento, intervalos de cores e valores podem emergir em formas, intervalos de linhas em formas, intervalos de pontos em linhas, gerando novas figuras com novos contornos cinéticos.

O estudo de certas situações ópticas onde a atenção flutua entre figura e fundo, e onde cada um por sua vez emerge como a figura ou o fundo, torna evidente que não há diferença fundamental, em um sentido óptico, entre a figura e o fundo, entre o espaço positivo e negativo. O movimento linear baseia-se não apenas na atividade das linhas existentes ou contornos de figuras, mas também nos contornos latentes dos intervalos entre essas figuras.

Os contornos cinéticos das figuras gerados pelos intervalos ópticos constituem parte integrante da organização plástica. Eles vivem, agem e movem-se com o mesmo poder cinético que as linhas e formas intencionalmente criadas.

60

\newpage
\section*{Page 63}

\noindent\textit{Carlotta Corpron. Volumes Leves}\par
\noindent\textit{Continuidade Linear de Gradações de Valor}\par
\par % Vertical separation for visual consistency with the page number placement
\centerline{61}

\newpage
\section*{Page 64}

62

\newpage
\section*{Page 65}

\textbf{Picasso. Mulher Sentada 1927}

\textit{Reprodução Cortesia do Art Institute de Chicago.}

\bigskip

\textbf{Sano Di Pietro. Madonna e Menino}

\textit{Cortesia do Museu de Arte do Smith College}

\bigskip

O movimento plástico pode ser repetido em várias qualidades visuais, como cor, valor tonal, textura, forma e assim por diante. O olho, ao passar de uma estimulação sensorial para outra, recebe um ímpeto acumulado que o leva a abraçar novas relações na superfície da imagem.

A música sugere uma excelente analogia. Uma unidade musical tocada por um instrumento é repetida contrapontisticamente em outros instrumentos, nas cordas, nos metais, nas madeiras, mesmo nos instrumentos de percussão. Cada unidade plástica, com sua qualidade sensorial específica, ecoa a anterior; luz, sombra, cor, formas, tudo mutuamente ajuda umas às outras, uma assumindo o movimento onde a outra parou, conduzindo à unidade completa.

``Os grupos de tons em uma melodia que estão harmonicamente conectados são como os elos de uma corrente; eles dão cor e brilho à melodia. Eles são o verdadeiro corpo da melodia, por mais estranho que possa parecer falar de corpo em conexão com um fenômeno linear como uma melodia. Não se deve esquecer que uma melodia é apenas primariamente linear, e que a comparação com uma linha curva se aplica apenas ao aspecto mais óbvio e externo de uma cadeia de tons. O fio melódico tem um volume ou espessura sempre mutável, mas sempre presente.''

\textit{*Hindemith, Craft of Musical Composition*}

\vfill
\raggedleft 63

\newpage
\section*{Page 66}

Nicolas Poussin. \textit{Desenho}

O melhor exemplo de orquestração espacial por movimento dentro das limitações das representações de objetos clássicos foi alcançado por Poussin.

Nicolas Poussin. \textit{Triunfo de Baco}
Cortesia da The William Rockhill Nelson Gallery of Art

\vspace*{\fill}
\begin{center}
64
\end{center}

\newpage
\section*{Page 67}

\begin{center}
II. Representação Visual
\end{center}

\newpage
\section*{Page 68}

No capítulo anterior, consideramos a imagem plástica como um organismo vivo, as leis do seu crescimento e estrutura. Tal organismo vivo está enraizado na natureza e depende da natureza visível para o seu alimento. Para uma compreensão mais completa da imagem, do seu papel e da sua eficácia, devemos aprender a sua relação com as características visíveis do nosso ambiente em constante mudança.

O homem é um ser dinâmico lutando individual e socialmente pela sobrevivência. Para sobreviver, ele deve se orientar em relação ao seu ambiente. Deve medir e ordenar os impactos visuais do seu ambiente para corresponder à natureza. Deve comunicar as suas descobertas aos seus semelhantes para o reforço mútuo das suas ações. Ele se afirma no mundo material por meio do seu equipamento sensorial, bem como do seu processo de pensamento. Assim, o controle da natureza inclui a domesticação da natureza através do olho, a assimilação visual de eventos espaço-temporais.

As imagens visuais são ferramentas para este controle progressivo da natureza. Cada nova conquista visual cria um novo horizonte, um novo quadro de referência, um novo ponto de partida para um desenvolvimento posterior. À medida que os aspectos da natureza mudam, o homem precisa reajustar essas ferramentas e desenvolver novos usos para elas. Assim como há progresso no processo de pensamento, também há evolução na compreensão sensorial. O desenvolvimento da visão leva não só a uma maior compreensão da natureza, mas também ao desenvolvimento progressivo das sensibilidades humanas e, consequentemente, a experiências humanas mais amplas e profundas.

É uma experiência comum que atividades originalmente voluntárias se tornem em grande parte automáticas com a prática. Ao chegar a uma cidade nova, tem-se consciência de cada passo dado. A orientação é uma atividade voluntária, cada passo envolvendo uma medição consciente de distância e direção. Mas com a familiaridade, a orientação torna-se mecânica. Vai-se de um lugar para outro sem pensar conscientemente na rota ou nos pontos de referência. Ao sentar-se pela primeira vez ao volante de um automóvel, deve-se dar atenção voluntária concentrada a cada movimento da mão e do pé. Com o tempo, porém, essas manipulações tornam-se automáticas. O motorista pode conversar ou ouvir o rádio sem ter medo de perder o controle do carro. O ato de falar exige uma coordenação complexa de músculos e cordas vocais. No entanto, em uma conversa, estamos principalmente cientes dos significados que desejamos expressar e quase nada conscientes dos movimentos da língua e dos lábios que usamos para formar os sons e transmitir o significado.

Como diz William James:
``A grande coisa, então, em toda a educação é tornar o nosso sistema nervoso um aliado em vez de um inimigo. É capitalizar as nossas aquisições e viver à vontade com os juros desse capital. Para isso, devemos tornar automáticas e habituais o mais cedo possível o maior número de ações úteis que pudermos. \ldots Quanto mais detalhes da nossa vida diária pudermos entregar à custódia sem esforço do automatismo, mais o poder superior da mente será libertado para o seu trabalho próprio.''

\vfill
\centering 66

\newpage
\section*{Page 69}

Vemos como os pintores, escultores, arquitetos, fotógrafos, designers de publicidade nos ensinam a ver. O valor social da imagem representacional é, portanto, que ela pode nos dar educação para um novo padrão de visão. A ação voluntária do pintor, ao se esforçar para se ajustar aos aspectos mutáveis dos eventos espaço-tempo, deve ser traduzida através da imagem objetiva na experiência do observador em um novo e automático padrão de visão.

Ver relações espaciais em uma terra plana é uma experiência diferente de vê-las em uma região montanhosa onde uma forma intercepta a outra. Para se orientar ao caminhar, é necessária uma medição espacial diferente de dirigir em um automóvel ou um avião. Compreender as relações espaciais e se orientar em uma metrópole hoje, entre as intrincadas dimensões de ruas, metrôs, trens elevados e arranha-céus, exige uma nova maneira de ver.

Como a geometria euclidiana era apenas uma primeira aproximação no conhecimento das formas espaciais, refletindo apenas um certo complexo limitado de propriedades espaciais, as formas tradicionais de representação visual eram apenas a primeira aproximação na percepção da realidade espacial.

Nos últimos cem anos, a prática tecnológica introduziu um novo e complexo ambiente visual. A tarefa do pintor contemporâneo é encontrar a maneira de ordenar e medir este novo mundo. Este desafio histórico o convoca a assimilar as novas descobertas e a desenvolver uma nova sensibilidade, um novo padrão de visão que possa liberar o sistema nervoso para uma escala mais ampla de orientação.

A representação visual opera por meio de um sistema de sinais baseado em uma correspondência entre as estimulações sensoriais e a estrutura visível do mundo físico. Os eventos espaço-tempo do mundo físico devem ser traduzidos para as relações de superfícies coloridas no plano da imagem. O homem aprendeu gradualmente a ordenar certas relações visíveis de eventos espaço-tempo, ou seja, de extensão, de profundidade e de movimento. O desenvolvimento histórico da representação mostra uma conquista gradual dessas relações ópticas nos termos da superfície bidimensional da imagem.

Esta correspondência óptica não é de forma alguma necessariamente congruente com a experiência espacial intrínseca em uma organização plástica da superfície da imagem. Uma fotografia de um cavalo correndo pode parecer inerte dentro das quatro bordas do plano da imagem, se sua posição, congelada no plano da imagem, tiver uma relação plástica com a superfície bidimensional que proporciona uma experiência estática. A mesma fotografia será dinâmica se as unidades ópticas forem ordenadas em referência às margens da imagem de modo a induzir uma experiência cinética. É necessário encontrar uma congruência entre esses dois referenciais: as relações observadas no mundo espacial real e a natureza espacial do plano da imagem bidimensional. Uma representação visual da natureza pode ser vital na experiência humana apenas se ela se tornar uma forma da própria natureza, alcançando uma qualidade orgânica, uma unidade plástica.

\par\vspace*{\fill}\hfill 67

\newpage
\section*{Page 70}

O objetivo é uma representação visual na qual o conhecimento mais avançado do espaço é sincronizado com a natureza da experiência plástica. O espaço-tempo é ordem, e a imagem é um "ordenador". Somente a integração desses dois aspectos da ordem pode tornar a linguagem da visão o que ela deve ser: uma arma vital de progresso.

A representação visual possui três tendências paralelas. A primeira é a tendência de aproximar, numa relação bidimensional, a totalidade da experiência espacial. É uma síntese que inclui não só o que se vê, mas o que se sabe sobre a coisa vista. Se alguém sabe que um homem tem duas pernas, ele as desenhará ambas, embora apenas uma perna seja vista do ângulo particular de observação. Se alguém sabe que um prato tem forma circular e uma cor local característica, ele representará esse prato em sua forma visual característica, embora de seu ângulo particular de visão o prato deva parecer elíptico, e sua cor seja alterada por iluminação variável. São mostradas as características conceituais das unidades espaciais, em vez das características ópticas aparentes. A segunda tendência é em direção ao registro gráfico mais preciso de objetos projetados na retina. O artista tenta colocar em uma superfície plana um aspecto óptico aparente do mundo móvel e espacial. A terceira tendência é em direção a uma representação do conteúdo do desejo e da vontade. A seleção e arranjo dos elementos representacionais são guiados pelo desejo do artista de liberar tensões emocionais, materializando nas formas simbólicas de representação os objetivos de seus desejos.

A história da representação visual mostra uma ênfase mutável, ora em um, ora em outro, desses componentes. A imagem representacional nunca é idêntica à realidade espacial, mas a aproxima de acordo com os padrões prevalecentes de interesse e conhecimento. Não se vê todos os aspectos das coisas e eventos visíveis; seleciona-se e arranja-se as estimulações visuais de acordo com a própria atitude em relação a essas coisas. No mesmo grau em que o conhecimento do ambiente e os hábitos e atitudes em relação ao ambiente mudam, os hábitos visuais de representação também mudam.

Uma reavaliação dos idomas de representação ocorre apenas quando novos elementos que invadem o campo do ambiente são suficientemente importantes para exigir atenção e quando não há tradições para moldar os hábitos visuais em relação a eles. As recentes mudanças na ciência e tecnologia exigem tal reavaliação. Novas experiências, científicas, tecnológicas e sociais, já não se encaixam mais no antigo quadro de referência. À medida que nosso conceito de realidade se aprofunda e nosso conhecimento do espaço se amplia, uma reavaliação fundamental das formas tradicionais de representação é inevitável. A revisão dos idomas da representação espacial e sua integração com a linguagem genuína da superfície da imagem é a tarefa vital do artista contemporâneo. As seções seguintes examinam os idomas herdados da representação visual e os avaliam em termos contemporâneos.

\newpage
\section*{Page 71}

Desenho de Caverna Espanhol. 30.000-10.000 a.C.

Reprodução Cortesia do The Art Institute of Chicago

\textit{A unidade única}

A forma mais restrita de compreensão espacial é a percepção da unidade espacial única. A forma mais simples de representação espacial é a de um único elemento espacial.

O homem primitivo tinha uma compreensão limitada de espaço e tempo. Para ele, cada experiência estava confinada à sua própria vida espaço-tempo, sem referência ao passado ou futuro ou a relações espaciais mais amplas. Ele sentia pouca, ou nenhuma, necessidade de comparar e medir. Sua representação visual era limitada, na maior parte, a unidades espaciais únicas. A compreensão visual das figuras que ele representava não estava conectada com a compreensão do espaço circundante. Ele se preocupava pouco com fundos, peso, cima ou baixo. Cada elemento vivia sua própria vida em completa independência espacial. As figuras representadas eram apenas transitórias na superfície da imagem. A figura e o fundo não tinham interdependência orgânica. Esse caráter sem moldura da superfície da imagem, a falta de um referencial espacial, pode ser a razão pela qual artistas pré-históricos frequentemente sobrepunham novas pinturas ao trabalho de seus predecessores. O desenho infantil revela uma atitude semelhante. Elementos espaciais ainda não são compreendidos em suas interconexões. Eles não têm um referencial unificado. Como não há um fundo espacial coerente ao qual relacionar os elementos, os desenhos em um plano de imagem têm apenas organização acidental. As crianças desenham até que a figura atinja os limites do papel. Então, elas viram o papel e preenchem o espaço disponível.

\vfill
\raggedleft 69

\newpage
\section*{Page 72}

\begin{raggedright}
Madonna e Criança com Anjos. \\
Escola Italiana. Século XIII \\
Galeria Nacional de Arte, \\
Washington, D. C. Coleção Kress
\end{raggedright}
70

\newpage
\section*{Page 73}

\raggedright % Garante o alinhamento à esquerda para todo o bloco de texto

\textit{Design Publicitário 1941}
\textit{Paul Rand.}

\par\bigskip % Separa o texto da 'legenda' do conteúdo principal do artigo

\textbf{Relação de tamanho}

Estamos acostumados a atribuir a uma projeção retiniana maior ênfase espacial, ou seja, uma característica de preenchimento de espaço maior. O tamanho, portanto, torna-se a declaração mais simples sobre o espaço --- o primeiro passo da organização do mundo espacial.

Nas primeiras formas de representação visual, o espaço é indicado pela extensão das áreas de cor na superfície da imagem. Nessas primeiras imagens representacionais, a hierarquia do tamanho estava intimamente associada à hierarquia de poder, força e importância. Assim, a primeira escala espacial tinha uma correspondência estrutural com a escala de valores. As relações de tamanho eram usadas não apenas como um sinal espacial, mas também como símbolos e como um meio de ênfase plástica.

A perspectiva renascentista destruiu essa correspondência estrutural de espaço, símbolo e ênfase plástica por uma imitação servil da imagem ótica aparente do mundo-objeto tridimensional. Após um longo eclipse, esse uso estrutural das diferenças de tamanho foi redescoberto por pintores contemporâneos, fotógrafos e cinegrafistas. A arte publicitária, desinibida pela tradição, também encontrou um uso dinâmico e estrutural do tamanho contrastante das superfícies coloridas. A página tem seu próprio mundo espacial, não no sentido naturalístico como uma ilusão de distâncias reais entre os elementos representados, mas no sentido de que nela o tamanho da imagem e da palavra estão em uma conexão plástica e significativa.

\vfill % Empurra o conteúdo para cima para permitir o número da página na parte inferior
\raggedleft 71 % Posiciona o número da página no canto inferior direito

\newpage
\section*{Page 74}

\begin{center}
\textit{Relação de profundidade por localização vertical}
\end{center}

Para o espectador, a linha do horizonte fornece um quadro de referência. Ele julga a posição do objeto que vê em relação à linha do horizonte e, assim, recebe uma impressão de sua distância em relação a si mesmo, bem como de outros objetos à sua frente. Mesmo que a linha do horizonte não seja aparente, as diferentes elevações dos elementos indicam uma posição em profundidade.

A representação na superfície bidimensional da imagem tem convencionalmente utilizado o significado espacial da localização vertical. A linha do horizonte visível ou latente foi mantida como o quadro de referência. O plano da imagem foi identificado com, e convencionalmente fixado ao, plano horizontal do solo. A parte inferior do plano da imagem representou o ponto visual mais próximo; consequentemente, o grau de elevação das unidades visuais indicou posições espaciais em recuo.

\newpage
\section*{Page 75}

\textit{Desenho de uma criança espanhola de onze anos}

K'o Ssu. \textit{Festa dos Pêssegos}

Cortesia do Instituto de Artes de Minneapolis

73

\newpage
\section*{Page 76}

Fotografia Aérea

Kasimir Malevich.\\
Composição Suprematista\\
1914

Ladislav Sutnar. Fotografia

74

\newpage
\section*{Page 77}

\textbf{Casa Mínima}

\textbf{Ladislav Sutnar.} \textit{Capa de Livro 1930.}

\bigskip

\textbf{William Burtin.} \textit{Design Publicitário 1941.}

Novas descobertas tecnológicas provocaram uma reavaliação fundamental da posição vertical como sinal de profundidade. A visão de pássaro e a visão de sapo em fotografias, e uma nova visão na observação aérea, foram os fatores mais importantes. Para o aviador, assim como para o fotógrafo, a linha do horizonte muda constantemente e, consequentemente, perde sua validade absoluta. Deixou de ser inevitável que a compreensão visual dos objetos e suas relações espaciais se baseassem em um quadro de referência que tivesse um horizonte constante---o horizonte visível ou latente fixo.

Assim liberados, os sinais de representação do espaço podem funcionar como forças plásticas. A ordem do espaço real e o espaço do plano da imagem estão em estreita congruência.

\newpage
\section*{Page 78}

\subsection*{Representação de profundidade por figuras sobrepostas}

Se uma forma espacial obstrui nossa visão de outra forma, não presumimos que a segunda deixa de existir por estar oculta. Reconhecemos, ao observarmos tais figuras sobrepostas, que a primeira ou a mais superior possui dois significados espaciais---ela mesma e o que está sob ela. A figura que intercepta a superfície visível de outra figura é percebida como mais próxima. Experimentamos diferenças espaciais ou profundidade. A representação de sobreposição indica profundidade. Ela cria uma sensação de espaço. Cada figura aparece paralela ao plano da imagem e tende a estabelecer uma relação espacial de recuo.

\vspace{\baselineskip}

\raggedright
\textit{O Último Julgamento} \\
\textit{Alemão 1460} \\
\textit{Cortesia da reprodução} \\
\textit{The Art Institute of Chicago}

76

\newpage
\section*{Page 79}

\raggedright % Default left alignment for the entire content

% Top-left block
\parbox[t]{0.45\textwidth}{ % Approximate width based on visual
\textbf{\large Clifford Eitel.}

\textit{Estudo da transparência}

Trabalho feito para o curso do autor
em Fundamentos Visuais
Patrocinado por
\textit{The Art Director's Club of Chicago}
}

\vspace{2em} % Vertical space between this block and the next main content

% Main heading
\textbf{\large Transparência, interpenetração}

\vspace{1em} % Space before main body paragraphs

% Main body paragraphs
Se alguém vê duas ou mais figuras parcialmente sobrepostas uma à outra, e cada
uma delas reivindica para si a parte comum sobreposta, então se depara
com uma contradição de dimensões espaciais. Para resolver esta
contradição, deve-se assumir a presença de uma nova qualidade óptica.
As figuras são dotadas de transparência; isto é, elas são capazes de
interpenetrar sem uma destruição óptica uma da outra. A transparência,
no entanto, implica mais do que uma característica óptica; implica
uma ordem espacial mais ampla. Transparência significa uma percepção simultânea
de diferentes localizações espaciais. O espaço não apenas recua, mas flutua em uma
atividade contínua. A posição das figuras transparentes tem um significado
equívoco, pois se vê cada figura ora como a mais próxima, ora como a mais distante.

A ordem do nosso tempo é amassar juntos o conhecimento científico e técnico
adquirido, em um todo integrado no plano biológico e social.
Hoje, dificilmente há aspectos do esforço humano onde o
conceito de interpenetração como dispositivo de integração não esteja em foco.
Tecnologia, filosofia, psicologia e ciência física estão usando-o como
princípio orientador. O mesmo fazem a literatura, pintura, arquitetura, cinema
e fotografia, e design de palco. Além disso, é um conhecimento técnico
comum em nossa vida diária. Ondas de rádio são
o exemplo mais claro disso.

\vspace{2em} % Vertical space before the quote block

% Bottom-left quote block
\parbox[t]{0.45\textwidth}{ % Approximate width based on visual
\small % Smaller font size for the quote
``Bolas de bilhar em movimento não conseguem
passar uma através da outra: o encontro significa deslocamento. Mas
ondas em movimento de diferentes centros (como na superfície de uma
lagoa) podem passar uma através da outra sem conflito,
somando-se umas às outras à medida que passam. E, normalmente, dois gases,
liberados no mesmo espaço fechado, se expandirão um através
do outro até que cada um preencha todo o espaço. No mundo físico, existem
numerosos exemplos de `interpenetração.' É concebível
que expansões políticas também possam interpenetrar como ondas,
em vez de colidir como bolas de bilhar?''

\raggedleft % Align attribution to the right within this box
--William Ernest Hocking.\\
\textit{O Propósito Mundial da América.}
}

\vfill % Push content to the top
\raggedleft % Align page number to the right
77

\newpage
\section*{Page 80}

\raggedright

\vspace*{2.5in} % Espaço aproximado da margem superior até a primeira legenda

\textit{Picasso. Retrato de Kahnweiler}\\
{\small\textit{Cortesia da Sra. Charles B. Goodspeed}}

\vspace*{3in} % Espaço aproximado entre as duas legendas

\textit{Amadee Ozenfant. Natureza-morta Purista}\\
{\small\textit{Cortesia da Art of This Century}}

\vfill % Empurra o número da página para o final

78

\newpage
\section*{Page 81}

\noindent G. F. Keck. Detalhe de uma Casa \\
Fotografia de \textit{W. Keck}

A arquitetura contemporânea utiliza a qualidade transparente de materiais sintéticos, vidro, plásticos, etc., para criar um design que integrará o maior número possível de vistas espaciais. O interior e o exterior estão em estreita relação, e cada ponto de vista no edifício oferece a mais ampla compreensão visível do espaço. Reflexos e espelhamentos, materiais de construção transparentes e translúcidos são cuidadosamente calculados e organizados para focar vistas espaciais divergentes em uma única apreensão visual.

\noindent L. Moholy Nagy. \\
Construção Espacial 1930

\newpage
\section*{Page 82}

O controle técnico de fontes de luz artificial, a projeção de imagens por meio da luz, tem contribuído também para a reavaliação da sobreposição e a introdução do dispositivo representacional da transparência. Os raios de luz que cobrem uma imagem são capazes de se interpenetrar; a luz aumenta a luz, a sombra aprofunda a sombra. O resultado é uma maior intensidade.

A emulsão fotográfica é caracteristicamente capaz de registrar em uma superfície de imagem duas ou mais projeções sobrepostas. O efeito resultante comprime dois ou mais aspectos espaciais e os molda em um tipo mais amplo de representação espacial. A fotografia de raios-X abriu um novo aspecto do mundo visível. Coisas até então escondidas do olho humano puderam ser penetradas e tornadas visíveis. Aqui, a transparência ganha um novo significado, porque a profundidade do objeto também é avaliada por sua densidade óptica.

Gyorgy Kepes. Fotomontagem 1937

Gyorgy Kepes. Design de Publicidade 1937

\newpage
\section*{Page 83}

William Burtin. \textit{Design de Publicidade} 1940.

O FÓRUM ARQUITETÔNICO
décadas de design
OUTUBRO 1940

Jack Waldheim. \textit{Fotografia Superposta} 1943.

81

\newpage
\section*{Page 84}

\raggedright
Exposição de pintura \\
contemporânea \\
norte-americana.

Paul Rand. Projeto de Cartaz

Frank Barr. Design Tipográfico 1941

CON POR BIEDERMAN

COLARES \\
POR \\
ANNI ALBERS \\
ALEX REED

Mês de Novembro

GALERIA KATHARINE KUH \\
540 N. MICHIGAN CHICAGO

82

\newpage
\section*{Page 85}

Cassandre. \textit{Air Orient} 1932\par
\textit{Cortesia do Museu de Arte Moderna}\par

A técnica do processo de impressão oferece outra oportunidade para o controle criativo da transparência. Uma impressão sobre outra condensará uma variedade de dimensões espaciais em um todo significativo.\par

\textit{Design da Capa 1934}\par

\newpage
\section*{Page 86}

\textbf{APROVEITE}

\textbf{PARA SE DIVERTIR MELHOR}

\textbf{MÚSICA}

E. McKnight Kauffer. Cartaz
Cortesia do Museu de Arte Moderna

\textbf{Verniz}

O verniz, quando desejado, é aplicado em caixas como uma exceção ao processo de impressão regular. É uma excelente proteção contra impressões digitais, poeira ou sujidade devido ao manuseio. Aumenta muito a aparência de uma embalagem de presente, adicionando brilho e profundidade de cor.

Gyorgy Kepes. Design Publicitário 1938

34

\newpage
\section*{Page 87}

\raggedright
Le Corbusier, Desenho \\
Cortesia de Carl O. Schniewind

Fernand Leger, Design Publicitário 1942 \\
Cortesia da Container Corporation of America

\textbf{\Large ONE} \\
Em uma única organização: materiais de celulose, fábricas de papelão e embalagens \\
CONTAINER CORPORATION OF AMERICA

85

\newpage
\section*{Page 88}

\subsection*{Perspectiva Linear}

A imagem retiniana dos objetos encolhe ou incha à medida que os objetos estão mais próximos ou mais distantes do espectador. Helmholtz diz:
\begin{quote}
O mesmo objeto visto a diferentes distâncias será representado na retina por imagens de tamanhos diferentes e subtenderá ângulos visuais diferentes. Quanto mais distante estiver, menor será seu tamanho aparente. Assim, da mesma forma que os astrônomos podem calcular as variações das distâncias do sol e da lua a partir das mudanças nos tamanhos aparentes desses corpos, sabendo o tamanho do objeto, um ser humano, por exemplo, pode estimar a distância de nós por meio do ângulo visual subtendido ou, o que equivale à mesma coisa, por meio do tamanho da imagem na retina.
\end{quote}

O uso dessa relação geométrica foi reintroduzido pelos pintores renascentistas como o principal recurso para representar relações espaciais. Seu objetivo artístico era o domínio científico óptico da natureza. Condicionados pelas aspirações e pela visão da Renascença, eles buscaram alcançar isso passo a passo, focando sempre em um aspecto, em um setor recortado, da riqueza ilimitada da natureza circundante. Assim como o anatomista---outro pioneiro do mesmo espírito, que conquistou o conhecimento eliminando os aspectos vivos e móveis do corpo---o artista-anatomista da imagem visual eliminou o fluxo das inúmeras relações visuais que o mundo visível tem para o espectador. Ele congelou a riqueza viva e flutuante do campo visual em um sistema geométrico estático, eliminando o elemento tempo sempre presente na experiência do espaço e, assim, destruindo as relações dinâmicas na experiência do espectador.

\medskip
\noindent\textit{Desenho em perspectiva de Piero della Francesca}

\subsection*{Perspectiva Inversa}

De acordo com os antigos cânones chineses, pintores chineses e japoneses atribuem à perspectiva linear um papel diametralmente oposto ao que lhe é dado pelos pintores ocidentais. Em seu sistema, linhas paralelas convergem à medida que se aproximam do espectador. Elas abrem o espaço em vez de fechá-lo. O espaço pictórico não é um diagrama óptico científico das posições aparentes dos objetos, mas um meio de experiência, um panorama bidimensional ativo para o espectador, que vive a imagem. A mesma abordagem foi utilizada em muitas pinturas europeias antigas.

\medskip
\noindent\textit{Xilogravura Alemã Antiga}

\newpage
\section*{Page 89}

\begingroup
\raggedright
Jere Donovan,\\
Fotomonstagem. Ação\\
Classe de Design Herbert Bayer 1939\\
Patrocinado por\\
A Guilda Americana de Publicidade
\par
\endgroup

A perspectiva linear proporcionou uma formulação unificada do espaço, mas restringiu as relações espaciais a um único ângulo de visão, um ponto de vista fixo, o do espectador, criando uma profundidade ilusória entre os objetos e uma distorção ilusória de sua forma real. Um detalhe sem importância pode interceptar na imagem em escorço o elemento mais significativo, tornando assim o todo ininteligível. Podemos apagar uma casa ou um homem segurando um dedo perto do olho. De um certo ângulo de visão, formas dessemelhantes podem aparecer como projeções ópticas semelhantes e formas semelhantes como dessemelhantes.

Se algum significado de profundidade deve fluir do encurtamento e da diminuição pelo uso da perspectiva, o observador deve estar familiarizado com os objetos em suas características tridimensionais reais. Uma constância de memória, além disso, está ligada a coisas familiares do nosso entorno. Mantemos um tamanho e uma forma constantes em nossa percepção, por mais que o tamanho e a forma da projeção retiniana possam variar com as mudanças em nosso ângulo de visão. Por exemplo, quando vemos dois homens, um a seis pés de distância e o outro a quinze, ambos nos parecem aproximadamente do mesmo tamanho. Quando olhamos para um prato de um ângulo oblíquo, ele deveria, pelas regras da perspectiva linear, aparecer elíptico; na verdade, ainda o vemos como redondo. Quando olhamos, a projeção retiniana se fixa em apenas um pequeno fragmento da relação espacial que realmente percebemos; complementamos a parte não vista com nossa memória constante de um fundo espacial unificado.
Diferenças de tamanho que não se registram na percepção visual direta e que parecem inalteradas, embora projetem imagens diferentes na retina, revelam, quando fixadas em um plano pictórico bidimensional, um grande poder de acentuar a ilusão de profundidade entre os objetos.

\newpage
\section*{Page 90}

\begin{center}
\small\itshape
Tintoretto, Hércules e Antaeus\\
Cortesia do Wadsworth Atheneum
\end{center}

\vspace{1.5cm} 

\begin{center}
\small\itshape
Toulouse-Lautrec, Bailarinas de Balé\\
Cortesia do The Art Institute de Chicago
\end{center}

\vspace{3cm} 

\begin{center}
\normalfont\itshape
Perspectiva Amplificada
\end{center}

\vspace{0.7em} 

\noindent
Assim que a Renascença introduziu a perspectiva, seus pintores começaram a achar o sistema fixo de representação do espaço menos que satisfatório. Alguns tentaram quebrar os limites indo a extremos. O espaço unificado da perspectiva linear foi saturado por distorções extremas. O contraste máximo entre pequeno e grande foi aplicado para injetar no espaço da imagem o máximo de vitalidade. A estrutura da perspectiva foi esticada ou condensada aos limites extremos, atingindo a maior expressão dinâmica possível dentro do sistema estático de perspectiva linear.

\par 

A perspectiva amplificada é usada em fotografia, fotomontagem e em filmes como um dispositivo potente para criar uma forte sensação de espaço.

\vfill 

\begin{center}
88
\end{center}

\newpage
\section*{Page 91}

S.F. FEDERAÇÃO FRANCESA DE TÊNIS R.C.
A.M. CASSANDRE 32

DE 21 DE MAIO A 5 DE JUNHO\\
GRANDE QUINZENA INTERNACIONAL DE TÊNIS\\
ESTÁDIO ROLAND GARROS -- PORTE D'AUTEUIL

A. M. Cassandre, Cartaz 1932\\
Cortesia do Museu de Arte Moderna

\newpage
\section*{Page 92}

{\itshape Perspectiva Múltipla, Simultânea}

\vspace{1.5em}

Outros pintores modificaram e romperam a unidade espacial estática da perspectiva linear, introduzindo em uma única imagem vários pontos de fuga e vários horizontes. O objetivo deles era trazer para o espaço da pintura a mais ampla relação espacial possível, moldando distintamente a perspectiva linear à natureza do plano da imagem. Leonardo da Vinci, em sua Adoração, introduziu vários pontos de vista e várias linhas de horizonte para tornar a paisagem no fundo claramente visível. Jan Van Eyck às vezes usava três ou mais pontos de fuga para aumentar o espaço interno de uma sala. Veronese, Tintoretto e outros pintores empregaram em uma única imagem muitos pontos de fuga e linhas de horizonte.

Seu trabalho foi o primeiro a romper com o sistema limitado de perspectiva linear que fazia o espectador pausar no tempo e no espaço — uma contradição com a natureza da experiência visual. Pela perspectiva múltipla, a fixação estática foi superada, porque a perspectiva simultânea significa movimento no espaço.

\vspace{2em}

\footnotesize
{\itshape Di Paolo. Batista no Deserto} \\
Cortesia do Instituto de Arte de Chicago

\vspace{1em}

\footnotesize
{\itshape Tintoretto, Vênus e Marte com as Três Graças} \\
Cortesia do Instituto de Arte de Chicago

\vfill
\raggedleft
90

\newpage
\section*{Page 93}

Perfeição mecânica da perspectiva linear

A visão libertada pela câmera fotográfica foi capaz de explorar territórios até então intocados da perspectiva. Aspectos ópticos latentes tornaram-se aparentes porque a câmera foi capaz de reproduzir objetos de um ângulo de visão que o olho desarmado não conseguiria alcançar com conforto razoável, se é que conseguiria. Não apenas as vistas frontais e de perfil habituais, mas também a vista de cima, a vista aérea, e a de baixo, a vista de sapo, foram registradas. O ponto de fuga que, na representação espacial tradicional, geralmente ficava no meio do plano da imagem, foi deslocado para a esquerda, direita, cima e baixo, em quase todas as posições possíveis. Para cada posição em mudança, havia não apenas um recorte correspondente do campo visual, mas também, dentro desse recorte, um encurtamento diferente.

A fotografia cinematográfica aumentou ainda mais a elasticidade do encurtamento e introduziu uma flexibilidade até então inédita no uso de diferenças de tamanho para acentuação espacial. O "close-up" quebrou a unidade espacial contínua tradicional herdada da pintura e do teatro e estendeu o espaço da imagem para uma dimensão amplificada. Em uma sequência, um "close-up," "plano médio" e "plano geral" trazem uma variedade viva e em movimento de espaço em expansão e condensação.

Acessórios ópticos dentro ou fora da câmera foram empregados para a exploração posterior das aparências das coisas. Espelhos, prismas e lentes especiais esboçaram, difundiram, distorceram, repetiram, moldaram as coisas e criaram imagens não correspondentes à percepção visual direta.

L. Moholy Nagy,
Bauhaus em Dessau 1926
Primavera 1929
Vista Aérea 1925

\raggedleft 91

\newpage
\section*{Page 94}

George Morris.\par
Distorção por Prisma 1940 \textbullet

\bigskip % Adding a bit of vertical space for visual separation as per image layout

M. Halberstadt. Espelhamento 1941 \textbullet\par
\textbullet\ Trabalho experimental realizado sob a direção do autor no Departamento de Luz e Cor da Escola de Design de Chicago.

\vfill % Pushes the next content to the bottom of the page, mimicking the image layout

James Brown.\par
Distorção em Espelho 1940

\bigskip % Adding a bit of vertical space for visual separation as per image layout

92

\newpage
\section*{Page 95}

\section*{Ruptura da perspectiva fixa}

A invenção e o aperfeiçoamento da câmera não foram de forma alguma o único fator a tender para a ruptura da validade absoluta da perspectiva linear. Toda a tendência social do mundo contemporâneo tornou tal ruptura inevitável.

O Renascimento, que redescobriu as regras dessa perspectiva, havia despertado forças econômicas que levaram a um interesse em cada faceta da compreensão e controle da natureza. Esse interesse, por sua vez, liberou um tremendo progresso científico e tecnológico. O progresso revolucionou a produção, remodelou a estrutura econômica e social e transformou a paisagem interior e exterior do homem.

Os novos dispositivos técnicos, máquinas, foram capazes de produzir e reproduzir, com uma velocidade até então desconhecida e em uma quantidade até então inimaginável, objetos, mercadorias para uso humano. Todos os esforços humanos estavam concentrados na produção de objetos. O próprio ser humano se perdeu em sua própria avaliação de um objeto capaz de produzir outro objeto. A natureza mecânica de toda a existência social e econômica estava assimilando o homem. Estava invadindo a esfera humana e destruindo-a. Ele se tornou uma máquina, ou parte de uma máquina. O homem estava perdendo seu status de indivíduo. Em sua própria vida, as leis ilusórias da perspectiva individual estavam sendo destruídas pela mecanização descontrolada. O aparente espaço econômico do indivíduo---sua crença em sua capacidade de construir sua própria vida, guiada apenas por seu próprio interesse, vontade e força---estava sendo desfeito pelo mecanismo econômico.

A complexidade do produto superou o controle humano. A riqueza da produção tornou-se inutilizável e desperdiçada pela falta de compreensão social, isto é, de direção planejada. Os novos objetos e novos dispositivos trouxeram para o campo visual uma riqueza de novo material. Havia mil coisas novas para ver e mil novas formas de ver, mas a maioria delas também foi desperdiçada porque não havia um princípio ordenador estabelecido para organizar o novo mundo visível.

Diante dessa situação, o indivíduo tentou dominá-la. Ele protestou contra ser apenas mais um objeto e buscou sua posição no espaço. Pintores, eles próprios arrastados para o conflito, usaram a imagem como um campo de testes, um campo de batalha. Eles forçaram seu interesse no objeto e em sua posição no espaço. Tiveram que dominar e compreender as características espaciais dos objetos para se compreenderem e, assim, redirecionar suas próprias vidas.

\raggedleft 93

\newpage
\section*{Page 96}

{\itshape
Escultura Africana em Madeira, \par
Fetichismo da Tribo Pahouin, \par
Gabão \par
Reprodução Cortesia \par
The Art Institute of Chicago \par
}

\par % Ensure a paragraph break after the caption block

{\centering\itshape
Análise espacial do objeto \par
}

\par % Ensure a paragraph break after the heading block

Um indivíduo confrontado com uma nova e complexa tarefa busca imediatamente alguma forma de precedente para auxiliá-lo. Ele faz um inventário de suas experiências passadas e das de outros. Da mesma forma, em tempos críticos, quando um grupo enfrenta problemas sociais ou culturais novos e complexos, cuja solução está além do padrão habitual, o primeiro passo instintivo é buscar soluções e extrair sabedoria de culturas distantes.

\par % Paragraph break

Em sua busca por uma nova ordem estrutural na qual a riqueza disponível pudesse funcionar, pintores contemporâneos, confusos e encurralados nesta turbulência do novo ambiente visual, também redescobriram soluções de culturas anteriores. A escultura negra da África forneceu, em pequena escala, uma resposta ao seu problema. Nestas formas simples, cada plano envolvente não submerge num todo ilusório, mas age como uma direção dinâmica individual que leva a outro plano, o qual, por sua vez, leva à compreensão do todo. Cada plano, em sua simplicidade desimpedida por detalhes, possui uma clara função estrutural dinâmica.

\par % Paragraph break

{\raggedright\itshape
Picasso, Dançarina 1907 \par
Coleção de \par
Walter P. Chrysler, Jr. \par
}

\par % Paragraph break

94

\newpage
\section*{Page 97}

\textit{Juan Gris, Pintura}\\
Cortesia do Museu de Arte do Smith College

Essa ideia dinâmica de seguir os planos que envolvem um objeto foi levada a outras conclusões. Os pintores descobriram que um ponto de observação, apesar da ênfase pela distorção, não era suficiente para dar a essência espacial do objeto. Então, o pintor moveu-se em torno do objeto, penetrou-o e usou todos os meios disponíveis para descrever o maior número possível de suas relações com o espectador e com outros objetos. Todas as ferramentas afiadas da perspectiva foram focadas em uma representação simultânea. Os pintores mudaram o ponto de visão para uma espécie de sequência cinematográfica e representaram a projeção de vários pontos de vista em uma única imagem.

\textit{Picasso, Natureza Morta "Jolie Eve" 1913}\\
Cortesia do Instituto de Arte de Chicago

\vspace*{\fill}
\raggedleft 95

\newpage
\section*{Page 98}

\raggedright
Ícone Russo, São Nicolau\\
\textit{Hammer Galleries, Nova Iorque}

\vspace{2em}

Com a diretaza de um propósito unívoco, os desenhos do homem primitivo sempre exibem a maneira mais clara de expressar o essencial de uma coisa visível. Quando um homem primitivo usa uma espécie de radiografia para mostrar a entidade espacial essencial das coisas ou quando usa simultaneamente o perfil e o rosto de uma figura, ele encontra o cerne do problema da representação. O espectador é conduzido na superfície da imagem a todas as referências espaciais significativas do sujeito; a experiência visual se torna uma experiência dinâmica.

Quando uma criança tenta a representação gráfica de uma situação espacial, ela não se satisfaz com uma projeção perspectiva acidental. Ela torce e inclina os vários aspectos visuais possíveis até que explique completamente os objetos que deseja representar. O resultado final é uma combinação de planta e elevação. Ao desenhar uma carroça, a criança dá ao cavalo, às rodas e às pessoas a projeção mais característica. Há finalmente uma fusão do mundo tridimensional e do plano da imagem bidimensional. Pintores medievais antigos frequentemente repetiam a figura principal muitas vezes na mesma imagem. Seu propósito era representar todas as relações possíveis que o afetavam, e eles reconheciam que isso só poderia ser feito por uma descrição simultânea de várias ações. Essa conectividade no sentido, em vez da lógica mecânica da óptica geométrica, é a tarefa essencial da representação. Franz Boas, em seu livro ``Arte Primitiva'', oferece um resumo lúcido do cerne da representação visual.

\vfill
\raggedright 96

\newpage
\section*{Page 99}

É facilmente inteligível que uma vista de perfil de um animal em que apenas um olho é visto e em que um lado inteiro desaparece pode não satisfazer como uma representação realista. O animal tem dois olhos e dois lados. Quando ele se vira, vejo o outro lado; ele existe e deve fazer parte de uma imagem satisfatória. Numa vista frontal, o animal aparece em perspectiva. A cauda é invisível, assim como os flancos; mas o animal tem cauda e flancos, e eles deveriam estar lá. Nos deparamos com o mesmo problema em nossas representações de mapas do mundo. Num mapa na projeção de Mercator, ou em nossos planisférios, distorcemos a superfície do globo de tal forma que todas as partes são visíveis. Estamos interessados apenas em mostrar, da maneira mais satisfatória possível, as inter-relações entre as partes do globo. O mesmo é verdadeiro em desenhos arquitetônicos ortogonais, particularmente quando duas vistas adjacentes, tomadas em ângulos retos entre si, são colocadas em contato, ou em cópias de desenhos em que as cenas ou designs representados em um cilindro, um vaso ou um pote esférico são desenvolvidos em uma superfície plana para mostrar, num único relance, as inter-relações das formas decorativas. Em desenhos de objetos para estudo científico, podemos também, por vezes, adotar um ponto de vista semelhante, e a fim de elucidar relações importantes, desenhar como se fôssemos capazes de olhar ao redor do canto ou através do objeto. Diferentes momentos são representados em diagramas nos quais movimentos mecânicos são ilustrados e nos quais, a fim de explicar o funcionamento de um dispositivo, várias posições de peças móveis são mostradas.

\par
\noindent\small\textit{Perspectiva Mista. Desenho Infantil}

\par
\noindent\small\textit{Perfil Misto. Desenho Infantil}
\noindent\small\textit{De Helga Eng. Psicologia dos Desenhos Infantis}

\par
``Na arte primitiva, ambas as soluções foram tentadas: a perspectiva, bem como a que mostra as partes essenciais em combinações. Uma vez que as partes essenciais são símbolos do objeto, podemos chamar este método de simbólico. Repito que no método simbólico, aquelas características são representadas que são consideradas permanentes e essenciais, e que não há tentativa por parte do desenhista de se restringir a uma reprodução do que ele realmente vê num dado momento.''

\par
\noindent\small\textit{Pintura Boschímana}

\vfill
\raggedleft 97

\newpage
\section*{Page 100}

\itshape ENTREGA OU DECEPÇÃO

Seis fatores fazem a grande diferença...

Hoje... apesar dos incríveis avanços feitos até agora, as Indústrias de Guerra Americanas enfrentam uma responsabilidade ainda maior para acelerar a produção.

A decepção deve definitivamente ser coisa do passado --- assim como o período de ``tentativa e erro.'' Unidade na entrega em ação na estratégia geral para a Vitória.

É por isso que os responsáveis pelas aquisições estão aplicando o microscópio aos fornecedores --- para determinar suas capacidades absolutamente.

Para que você saiba exatamente o que sua organização está fazendo, \textsc{Em Termos Dos Seis Fatores Pelos Quais Você Provavelmente Nos Julgará}, é por isso que as informações neste folheto foram indexadas de acordo. A união desses seis fatores é sua garantia e promessa de desempenho responsável a serviço da Nação.

{\large\itshape Morton Goldsholl. Design Publicitário 1943}

A arte publicitária, não limitada por considerações tradicionais, era livre para desenvolver uma apresentação visual na qual cada figura é retratada na perspectiva que confere a maior ênfase à sua conexão em um significado.

{\large\itshape Redescoberta das forças plásticas básicas: linhas e planos de cor}

A exploração da natureza espacial dos objetos, andando --- na imaginação --- ao redor deles, e investigando seu volume visível, tornou a imagem mais informe. Mas nesta aglomeração informe dos diferentes planos envolventes, forças plásticas até então ocultas foram reveladas. Linhas e formas podiam agora manifestar uma qualidade espacial dinâmica que antes havia sido submersa na imitação de um único aspecto visual aparente.

A natureza e as limitações do plano-imagem bidimensional, as forças plásticas específicas das formas de cor devido à sua posição e área, foram novamente reconhecidas como fatores que controlam a construção do espaço nesta superfície pictórica. ``Na arte, o progresso não reside numa extensão, mas no conhecimento das limitações'', disse Braque. Em vez de serem derivadas diretamente dos objetos, as formas foram moldadas para se ajustar à construção do espaço no plano-imagem. A imagem tornou-se uma arquitetura de planos de cor criados pelo vão entre os planos, e pelo seu movimento virtual da superfície da imagem. Após o longo período de catalogação dos aspectos aparentes da natureza, o espectador novamente tornou-se parte integrante da imagem pictórica. A imagem tornou-se mais uma experiência espacial dinâmica em vez de um inventário morto de fatos ópticos.

98

\newpage
\section*{Page 101}

Assim como o esqueleto de aço na arquitetura, que une as paredes em um todo espacial, a extensão espacial é alcançada na imagem pela justaposição de linhas e planos. Quando um plano se move em virtude de sua cor e forma em uma direção, a estrutura linear justaposta a ele o traz de volta.

Este entrelaçamento de planos e linhas é um passo importante em direção à redescoberta da ação das forças plásticas. Uma leveza sem precedentes é alcançada, uma estrutura de espaço aberto na qual cada movimento pode ser seguido claramente. A massa dos volumes tridimensionais e seus movimentos gravitacionais unilaterais são agora trocados por um espaço dinâmico onde os elementos se expandem em todas as direções de acordo com as interações mútuas dos planos de cor que recuam ou avançam e o fluxo rítmico de suas linhas.

Picasso,
\textit{Menina com Chapéu Amarelo 1921}
Coleção de Walter P. Chrysler, Jr.

parabéns para \textbf{CORONET V.S.Q. BRANDY...AND SODA}
\textit{qualidade muito especial}

Sabor mais fresco ao beber... e depois!
O brandy Coronet com soda é o primeiro entre os grandes
drinques em suavidade, maciez, e efervescência saborosa.
Coronet é um brandy americano de luxo tão distinto
quanto você já provou... em qualquer lugar... a qualquer hora!
* faça seus dólares lutarem... compre títulos de guerra e selos!

Paul Rand,
\textit{Design Publicitário 1943}

99

\newpage
\section*{Page 102}

\noindent\textbf{\textit{Integração das forças plásticas}}

A desintegração do sistema unificador da perspectiva linear criou duas grandes dificuldades. Uma era que o número crescente de dados espaciais era demasiado numeroso para ser incluído no plano da imagem. A outra era que as energias plásticas, libertadas do objeto e da disciplina da perspectiva linear, se descontrolaram. Para contrariar estas dificuldades, os pintores introduziram dois dispositivos: primeiro, a compressão de planos através de interpenetrações; e, segundo, um controlo linear rítmico das superfícies da imagem.

\noindent\textbf{\textit{Compressão, interpenetração}}

O número crescente de pontos de referência trouxe o exterior e o interior, esquerda e direita, topo e base do objeto simultaneamente diante do olho. Somente estendendo o plano da imagem ao infinito todos os aspectos visíveis poderiam ser simultaneamente englobados. Mesmo que isso fosse possível, não seria uma solução; pois tal área se estenderia além do alcance visual. A área limitada da superfície da imagem ditava os métodos possíveis de reunir esses muitos fatos visíveis. A busca dos pintores mudou sua direção da extensão para a concentração.

Eles começaram a comprimir a multitude de dados ópticos dentro dos limites da superfície da imagem por meio de uma interpenetração de um plano com outro. Reconheceram, também, que os planos de cor liberados do objeto tinham uma ação centrífuga da superfície, e buscaram desenvolver uma força de equilíbrio que traria ordem a essa anarquia. A forma mais simples de integração que encontraram foi o entrelaçamento de elementos divergentes, através de uma interação rítmica de valores opostos, positivo e negativo. Na tecelagem, a recorrência dos diferentes fios coloridos cria unidade por descontinuidade rítmica. Os pintores inventaram um dispositivo similar. Por uma troca de valores opostos, por analogia de opostos de dentro para fora, preto e branco ou cores contrastantes, eles foram capazes de estabelecer um ritmo comum e, consequentemente, uma unidade. A ordem plástica foi novamente alcançada.

Pela interpenetração de diferentes linhas e planos, pelo entrelaçamento de positivo e negativo, escuro e claro, uma ação recíproca é produzida. Numa superfície clara, linhas ou formas escuras, e numa superfície escura, linhas ou formas claras, não só se ligam numa descontinuidade rítmica, mas, ao mesmo tempo, por contraste máximo de cada unidade individual, atingem uma maior intensidade. O antigo signo chinês oferece uma demonstração clara de uma unidade de opostos pela interdependência de cada uma das suas partes.

\vspace*{\fill}
\noindent 100

\newpage
\section*{Page 103}

\noindent
\textbf{Braque, \textit{Natureza Morta sobre Mesa}} \\
\textit{Coleção Chester Dale} \\
\textit{Cortesia do Instituto de Arte de Chicago}

\bigskip

\noindent
\textbf{\Large DIVERSIFICAÇÃO}

\noindent
\textit{Uma embalagem de papelão para cada produto} \\
\textbf{CONTAINER CORPORATION OF AMERICA}

\bigskip

\noindent
\textit{A. M. Cassandre, Design Publicitário 1937} \\
\textit{Cortesia da Container Corporation of America}

\bigskip

\begin{center}
101
\end{center}

\newpage
\section*{Page 104}

\noindent\textit{Integração do espaço por linhas equívocas.}

\noindent\textit{Casamento dos contornos}

\bigskip

Outro dispositivo introduzido para a integração dos planos de cores caóticos foi o uso de uma linha de contorno comum às várias unidades espaciais. Este contorno comum adquire um duplo significado, como um trocadilho óptico. Refere-se ao espaço interno e externo simultaneamente, e o espectador é, portanto, forçado a uma participação intensiva enquanto procura resolver a contradição aparente. Mas a linha de contorno equívoca faz mais do que unificar diferentes dados espaciais. Ela age como uma urdidura, tecendo os fios dos planos de cores em uma unidade rítmica. Este fluxo rítmico da linha injeta na superfície da imagem uma intensidade sensual.

\bigskip

\noindent\textit{Diagrama linear de uma pintura de A. Ozenfant}

\newpage
\section*{Page 105}

\noindent C\textsuperscript{ie} INT\textsuperscript{ale} DES WAGONS-LITS
\par
\noindent Companhia Internacional de Vagões-Leito
\par
\vspace{1em}

\noindent GRANDS RESEAUX DE CHEMINS DE FER FRANÇAIS
\par
\noindent GRANDES REDES FERROVIÁRIAS FRANCESAS
\par
\vspace{1em}

\noindent RESTAUREZ-VOUS
\par
\noindent RESTAURE-SE
\par
\noindent AU
\par
\noindent NO
\par
\noindent WAGON-BAR
\par
\noindent VAGÃO-BAR
\par
\noindent \textit{consommations -- petits repas}
\par
\noindent \textit{consumações -- pequenas refeições}
\par
\noindent A PEU DE FRAIS
\par
\noindent A BAIXO CUSTO
\par
\vspace{2em}

\noindent A. M. Cassandre, Poster Courtesy of The Museum of Modern Art
\par
\noindent A. M. Cassandre, Cartaz Cortesia do Museu de Arte Moderna
\par
\vfill
\noindent \raggedleft 103

\newpage
\section*{Page 106}

A. M. Cassandre, Cartaz 1935

A imagem fotográfica empregada para uma mensagem publicitária sempre colocou o problema de trazer elementos diversos para uma fusão harmoniosa. Elementos plásticos e verbais operam na mesma superfície, cada um com sua própria força agindo em sua própria direção. O texto, a qualidade caligráfica ou mecânica dos elementos desenhados, a fotografia, as cores, as formas, são diferentes em sua perspectiva, assim como em seu significado plástico e associativo. Para perceber as diferenças, deve-se comparar os elementos.

O contorno de um rosto é um esboço de um copo, uma garrafa, e também de uma linha de texto. A qualidade óptica idêntica, a linha de contorno comum, cria uma unidade espacial, nos termos da superfície bidimensional. Contudo, por ligar os diferentes elementos, força a comparação de suas diferenças. Essas diferenças ópticas, através de sua contiguidade inevitável, tornam-se contradições ópticas que só podem ser resolvidas em um novo significado comum.

Jean Carlu, Cartaz

104

\newpage
\section*{Page 107}

LA CARTELETTRE

Juan Gris, Litografia "Jacob, Não Corte, Senhorita." Cortesia do Museu de Arte Moderna

\newpage
\section*{Page 108}

G. Braque, Pintura
Cortesia da Phillips Memorial Gallery

McKnight Kauffer, Pôster 1933
Cortesia do The Museum of Modern Art

VOCÊ PODE TER CERTEZA DA SHELL
ATORES PREFEREM SHELL

106

\newpage
\section*{Page 109}

\begin{center}
    {\Large\textbf{\textit{Eliminação final da ordem de perspectiva fixa}}}
\end{center}

\begingroup
\raggedright

Os pintores cubistas apenas tentaram uma nova formulação visual das dimensões ampliadas do ambiente. Eles perceberam corretamente que uma perspectiva fixa não é suficiente para descrever fatos espaciais dinâmicos, e experimentaram inúmeras projeções-perspectivas simultâneas. Mas a imagem resultante ainda estava tão intimamente associada ao antigo conceito de objeto que não conseguia cobrir todas as possíveis experiências espaciais na vida contemporânea. Quanto mais complicado o ambiente e maiores as diferenças nas experiências, mais necessário se tornou encontrar uma simplificação da linguagem. Uma linguagem visual que reduzisse toda a experiência, antiga e nova, ao menor denominador comum, ainda precisava ser descoberta. Em uma equação matemática, os elementos são eliminados e simplificados até que apenas a estrutura principal do equilíbrio permaneça. Os pintores seguiram um procedimento análogo. De sua "equação" visual, eles despiram todos os elementos não essenciais. Reduziram a imagem à sua estrutura mais elementar.

\par

O trabalho dos pintores cubistas apenas abriu o caminho para um manejo mais controlado das forças plásticas na superfície da imagem. Seu trabalho apenas sugeria que a imagem tem vida própria, e que as forças plásticas, linhas e planos, podem criar uma sensação espacial sem retratar objetos. E quanto maior o seu afastamento da semelhança com o objeto, mais claras se tornavam as qualidades dinâmicas das forças plásticas. Quanto mais ousada sua tentativa de organização dessas forças, mais aparente se tornou a natureza do plano da imagem, como algo inteiramente diferente da ilusória ordem ótica geométrica do mundo-objeto. O plano da imagem foi gradualmente reconhecido como uma construção com suas próprias leis estruturais únicas, que não podiam ser misturadas ou intercambiadas com as leis estruturais do familiar mundo-objeto. Construir com pedra, madeira ou concreto armado sempre tem seus respectivos requisitos estruturais. Construir na superfície bidimensional com elementos bidimensionais também exige um manejo próprio. A eficiência e a força da imagem dependem das estimativas corretas das leis ditadas pelo meio bidimensional empregado.

\par

Os pintores trabalharam em seguida para eliminar os fragmentos remanescentes da representação do objeto, que eles passaram a considerar como um peso morto. A simplificação tinha dois polos. Um era a eliminação sucessiva de todas as características acidentais das unidades da imagem, um retorno aos elementos geométricos básicos --- principalmente à forma retangular --- e à linha reta. O outro era a busca pela máxima precisão possível na relação desses elementos entre si e com a imagem como um todo.
\endgroup

\vfill
\raggedleft
107

\newpage
\section*{Page 110}

Estas implicações inerentes à desagregação do sistema de perspectiva fixa cristalizaram-se em desenvolvimentos diametralmente opostos. Um ocorreu na Europa Oriental, onde uma ruptura final estava sendo feita com os padrões herdados da vida social, e onde o tremendo reservatório de recursos humanos e materiais não utilizados estava sendo liberado. O outro ocorreu na Europa Ocidental --- na Holanda --- onde a turbulência da última guerra mundial havia afetado menos um desenvolvimento pacífico baseado em padrões passados, e onde tudo estava concentrado na preservação das condições disponíveis. Os objetivos e direções de cada um correspondiam ao caráter do contexto social do qual emergiu.

Na Rússia, Malevich, Rodchenko, Tatlin, El Lissitzky e outros levaram adiante a libertação explosiva das forças plásticas. O prazer de expandir e esticar o espaço até que a matéria fosse completamente eliminada era sua força motriz. Os dispositivos visuais característicos que usavam eram o arranjo diagonal dinâmico dos elementos, sua suspensão no espaço de fundo, o vazio que os absorvia. Essa explosão no espaço necessariamente carecia de uma ordem clara como um todo. O plano-imagem era considerado apenas um ponto de partida.

Na Holanda, Doesburg e Mondrian buscavam alcançar a compressão total do espaço, tornada possível pela limitação do plano-imagem bidimensional. Seu ideal era o uso mais econômico das forças plásticas para trazer um equilíbrio dinâmico entre o recuo e o avanço dos planos de cor e linhas na superfície da imagem. Seu trabalho baseava-se na restrição, com o equilíbrio como objetivo. Eles buscavam ordenar o espaço em uma relação perfeitamente medida de cor e linha. Sua análise intransigente da base da expressão plástica teve uma influência decisiva na cultura visual contemporânea. Da arquitetura ao design publicitário, dificilmente há alguma manifestação de atividade visual que possa evitar as implicações dessas duas principais tendências.

\textbf{\textit{Abertura final da superfície da imagem}}

O ambiente em mudança e os novos padrões tecnológicos abriram um novo horizonte do mundo visível. Para abranger essas dimensões mais amplas, os pintores retornaram ao denominador comum mais baixo da representação espacial. Eles redescobriram as forças espaciais no plano-imagem e suas leis de organização inerentes ao processo de percepção visual, condicionadas pela natureza de um plano-imagem bidimensional. Mas o controle criativo dessas leis foi identificado com o próprio espaço. A sensação plástica pura foi dissociada do ambiente visível de onde ela se origina. A perfeição do instrumento que poderia produzir essa sensação de espaço tornou-se um fetiche --- um valor independente. A experiência espacial foi concebida apenas abstratamente. Como Malevich afirmou:

``Todas as ideias sociais, grandes e importantes que sejam, desenvolvem-se a partir da sensação de fome; todas as obras de arte, pequenas e insignificantes que pareçam, emergem de uma sensação plástica. Chega um

108

\newpage
\section*{Page 111}

finalmente, entender os problemas da arte e os problemas do estômago e as razões para cada um, são muito diferentes. Sob o suprematismo, entendo a supremacia da sensação pura nas artes plásticas\ldots do ponto de vista dos suprematistas, a aparência do mundo-objeto é sem sentido; o importante é a sensação como tal, independente das circunstâncias.

A filosofia subjacente a esta purificação final do plano pictórico do mundo-objeto levou à rejeição última de qualquer tentativa de representar a realidade objetiva. ``Tudo o que chamamos de natureza é, em última análise, um quadro de fantasia'', diz Malevich, ``sem a menor semelhança com a realidade.''

Mas, como já foi dito antes, o homem afirma-se no mundo material não apenas por meio do pensamento, mas também por meio de todos os seus sentidos. A arte é uma forma sensível de consciência, um instrumento importante na conquista da natureza, e a representação é a assimilação criativa da natureza. A conquista artística do espaço não é um fim em si mesma, nem é uma questão apenas dos sentidos. Nisto residia a limitação desses pioneiros da linguagem da visão. Eles haviam dado o primeiro passo em direção à liberdade, mas foram impedidos pela perda de fé na existência humana integrada. Seu trabalho foi moldado em pseudomaterialismo e resultou em isolacionismo---da experiência sensorial. A divisão do trabalho, ditada por considerações míopes, criando um indivíduo unilateral, deu origem também à divisão dentro do indivíduo, uma relação hostil entre sentido e razão. Em vez de usar a conquista dos sentidos para uma maior integração do homem com o seu entorno, os pintores, em sua revolta contra um sistema social complexo e sem plano, levaram essa divisão fatal para a esfera da expressão criativa.

No entanto, porque trabalharam com ferrenha honestidade; porque não se contentaram com meias-medidas, mas se entregaram completamente à redescoberta dos materiais com que lidavam; porque trouxeram à sua percepção de novos ambientes visuais sentidos livres da névoa da tradição, a incorreção de algumas de suas atitudes teóricas importa muito menos do que a solidez do fundamento concreto que construíram para o novo controle representacional do mundo visível.

Duas inovações são a seu crédito. Ao reduzir a unidade plástica às formas mais elementares, a uma simplicidade geométrica e a algumas cores básicas, restabeleceram os elementos construtivos genuínos da arquitetura espacial no plano pictórico. Pelo uso do eixo diagonal, contradizendo a ordenação espacial horizontal-vertical aceita, revelaram um poderoso dispositivo para criar uma experiência espacial dinâmica. As formas básicas facilitaram a justaposição de uma forma contra a outra e, assim, manifestaram abertamente as tensões e pressões da experiência que as interligava. Porque o eixo diagonal está em contradição com uma direção principal do espaço, cada forma em posição diagonal tende a girar em direção às linhas principais de organização visual---o eixo horizontal-vertical---aumentando assim a tensão dinâmica.

\par
\raggedleft 109

\newpage
\section*{Page 112}

\textit{Kasimir Malevich, Elementos Suprematistas}

\textit{El Lissitzky, Composição} \\
\textit{Cortesia de Art of This Century}

\newpage
\section*{Page 113}

Kasimir Malevich, \textit{Composição Suprematista}\par
Cortesia da Art of This Century\par
\vspace*{\fill}
\raggedleft 111

\newpage
\section*{Page 114}

A pesquisa em movimentos, tensões e esforços na superfície da imagem teve uma grande influência nas artes aplicadas. Designers de cartazes e vitrines exploraram os recém-descobertos idiomas e mudaram seus métodos de uma simetria estática para um equilíbrio dinâmico elementar.

um guia para estudantes de design\\
para a Feira Mundial de Nova Iorque\\
compilado para\\
revista P/M \dots pela Escola Laboratório\\
de Desenho Industrial

Joseph Binder, Cartaz

Paul Rand, Design de Capa

\newpage
\section*{Page 115}

\begin{flushleft}
M. Martin Johnson,\\
Quatro Designs Publicitários\\
Cortesia da Abbott Laboratories
\end{flushleft}

Inovações da expressão espacial contribuíram também para um rejuvenescimento da tipografia, pois a página impressa é também um plano-pictórico. As possibilidades mecânicas do processo de impressão e a redescoberta de relações plásticas elementares foram testadas em todas as conexões possíveis.

O espaço da página impressa passou a ser conscientemente considerado como um problema plástico. Os elementos foram reduzidos às suas formas geométricas básicas. A redução ao essencial está na fronteira do reconhecimento das formas dos objetos. Cada detalhe desnecessário é eliminado. O olhar do espectador é guiado com uma certeza inconfundível para as formas essenciais e suas relações. A interação das formas básicas possui uma qualidade forte e dinâmica baseada nas claras relações plásticas de cores, formas e linhas ao redor do eixo diagonal.

\vspace{\baselineskip}

\begin{flushleft}
NESTE NÚMERO:
\par
1. Revisão de Ciência Básica -\\
A Coagulação do Sangue.
\par
2. A Vitamina E é Importante\\
na Nutrição Humana?
\par
3. Doença Hemorrágica do\\
Recém-nascido, uma Doença Prevenível.
\par
4. A Escolha de uma Preparação de Bismuto\\
para o Tratamento da Sífilis.
\par
5. Olhando para o Futuro - Heparina Purificada,\\
um Anticoagulante para Uso Clínico.
\par
\vspace{\baselineskip}
Publicado pela The Upjohn Company,\\
Kalamazoo, Michigan.
\end{flushleft}

\vspace{\baselineskip}

\begin{flushleft}
W. Burtin,\\
Design Publicitário
\end{flushleft}

\vfill

\begin{center}
113
\end{center}

\newpage
\section*{Page 116}

\raggedright

A UNICAP AGORA CONTÉM:

Vitamina A 5000 unidades U.S.P.
Vitamina D 500 unidades U.S.P.
Vitamina B$_1$ (Tiamina) 1 mg.
Vitamina B$_2$ (Riboflavina) 1 mg.
Vitamina C 30 mg.
Vitamina B$_6$ 1 mg.
Pantotenato de Cálcio 1 mg.
Niacina 15 mg.
Fração Hepática 2 (integral) 150 mg.
Ferro (como Fosfato Ferroso) 15 mg.
Fosfato de Lactoflavina 1 mg.

\vspace{1em}

\textbf{UMA UNICAP POR DIA...}

um suplemento vitamínico adequado para a dieta deficiente média

\vspace{1em}

Para acompanhar o conhecimento atual de vitaminas e também os princípios enunciados em uma edição recente do Boletim do Conselho Nacional de Pesquisa, a fórmula "Unicap" foi adicionada. É geralmente aceito que a pan-deficiência não é clínica na nutrição humana. Os sintomas que são, de fato, devidos à deficiência de vitaminas na maioria dos casos não são claramente definidos. A deficiência de ácido pantotênico e outros fatores como a colina causam alterações na pele, cabelo e várias glândulas endócrinas.

\vspace{1em}

Vitaminas UNICAP Fornecidas em Frascos de 24, 100 e 250.

\vspace{1em}

A Empresa Upjohn\\
Kalamazoo, Michigan\\
Produtos Farmacêuticos Finos

\vspace{3em}

W. Burtin, \textit{Design Publicitário}

\vspace{3em}

Taylor Poore, \textit{Pôster 1939}

\newpage
\section*{Page 117}

\raggedright

\section*{A construção do espaço na superfície da imagem}

O espaço que o pintor tenta abranger é basicamente a ordem visível dos eventos que ele está vivenciando. A pintura é uma forma de pensar. É, portanto, natural e inevitável que os passos que o pintor dá para formular a experiência espacial sejam condicionados por suas ideias e concepções da ordenação da existência social.

Quando a expansão para o espaço ilimitado, a ruptura com o antigo quadro de referência, foi alcançada, o pintor voltou-se novamente para a busca de uma ordem concreta. Ele estendeu a mão para o que lhe parecia a única ordem positiva na vida como a via --- a ordem da máquina, a construção precisa e fria do engenheiro. O avanço tecnológico, com sua precisão e economia, parecia ser a única chave para a melhoria das condições sociais. Parecia que a redução do custo de produção resolveria o caos social. O técnico foi saudado como o profeta de uma nova ordem social. E o artista procurou aliar-se ao profeta.

Rodchenko,\\
\textit{Composição 1918}\\
\textit{Cortesia do Museu de Arte Moderna}

Rodchenko,\\
\textit{Composição 1919}

Rodchenko,\\
\textit{Construção de Linha 1920}

A técnica foi identificada com a arte, e a arte da técnica foi considerada uma força independente de mudança social. Novamente, a abordagem estava incompleta. Um elemento havia sido confundido com o todo.

Os pintores combinaram os elementos flutuantes e estereométricos e os soldaram numa construção que tem seu modelo na máquina. A máquina é a fonte de inspiração, não apenas em suas qualidades superficiais e forma exterior, mas também em seu princípio de construção. Linhas, planos e formas são combinados em uma nova interconexão dinâmica, transparente e impenetrável. Ferramentas mecânicas, como o compasso e a régua, entraram em voga à medida que os pintores se esforçavam para alcançar a maior semelhança possível com a máquina.

\vfill
\raggedleft
115

\newpage
\section*{Page 118}

Frank Levstik, Estrutura de Aço. Fotografia

L. Moholy Nagy, Construção em Brancos 1928

116

\newpage
\section*{Page 119}

O uso predominante de materiais contemporâneos na construção, estruturas de aço, paredes de vidro, etc., adicionou mais inspiração. Os vãos abertos, conferindo leveza à construção e permitindo um fluxo do espaço interno; as novas relações de carga e suporte, proporcionando uma visão clara dos mecanismos espaciais do edifício, atuaram como estímulos. Na arquitetura, havia superfícies abertas e transparentes em vez de paredes sólidas; no plano pictórico, da mesma forma, em vez de superfícies opacas, passou a existir a interpenetração transparente de planos e os esqueletos abertos de linhas.

A rede aberta de linhas se expande em várias direções no espaço, e uma espécie de balanço óptico é alcançado---uma construção espacial dinâmica.

\raggedleft
Ladislav Sutnar. Design de Publicidade
\par

\noindent\textit{informações essenciais do produto}
\par
\noindent\textit{acessível, atualizado}
\par
\noindent\textit{elimina desperdício, acelera a produção}
\par
\vspace{0.5em}
\noindent\textit{sistema de informação do produto}
\par
\vspace{0.5em}
\noindent\textit{desperdício por falta de coordenação}
\par
\noindent\textit{eliminação de desperdício através da coordenação}
\par
\vspace{0.5em}
\noindent\textit{produtos 1 2 3 4 5}
\par
\noindent\textit{fabricante a}
\par
\noindent\textit{fabricante b}
\par
\noindent\textit{fabricante c}
\par
\noindent\textit{fabricante d}
\par
\noindent\textit{fabricante e}
\par
\noindent\textit{produto}
\par
\vspace{0.5em}
A crescente necessidade de velocidade na produção de guerra se reflete na demanda crescente por informações do produto.
\par
Para ser útil, tal informação do produto deve ser abrangente, concisa,
\par
O pré-arquivamento de catálogos foi desenvolvido como um meio para controlar o fluxo de informações essenciais do produto.
\par
\vspace{0.5em}
\noindent\textbf{Serviço de Catálogos Sweet's}
\par
\noindent\textit{distribuição - impressão - design}
\par
\vspace{0.5em}
\noindent\textit{design Sutnar}
\par

\newpage
\section*{Page 120}

\raggedright
\textbf{voltando para a escola...}

Paul Rand. Design de Publicidade

118

\newpage
\section*{Page 121}

\section*{Fechando a superfície: completo relacionamento das forças espaciais}

As condições no mundo em que o pintor viveu clamavam por ordem. A ciência
e a tecnologia haviam avançado; eles haviam negligenciado totalmente a domesticação,
no plano social, dos novos campos para os quais se moveram. Males sociais,
luta doméstica e internacional, desemprego e energias não utilizadas,
dias mal organizados e, finalmente, um indivíduo desajustado foram os frutos
dessa negligência.

Nesta caótica existência social falsificada, onde quase todos os materiais
foram mal utilizados, incluindo o próprio ser humano, a arquitetura deu o
primeiro passo concreto para construir honestamente em termos do presente. Arquitetos
pioneiros reconheceram que o novo conhecimento exigia um novo princípio
de construção, que para fazer uso da compreensão científica das qualidades
estruturais, como tensão, deformação, peso e carga; dos novos
materiais, aço, vidro, concreto, eles deveriam primeiro remover todos os detritos dos
estilos herdados. As máquinas e a produção em massa tornaram a
imitação de estilos passados cada vez mais obsoleta. Quarenta anos atrás, Frank
Lloyd Wright disse:

\begin{quote}
``Uma necessidade estrutural que moldou Panteões, monumentos e
templos foi reduzida pela máquina a um esqueleto de aço, com-
pleto em si mesmo sem o toque do artista artesão. . . . A estrutura de aço
havia sido reconhecida como uma base legítima para um revestimento
simples e sincero de material plástico que revela sua natureza essencial
e idealiza seu propósito sem afetação ou pretensão estrutural. A máquina
suaviza a necessidade da tentação de enganos estruturais mesquinhos,
acalma a luta cansativa para fazer com que as coisas pareçam o que não
são e nunca poderão ser. . .''
\end{quote}

A nova disciplina da honestidade estrutural tem importantes implica-
ções práticas. Na melhor arquitetura moderna, um edifício começa, não pelo
exterior -- a fachada -- mas pelo interior -- a planta baixa. As paredes
articulam este espaço dividindo-o e subdividindo-o. Elas excluem o
exterior, oferecendo proteção contra chuva, vento e sol, mas também modelam
o espaço interior e o ritmo da vida dentro dele. As paredes horizontais e verticais
estão em uma relação clara e funcional. A profundidade das paredes, planos
recuados e avançados, articulam o espaço em uma ordem dinâmica de
viver. O resultado é ordem estrutural, um equilíbrio do organismo
em funcionamento, um espaço vivo.

\newpage
\section*{Page 122}

R. B. Tague. \textit{Análise dos Planos de Recuo e Avanço de uma Casa de Frank Lloyd Wright.}

O espaço vital implica um equilíbrio perpétuo de direções opostas. As dimensões ampliadas do conhecimento humano exigem um novo equilíbrio entre o homem e a natureza e entre o indivíduo e a sociedade. Como disse Corbusier: ``O indivíduo e a comunidade nessa relação corretamente proporcionada que é o equilíbrio da própria natureza---tensão entre dois polos. Se há apenas um polo, os resultados tendem a zero. Extremos destroem a vida, pois a vida segue um curso médio entre os extremos. O equilíbrio indica a presença de movimento contínuo e infalível. Sono, torpor, letargia e morte não são um estado de equilíbrio. Equilíbrio é o ponto onde todas as forças se encontram e se resolvem---o ponto de equilíbrio. Assim, o futuro urbanista pode ler o futuro destino da sociedade.''

O pintor Mondrian expressou a mesma compreensão no domínio das artes plásticas. Ele escreve:
``Toda expressão de arte tem suas próprias leis que concordam com a lei principal da arte e da vida: a do equilíbrio. Dessas leis depende o grau de equilíbrio que pode ser alcançado e, portanto, em que ponto o desequilíbrio pode ser destruído.
``Na Natureza, uma libertação completa do sentimento trágico não é possível. Na vida, onde a forma física não é apenas necessária, mas da maior importância, o equilíbrio será sempre muito relativo. Mas o homem, evoluindo em direção ao equilíbrio de sua dualidade, criará em grau cada vez maior na vida, como na arte, relações equivalentes e, portanto, equilíbrio. A vida social e econômica hoje já demonstram seu esforço em direção a um equilíbrio exato. A vida material não será para sempre ameaçada e tornada trágica. Nem nossa vida moral será sempre oprimida pela dominação da existência material.''

Isso expressa a principal corrente do pensamento contemporâneo pioneiro em todos os campos do esforço humano; o desejo ardente de compreender e ordenar as forças que agem cegamente em nossa vida hoje.

\begin{small}
$\bullet$ Piet Mondrian, \textit{Arte Plástica Pura}, 1942
\end{small}

120

\newpage
\section*{Page 123}

Este pensamento é ordem e honestidade---nos termos da expressão plástica---um equilíbrio perfeito dos elementos---equilíbrio identificado com a própria superfície bidimensional.

O espaço é concebido pela sua contração às duas dimensões. A extensão é expressa pela sua contração às duas dimensões. Os movimentos espaciais individuais dos planos de cor são medidos e expressos pela tensão criada ao puxá-los de volta para as duas dimensões. O objetivo é o controlo perfeito, uma ordem coletiva. Nenhum elemento pode viver sozinho. A sua vida é também a do todo bidimensional, o que só é possível se todos os elementos estiverem perfeitamente relacionados e equilibrados. O plano da imagem tornou-se como uma membrana esticada. Cor, formas e linhas, estendendo-se no espaço para cima e para baixo, lateralmente, para dentro e para fora em profundidade, efetuaram uma relação precisa na qual qualidades espaciais opostas e em movimento se equilibravam no plano da imagem bidimensional com uma precisão quase matemática.

A vitalidade de qualquer equilíbrio depende da força das forças opostas que estão em balanço. Em termos visuais, depende também de quão abertamente essas forças se afirmam. Para atingir este máximo de equilíbrio dinâmico, a superfície da imagem foi construída a partir de opostos básicos, formas retangulares e linhas retas horizontais e verticais, azul, vermelho e amarelo, cores puras.

\begin{quote}\small
``Portanto, a arte tem de atingir um equilíbrio exato através da criação de meios plásticos puros compostos em oposições absolutas. Desta forma, as duas oposições (vertical e horizontal) estão em equivalência, ou seja, têm o mesmo valor: uma necessidade primordial para o equilíbrio. Por meio da abstração, a arte interiorizou a forma e a cor e levou a linha curva à sua tensão máxima: a linha reta. Usando a oposição retangular---a relação constante--- estabelece a dualidade individual universal: unidade.''\\
Piet Mondrian.
\end{quote}

\small
Piet Mondrian.\\
\textit{Composição}\\
Cortesia de\\
\textit{Art of This Century}

\vfill
\null\hfill 121

\newpage
\section*{Page 124}

\raggedleft
DIE STIJL

\vspace{0.5em} % Small vertical space

HAGEMEIJER \& CO \\
AMSTERDAM \\
PRINSHENDRIKKADE 159 \\
TELEFOON 64777
\par
\raggedright % Reset to default left alignment for body text

\bigskip % Standard vertical space

Theo Van Doesburg. Desenho e Design Tipográfico

\bigskip % Standard vertical space

O novo princípio de ordenamento e a redescoberta da natureza genuína do plano bidimensional da imagem tiveram uma influência rejuvenescedora sobre pintores e outros. Pintores e designers, confrontados pelos exemplos perfeitos de relação visual e disciplina criativa, tornaram-se críticos do seu próprio trabalho e começaram a compreender o meio que manipulavam. A formulação cristalina da ordem plástica serviu como um espelho para mostrar todas as concepções erradas e todas as rejeições do uso honesto da superfície bidimensional. Ajudou também a evidenciar a discrepância entre a natureza genuína dos respectivos materiais e os seus usos atuais. Tipografia, design de produto e todos os outros campos da criação ótica, ganharam com a reanálise das leis inerentes do seu meio e pela busca por um melhor equilíbrio.

Os próprios pintores direcionaram os primeiros passos nesta busca. Doesburg, já em 1916, aplicou as descobertas à tipografia. Sua reanálise dos princípios estruturais fundamentais das artes plásticas teve uma influência de longo alcance no design de publicidade e na tipografia. Elementos horizontais e verticais em uma relação clara e contrastante levam a uma subdivisão da superfície com equilíbrio dinâmico. Arranjos simétricos de letras e elementos retangulares simples abriram o caminho para uma nova tipografia cuja lógica espacial interna é ditada pela natureza da percepção visual com ênfase funcional na mensagem. Novas fontes foram projetadas, baseadas nos princípios visuais descobertos pelos pintores.

A reforma no campo da tipografia era lamentavelmente necessária. Nossa forma atual de escrita é um aglomerado incompletamente assimilado de sinais que refletem diversas origens históricas e várias ferramentas. Uma unidade harmoniosa entre esses elementos nunca foi alcançada. Letras maiúsculas e minúsculas nunca possuíram unidade formal, mesmo nas melhores formas tipográficas clássicas. A letra impressa é um fóssil histórico em conformidade nem com as leis da organização visual nem com a psicologia tecnológica do homem. Em vez de ser livre para escolher seu próprio caminho ao se mover pela página, o olho é regimentado ao longo de uma linha de impressão controlada por padrões tipográficos tecnicamente arcaicos.

\vfill % Pushes the page number to the bottom

122

\newpage
\section*{Page 125}

Piet Mondrian. Pintura 1932.

Produtos de madeira 4-square.

Weyerhaeuser Sales Company, St. Paul, Minnesota.

$\frac{2}{2}$

Ladislav Sutnar. Projeto de Capa.

\newpage
\section*{Page 126}

\section*{Instalação}

\begin{itemize}
    \item Parede Exposta
    \item Piso do Elevador
    \item Soldagem
    \item Assentamento
    \item Cargas Rolantes
    \item Resistente a Cargas Estáticas
    \item Corredores
    \item Pisos
    \item Armazenamento
\end{itemize}

\vspace{1em} % Add some vertical space between logical blocks

\section*{Especificações de Instalação}

\subsection*{Subpiso de Madeira Suspenso}
Antes da instalação, certifique-se de que o subpiso esteja limpo, seco, nivelado e firmemente fixado. Tábuas soltas devem ser pregadas. Se a superfície estiver irregular ou áspera, aplique um composto de nivelamento adequado ou instale uma nova camada de compensado (mínimo de 1/4 de polegada) firmemente pregada a cada 6 polegadas em um padrão de grade. Certifique-se de que todas as cabeças dos pregos estejam abaixo da superfície.

\subsection*{Subpisos de Concreto: Novos}
Lajes de concreto novas devem estar completamente curadas e secas antes da instalação do piso. Permita pelo menos 60 dias para a cura. A superfície deve estar lisa, limpa e livre de toda poeira, sujeira, graxa, óleo, tinta, compostos de cura e substâncias estranhas. Teste o teor de umidade usando um teste de cloreto de cálcio. Quaisquer pontos altos devem ser lixados e os pontos baixos preenchidos com um composto de reparo adequado.

\subsection*{Subpisos de Concreto: Antigos}
Pisos de concreto existentes devem estar limpos, secos e livres de todo material estranho, incluindo resíduos de adesivo antigo, tinta e cera. Remova todo o concreto solto, descascando ou desintegrado. Preencha todas as rachaduras, buracos e depressões com um composto de reparo adequado à base de cimento Portland. Lixe quaisquer pontos altos. Teste o teor de umidade. Se houver umidade, um sistema de barreira de umidade deve ser instalado.

\subsection*{Subpiso de Mármore, Terrazzo, Ladrilhos Cerâmicos ou Metal}
Essas superfícies devem ser lisas, limpas, secas e firmemente aderidas. Certifique-se de que não há ladrilhos soltos ou rachaduras. Quaisquer ladrilhos soltos devem ser re-aderidos, e rachaduras preenchidas. Superfícies de terrazzo e mármore podem precisar ser levemente abrasadas para garantir uma boa aderência do adesivo. Subpisos de metal devem estar livres de ferrugem, limpos e rígidos. Aplique um primer adequado em superfícies metálicas, se recomendado pelo fabricante do adesivo.

\section*{Recomendações de Manutenção}
Para garantir a longevidade e a aparência do Ladrilho de Asfalto Industrial da Armstrong, a limpeza regular é essencial. Varra ou passe pano de pó diariamente para remover sujeira solta e grãos. Passe um pano úmido conforme necessário com um limpador neutro. Evite detergentes agressivos ou água em excesso. Limpe derramamentos imediatamente. Se for necessária uma limpeza profunda, use uma máquina de piso com as almofadas apropriadas. Não use limpadores à base de solvente ou cera em pasta.

\section{Cooperação Armstrong}
A equipe técnica da Armstrong está disponível para auxiliar com perguntas específicas sobre instalação e manutenção. Para especificações detalhadas, fichas de dados de produtos ou assistência com projetos complexos, entre em contato com seu representante da Armstrong. Nosso objetivo é garantir a aplicação e o desempenho bem-sucedidos de nossos produtos em seus ambientes industriais. Informações sobre outros produtos Armstrong também estão disponíveis mediante solicitação.

\newpage % Represents the visual separation for the grey column content

\section*{Especificações de Instalação de Ladrilhos de Asfalto Industriais da Armstrong}

O Ladrilho de Asfalto Industrial da Armstrong é adequado para instalação em pisos de fábrica ou pisos de uso pesado que estão sujeitos a tráfego intenso de empilhadeiras. Pode ser usado em subpisos de concreto ou em subpisos de madeira ou metal acima do nível do solo, no nível do solo e suspensos. É resiliente, resistente e durável. Este material é feito de uma composição cuidadosamente equilibrada e, portanto, resiste à deformação. Não use o Ladrilho de Asfalto Industrial da Armstrong em locais sujeitos a condições excessivas de óleo, graxa ou umidade latente excessiva no solo.

\subsection*{Tipos Típicos de Instalação:}
\begin{itemize}
    \item Armazéns de Fábrica
    \item Salas de Bateria
    \item Túneis de Ponte
    \item Grandes Fábricas
    \item Vestiários
    \item Corredores
    \item Salas de Faculdade
    \item Salas de Aula de Escola
    \item Dormitórios
    \item Salas de Jantar
    \item Edifícios de Escritórios
    \item Gráficas
    \item Lojas de Varejo
    \item Salas de Venda
    \item Salas de Expedição e Recebimento
    \item Cozinhas de Restaurante
    \item Lojas e Corredores de Hotel
    \item Salas de Bilhar
    \item Áreas de Carga
    \item Hospitais
    \item Salas de Trabalho
\end{itemize}

Armstrong Cork Company\\
Construindo os Pisos Certo. Lancaster, Pa.

\vspace{2em} % Visual separation for the bottom text

Ladislav Sutnar, Projeto de Catálogo 1942

\newpage
\section*{Page 127}

Lester Beall. Design de Publicidade

por que

que informação essencial do produto

como

???

Ladislav Sutnar. Design de Publicidade

\newpage
\section*{Page 128}

A redescoberta da ordem em termos de experiência plástica foi condicionada pelo pano de fundo social, pela necessidade urgente de um equilíbrio no plano socioeconômico. Mas essa ordem só poderia ser alcançada por um ataque frontal à própria base das contradições sociais. Esforços que evitaram o desafio aberto às causas das contradições poderiam alcançar apenas um semi-equilíbrio sacrificando aspectos vitais de uma vida humana integral. A liberdade individual, indisciplinada, tornou-se uma licença e gerou uma perda de fé nas qualidades individuais. A regulamentação, o sacrifício do indivíduo, tornou-se o novo conceito social de regressão e meias-medidas. Este conceito ancorou-se inclusive no reino do pensamento plástico.

Para alcançar um equilíbrio perfeito com as duas dimensões da imagem, as qualidades plásticas individuais foram sacrificadas. A riqueza da variedade de formas, a riqueza de cores e valores foram reduzidas a equações plásticas estereotipadas de formas retangulares. A ordem tornou-se um fim em si mesma, em vez de um princípio orientador. Ela criou seu próprio mundo – um mundo de restrições puritanas. Essa clareza contida de equilíbrio impôs uma restrição rígida a uma ordem posterior. Buscando eliminar todas as impurezas, ela também tendeu a eliminar muitas variedades de experiência visual.

Claramente, era necessário o arcabouço de um equilíbrio que tivesse espaço para a individualidade dos elementos. Esse equilíbrio começa a aparecer. Doesburg se opôs a dois sistemas, o horizontal-vertical e o diagonal. Helion começou a preencher, passo a passo, o terreno abstrato, trabalhando com uma nova variedade de elementos plásticos, modelando as formas planas, e torcendo e esticando os retângulos em novas formas e unindo-os em composições.

\raggedleft
Theo Van Doesburg.\\
\textit{Composição Aritmética 1930}

126

\newpage
\section*{Page 129}

\noindent Jean Hélion. \textit{Linogravura}.
\vfill
\raggedleft 127

\newpage
\section*{Page 130}

Jean Helion. Linoleogravura 1936

\vspace*{\fill}
128

\newpage
\section*{Page 131}

\subsection*{\textit{Adaptação ao ambiente contemporâneo}}

Inovações em expressões representacionais causaram um progresso importante em direção ao domínio óptico das experiências espaço-tempo contemporâneas. Mas a comunicação visual só pode ser eficiente se se submeter à nova paisagem e à nova psicologia do homem contemporâneo. E, simultaneamente com o domínio do novo espaço mais amplo, a comunicação visual foi forçada a fazer algumas adaptações significativas à cena contemporânea.

O número de pessoas em uma audiência define não apenas a qualidade e a intensidade da voz do orador, mas também a natureza da fala. O caráter de um diálogo é naturalmente diferente de um discurso para uma reunião em massa. A pintura de cavalete, a expressão de um período histórico, desenvolveu uma forma de diálogo visual. Falava uma linguagem de ``tête-à-tête''. Foi a manifestação histórica na arte pictórica do espírito do individualismo. Mas o pano de fundo histórico mudou. As dimensões individuais estão perdendo sua significância inflada. Assim como os pontos estão em uma linha ou linhas em um plano, o indivíduo foi reconhecido como insubstituível apenas nos termos da dimensão mais ampla. A interdependência social trouxe um novo significado para o indivíduo, o indivíduo social. Para falar com este novo homem, era necessária uma linguagem diferente, uma linguagem que deve penetrar profundamente em regiões individuais, mas ao mesmo tempo falar com o maior grupo possível. Isso significa falar simultaneamente para muitos. O número da audiência exige uma amplificação do som e uma simplificação da linguagem para o interesse comum e os idiomas comuns. O microfone ajuda a ajustar a voz às maiores dimensões da audiência. O espectador em massa exige a amplificação da intensidade óptica e uma simplificação da linguagem visual em direção a idiomas comuns. Tais idiomas exigem simplicidade, força e precisão.

\begin{quote}
``Na minha busca por brilho e intensidade, fiz uso da máquina como outros artistas utilizaram o corpo nu ou naturezas-mortas. .... Nunca me diverti copiando uma máquina. Eu invento imagens de máquinas como outros, com sua imaginação, inventaram paisagens. .... O elemento mecânico em meu trabalho não é um preconceito nem uma atitude, mas um meio de dar uma sensação de força e poder.''
\end{quote}

\textbullet\ \textit{Fernand Leger, Propos d'artistes, 1925}

\vspace*{\fill}
\begin{flushright}
129
\end{flushright}

\newpage
\section*{Page 132}

\large\textit{Simplicidade e intensidade}

Sinais de trânsito, recentemente importantes em um mundo em movimento, são as declarações visuais mais simples projetadas para o observador móvel. Eles são intensos em cor, simples em forma, e cada um deles constitui claramente uma unidade.

Máquinas, automóveis, bondes, trens elevados, aeroplanos, displays de luz cintilante, vitrines, tornaram-se características comuns da cena contemporânea. Juntamente com a nova riqueza de efeitos de luz de fontes de luz artificial, a dimensão aumentada da paisagem com os arranha-céus e sua intrincada ordem espacial interna acima, e os metrôs por baixo, eles deram uma velocidade e densidade incomparavelmente maiores às estimulações de luz que atingem o olho do que qualquer ambiente visual anterior jamais havia apresentado.

Não há tempo agora para a percepção de muitos detalhes. A duração dos impactos visuais é muito curta. Para atrair o olho e transmitir o significado completo nesta turbulência visual de eventos, a imagem deve processar, como o sinal de trânsito, simplicidade de elementos e força lúcida.

\large\textit{Precisão}

A produção industrial introduziu novos objetos: máquinas e produtos de máquinas, unidades padronizadas e prontas. Eles foram produzidos com a máxima precisão e controle ditados pelas necessidades funcionais, utilidade e economia. Na confusão do mundo-objeto circundante, essas coisas apareceram como as únicas criações humanas de perfeição, coordenação e sentido. A clareza funcional mecânica da máquina, a harmonia perfeita de suas partes e a rigidez inconfundível de suas relações internas foram uma inspiração para homens que buscavam qualidades semelhantes na imagem. Clareza, precisão e economia eram valores imperiosos em um mundo sufocando sob o peso morto do individualismo indisciplinado.

Leger diz:
\begin{quote}
``A técnica deve ser cada vez mais exata, a execução deve ser perfeita.\ldots Prefiro uma pintura medíocre perfeitamente executada a uma imagem, bela na intenção, mas não executada. Hoje em dia, uma obra de arte deve suportar comparação com qualquer objeto manufaturado. Somente a imagem, que é um objeto, pode sustentar essa comparação e desafiar o tempo ou \ldots Eu nego absolutamente o objeto como um fator que reage em um conjunto plástico.''
\end{quote}

\raggedright 130

\newpage
\section*{Page 133}

Lester Beall. Cartaz

BEALL

RÁDIO

ADMINISTRAÇÃO DE ELETRIFICAÇÃO RURAL

\newpage
\section*{Page 134}

F. Levstik. Fotografia

F. Leger. Pintura \\
Cortesia do The Museum of Modern Art

132

\newpage
\section*{Page 135}

Fernand Leger, Pintura.

A. M. Cassandre. Pôster, com o texto "DUBONNET" e "VINHO TÔNICO COM QUININA". Cortesia do Museu de Arte Moderna.

\newpage
\section*{Page 136}

\centering\itshape Luz e cor\par

\indent A experiência espacial está intimamente conectada com a experiência da luz. Sem luz não há visão, e sem visão não pode haver espaço visível. Espaço no sentido visual é espaço-luz. Geralmente este espaço-luz não é aparente ao olho. Percebemos as relações espaciais apenas quando a luz é interceptada por algum meio. O que realmente vemos como mundo espacial é a maneira pela qual a luz é dissecada e redirecionada, isto é, modulada por esses meios. Os modos sensoriais de registrar a luz modulada, as várias sensações de cor, então, tornam-se os meios para a ordenação espacial de objetos e eventos.\par

\indent Mas a experiência da luz, a cor significa mais do que dados sensoriais do mundo espacial. A palavra luz ou cor conota riqueza, saúde e totalidade. A luz, assim como a cor, não é meramente um sinal espacial do ambiente; é uma necessidade humana básica. Na ânsia por cor expressa-se uma das mais profundas apreensões da realidade pelo homem. ``A luz, portanto, usando o pleno significado da palavra, transmite energia que é o suporte da vida, e dá aos seres vivos o poder de observação; e é semelhante à matéria de que são feitas todas as coisas animadas e inanimadas. O universo é sua esfera de ação. Não fazemos mais do que justiça ao falarmos do Universo da luz.'' \textbullet\par

\indent A luz é a energia básica vital para qualquer existência orgânica. A orientação, em seu significado fundamental, é a adaptação do homem às energias solares contidas na infinita variedade das formas da natureza. A experiência da luz\,---\,em outras palavras, a sensação de cores\,---\,representa a segurança do organismo e, portanto, possui uma qualidade de afirmação. Experimentar a cor é interpretar o cerne da realidade física em termos de qualidades sensoriais. Quando se veem cores, sem a noção de que residem nos objetos, a reação sensorial de uma pessoa tem conotações que se originam de sua compreensão da luz como a condição básica da vida. A sensação de cor é sempre, portanto, um símbolo de satisfação do sistema nervoso.\par

\bigskip
\textbullet \textit{Sir William Bragg, O Universo da Luz}\par
\bigskip

134\par

\newpage
\section*{Page 137}

\centering\textit{\textbf{As fontes da experiência da cor}}
\par\bigskip

\noindent A cor é uma experiência---um evento psicológico. A luz e as diferentes distribuições de luz, por absorção, dispersão e difração, são incolores. Elas se tornam cor apenas quando passam pela estrutura do conjunto receptor visual e são registradas pelo cérebro. A experiência da cor tem, então, três fontes básicas. A primeira é a matéria-prima física, a energia radiante modulada pelo ambiente. A segunda são os dados fornecidos diretamente pelos sentidos. E a terceira são os dados fornecidos pela memória, incluindo associações induzidas por alguma correspondência entre a estrutura da estimulação sensorial atual e as anteriores, ou pela conexão repetidamente experimentada entre uma estimulação sensorial particular e um evento.

\par\bigskip
\par\bigskip

\centering\textit{\textbf{A modulação física da luz}}
\par\bigskip

\noindent A luz pode ser percebida diretamente como fonte de luz --- o sol, fogo, luz elétrica, gás luminoso, etc. --- colorida ou sombreada por sua própria intensidade.

\par\bigskip

\noindent A luz pode ser modulada em escala submicroscópica e percebida como valor intrínseco constante ou cor. O branco do papel, o verde das folhas, o preto do veludo --- são resultados da modulação da luz incidente pela estrutura submicroscópica das respectivas substâncias.

\par\bigskip

\noindent A luz pode ser modulada em uma escala mais grosseira pela extensão tridimensional dos objetos. Então, percebe-se a forma escultural através da modelagem por sombreamento.

\par\bigskip

\noindent A luz pode ser modulada e articulada pelas várias substâncias, como no lançamento de sombras ou na reflexão, difundindo a luz; isto é, pode ser observada como o bloqueio, clareamento, curvatura dos corpos luminosos opticamente latentes anteriormente. Então, percebe-se a luz que preenche o espaço.

\vfill

Frank Levstik.

Fotografia

\newpage
\section*{Page 138}

\raggedright
Milton Halberstadt, \textit{Fotografia}

R. B. Tague \& W. W. Keck. \textit{Estudo de Luz}\\
Escola de Design em Chicago.\\
Oficina de Luz

136

\newpage
\section*{Page 139}

Nathan Lerner. Estudo do Espaço de Luz.\\
Escola de Design em Chicago. Oficina de Luz.

\newpage
\section*{Page 140}

\noindent{\large\textit{A fonte das sensações de cor no aparelho receptor}}

Distinguimos três qualidades diferentes de sensação de cor: matiz, ou cor; brilho, ou valor; e saturação, ou profundidade. Todas são baseadas na avaliação fisiológica das fontes físicas.

O matiz, ou a cor real, é induzido pelas diferenças no comprimento de onda das energias radiantes e pela estrutura particular da superfície retiniana sobre a qual atuam. Essa interação única do agente luminoso e da estrutura retiniana fornece o fundamento da qualidade de sensação de vermelho, amarelo, azul e assim por diante. A duração da estimulação desempenha um papel decisivo na sensação de matiz. A estimulação luminosa deve ter uma certa duração para induzir a sensação de cor; intervalos curtos produzem apenas a sensação de brilho.

Brilho, ou seja, a sensação de que uma cor parece mais clara ou mais escura que outra, governa o valor da cor. É condicionado em parte pela intensidade da estimulação, em parte pela estrutura neural da retina. A sensibilidade desigual da retina a diferentes comprimentos de onda determina em grande parte quais cores parecem mais luminosas ou mais brilhantes que as outras. O amarelo, por exemplo, parece mais brilhante que o azul ou o verde.

Saturação é a medida do conteúdo real de cor em uma dada sensação. Quando vemos um vermelho como mais vermelho que outro, experimentamos uma qualidade sensorial particular manifestada em uma pureza menor ou maior, tornando as cores mais ou menos ricas e plenas. Vermelho e rosa, amarelo intenso e amarelo pálido são percebidos como qualidades de sensação diferentes. A duração da estimulação afeta a saturação da cor. Uma estimulação muito longa reduzirá a saturação. Uma intensidade muito baixa ou muito alta tende progressivamente a eliminar a saturação. Além disso, a estrutura da retina modifica a saturação. Algumas cores perdem sua profundidade através da estimulação da periferia do campo retiniano.

\noindent{\large\textit{Interação dinâmica das sensações de cor}}

Nunca se registram sensações de cor isoladas. O campo visual normalmente consiste em inúmeras qualidades ópticas e, portanto, as sensações de cor só podem ser percebidas em uma interação dinâmica de diferentes tipos de estímulos retinianos. As inter-relações dinâmicas das sensações de cor são, então, a fonte das características mais importantes das experiências de cor; ou seja, contraste e valor espacial.

138

\newpage
\section*{Page 141}

Matiz, brilho e saturação de uma superfície são modificados pelas superfícies adjacentes. O efeito de contraste é sempre na direção da maior oposição de cores. Se uma superfície vermelha e uma verde forem justapostas na mesma superfície da imagem, o vermelho parecerá mais vermelho do que pareceria se visto em um fundo colorido de matiz mais próximo. Da mesma forma, o verde parece mais verde quando visto em um fundo amarelo, azul ou marrom. Se uma superfície cinza for cercada por uma superfície colorida, o cinza assumirá uma tonalidade complementar à cor que o envolve. Se a cor circundante for vermelha, a cor induzida do cinza será esverdeada; se a cor circundante for verde, a cor induzida parecerá avermelhada; se azul, amarelada, e assim por diante. O efeito de contraste será mais potente quando o cinza tiver um brilho igual ao da cor adjacente e quando essa cor for altamente saturada, ou seja, quando o contraste de brilho for reduzido ao mínimo. O grau do efeito de contraste está em relação direta com a proximidade das cores entre si na superfície da imagem. Se as superfícies coloridas forem divididas com linhas pretas ou coloridas, o resultado contrastante será diminuído em proporção direta à largura das linhas. O resultado do contraste é mais emocionante quando a saturação das cores é a maior. As tonalidades no extremo azul do espectro manifestam um contraste mais forte do que as cores no extremo vermelho do espectro.

As superfícies coloridas também são modificadas em suas áreas. Uma figura de cor clara em uma área escura parece maior do que uma figura de cor escura do mesmo tamanho em um fundo claro. Uma superfície branca parece expandir mais, e uma preta contrastar mais. O amarelo parece maior que o verde, o azul menor. Brilho e saturação são fatores importantes nessas mudanças relativas. Cada diferença de brilho amplifica a intensidade da outra, melhorando assim a irradiação resultante, ou seja, a expansão das cores. Goethe observou que o vermelho amarelado parece "furar o olho". Por outro lado, "assim como gostamos de perseguir um objeto agradável que se afasta de nós, assim gostamos de olhar para o azul, não porque ele nos pressione, mas porque nos atrai". Um olho construído para focar a luz vermelha de uma distância infinita na retina pode fazer o mesmo com raios violetas de uma distância de apenas sessenta centímetros. Por essas e outras intrincadas razões fisiológicas, matiz, brilho e saturação em sua inter-relação dinâmica no campo visual são percebidos como avançando, recuando ou circulando, ou parecem ter pesos diferentes---caindo ou flutuando.

"Assim, percebe-se que os processos de cor desempenham um papel duplo na função de espaço-cor da visão; eles contribuem com a matéria ou substância do campo visual e, ao mesmo tempo, determinam a forma como o campo é organizado tanto bidimensional quanto tridimensionalmente. Se a cor ou o espaço devem ser considerados primários, é muito cedo para decidir, mas as evidências atuais apontam para o crescente reconhecimento da importância da cor para as discriminações espaciais."

\textbullet{} Harry Helson, \textit{Problemas de Constância de Cor}, Journal of the Optical Society of America, Vol. 33, No. 10

\newpage
\section*{Page 142}

\begin{center}
\large\textit{A fonte de memória da experiência de cor}
\end{center}

As impressões retinianas são instantaneamente sobrepostas pela memória de experiências anteriores. O azul sugere imediatamente o céu azul; o verde, as gramas verdes; o branco, a neve branca. Experimentamos as estimulações de cor principalmente em referência ao mundo dos objetos e, consequentemente, a cor significa a cor dos objetos.

Essa sobreposição de memória também tende a manter a cor do objeto relativamente inalterada, apesar das mudanças de iluminação. Uma superfície branca, embora tingida pela luz atmosférica mutável para avermelhada, amarelada ou azulada, é percebida constantemente como branca.

"Até tempos muito recentes, a tez do homem era concebida como essencialmente permanente. Pelo menos as fortes mudanças que realmente ocorrem em diferentes posições não foram pintadas até tempos muito recentes. Uma pessoa de tez clara, parada entre um arbusto verde e uma parede de tijolos vermelhos, certamente tem o rosto verde de um lado e vermelho do outro, e se o sol brilha em sua testa, pode ficar intensamente amarelo às vezes. Ainda assim, não estamos, ou pelo menos não estávamos, acostumados a representar esses traços eminentemente realistas. Preferimos concentrar nossa atenção no que é permanente na tez individual, como visto na luz do dia difusa comum. Estamos acostumados a ver as luzes momentâneas acidentais enfraquecidas em favor da impressão permanente." A cor parece residir nos objetos, totalmente independente da iluminação.

Da memória também vem outro tipo de associação. Ver um objeto significa mais do que colocá-lo em um quadro de referência do mundo tridimensional. Mesmo enquanto se vê a cor como substância, também se a vê como fria ou quente, brilhante, alegre, triste, deprimente, irritante, agradável, grosseira, refinada, selvagem, domesticada, excitante, relaxante, suja, limpa, rica e possuidora de inúmeras outras qualidades sensoriais. Essas associações têm sua origem em parte no processo neuromuscular, mas também em parte na soma total das outras sensações dominantes conectadas à cor vista. O vermelho da flor, o azul do céu, o branco da neve trazem de volta sentimentos que já se tem por essas coisas. Quando se diz que se vê água fria ou um vermelho ardente, está-se dizendo que sua percepção é uma mistura intersensorial, uma fusão de duas ou mais experiências sensoriais.

\noindent
- Franz Boas, Arte Primitiva

\vfill

\begin{center}
\small
O Jogo de Guerra. \textit{Detalhe de As Aventuras de Kibi} \\
Cortesia do Museu de Belas Artes, Boston
\end{center}

\vfill
\raggedright
\noindent 140

\newpage
\section*{Page 143}

\noindent
\textit{Relações de valor}

\vspace{1em}

O homem é, como vimos, primariamente consciente do objeto. Ele mede o mundo circundante em termos de coisas e, assim, gradualmente aprende a se orientar em seu ambiente. Ele também aprende a avaliar as diferenças de brilho que chegam aos seus olhos, referindo-as a objetos. Ao relacionar a magnitude das coisas ao seu redor com seu próprio tamanho, e ao atribuir uma constância psicológica de tamanho e forma a cada objeto familiar, ele confere aos objetos que conhece constância de cor e brilho. As relações constantes de cor e brilho estão servindo como um medidor elementar para ordenar as relações espaciais.

As pinturas infantis, obras de arte de tribos primitivas, pinturas assírias e egípcias, pinturas europeias primitivas e as da Ásia Oriental, negligenciam totalmente a representação da iluminação. Elas usam a gradação de valor apenas para segregar uma forma da outra e, assim, indicar profundidade e distância entre as coisas. Os valores de brilho, como usados por esses primeiros pintores, tinham um papel simbólico claro para o objeto como um todo e não eram sobrecarregados com detalhes de observação minuciosa ou prejudicados pelo sistema geométrico fixo da perspectiva de iluminação. Portanto, cada forma, em seu respectivo valor, podia funcionar com força e estruturalmente.

\vspace{2em}

\centering
\textit{Pintura de uma Criança}\\
\textit{Cortesia da Escola de Arte do Instituto Munson-Williams-Proctor}
\par

\newpage
\section*{Page 144}

MOULIN ROUGE \\
BAILE \\
TODAS AS NOITES \\
A GOULUE

CH LEVY 10 Rue Martel Paris

Toulouse-Lautrec. Moulin Rouge \\
Cortesia do Museu de Arte do Smith College

O \\
TEATRO GLOBE

Lester Beall. Design Publicitário

142

\newpage
\section*{Page 145}

\textit{Perspectiva da iluminação; modelagem por sombreamento}

Em condições ordinárias, os objetos em nosso ambiente não recebem iluminação uniforme de todos os ângulos. Um objeto sólido receberá mais luz de um lado do que de outro, porque esse lado está mais próximo da fonte de luz e, assim, interceptará a luz e projetará sombras nos outros lados. Essa variação de valores tonais criada pela iluminação irregular é o que o olho realmente percebe.

A superfície de uma esfera, um cubo ou qualquer outra forma, apresenta sua própria distribuição característica de luz. Uma superfície esférica reflete a luz em um fluxo contínuo do claro ao escuro. Uma superfície angular reflete a luz com contrastes súbitos dos valores claro e escuro. Cada forma básica tem um padrão fundamental de luz e sombra. Uma gradação tonal que flui uniformemente evoca em nós uma sensação de forma suavemente curvada. Uma mudança brusca e repentina de tom nós traduzimos como significando uma superfície nítida ou angular.

Se formas relativamente opacas interceptam o caminho da luz, corpos de sombra são formados. A natureza da fonte de luz, a distância entre ela e o objeto, e o ângulo dos raios de luz incidentes definem o caráter espacial da sombra. Assim, o comprimento, a forma, o valor de brilho da sombra projetada nos fornecem informações adicionais sobre as formas dos sólidos e também indicam a extensão e a forma dos intervalos espaciais entre os sólidos.

Desde a descoberta da perspectiva, os pintores começaram a representar a imagem óptica da luz moldada e curvada pelos vários meios do ambiente. Eles desenvolveram uma habilidade progressiva primeiro na delineação da aparência escultural tridimensional do mundo-objeto, depois no domínio da luz e sombra como forças articuladoras do espaço, e finalmente na representação do espaço como luminoso, dissolvendo a solidez em substância de luz.

Em sua busca pela fidelidade óptica da representação, os pintores foram forçados a condensar cada vez mais o tempo de observação. Eles o reduziram quase ao infinito. A experiência visual, no entanto, é uma experiência espaço-temporal. Uma distância -- uma extensão no espaço -- só tem significado se um certo tempo for necessário para percorrê-la. A própria existência da matéria é inseparável do tempo. É impossível conceber um objeto material como existindo instantaneamente. Quanto mais precisa se tornava a representação da interação da luz sobre o objeto, mais a representação se afastava de uma verdadeira expressão visual da extensão espacial. Ela nunca poderia alcançar uma fusão íntima da experiência espaço-temporal.

A representação passou por um desenvolvimento semelhante ao da perspectiva linear. Os pintores começaram a se insurgir contra as amarras impostas à representação do espaço pela unidade de iluminação fixa. Ao quebrar essas amarras, eles alcançaram a emancipação progressiva das cores e valores como forças plásticas.

\vspace{\fill}
\hfill 143

\newpage
\section*{Page 146}

\raggedright
Rafael Santi. Madona Alba. Cortesia da Galeria Nacional, Washington, D. C. Coleção Mellon.

144

\newpage
\section*{Page 147}

Rembrandt. \textit{Retrato do Artista}\par
Cortesia do The Metropolitan Museum of Art\par
\vspace{1.5em}

\noindent\textit{Modificações da perspectiva de iluminação}\par
\vspace{1em}

A experiência diária com iluminação natural e artificial nos leva, via de regra, a esperar que a direção da luz venha de cima. Cada desvio desta condição de luz padrão é registrado e interpretado por nós como uma exageração das dimensões espaciais. A iluminação vinda de baixo, de trás, ou de um ou outro lado inesperado cria um efeito espacial dinâmico. Os pintores exploraram esta amplificação da perspectiva de iluminação. De forma semelhante, assim como a perspectiva linear foi esticada ou condensada aos seus limites máximos até atingir o maior poder dinâmico possível dentro de suas limitações, a perspectiva de iluminação foi curvada, moldada, esticada ao máximo. Um equivalente ao encurtamento amplificado foi assim empregado nos termos de luz e sombra. Em vez de uma modelagem suave das formas por gradação suavemente curvada, escalas de valores condensadas e esticadas foram introduzidas. Outro passo que os pintores deram corresponde ao recurso da perspectiva simultânea. Correspondendo ao uso simultâneo de vários pontos de fuga e vários horizontes, eles empregaram em uma única imagem múltiplas perspectivas de iluminação contraditórias. Eles modificaram e adaptaram a distribuição de valores às demandas do plano da imagem.\par
\vspace{2em}
\null\hfill 145\par

\newpage
\section*{Page 148}

\raggedright

\centering\textit{Reavaliação da perspectiva da iluminação}\par

Os efeitos de luz e sombra em uma imagem representacional implicam uma abstração. Eles provêm de um ponto de vista fixo e indicam a fixação da posição do espectador, da fonte de luz e da posição do objeto. Mas as relações de luz e sombra são, na realidade, transitórias, acidentais e ilusórias. A representação de um objeto sob tal iluminação fixa significa sua prisão no tempo, e é, consequentemente, um aspecto muito limitado de eventos espaciais. Os pintores cubistas tornaram-se conscientes dessa contradição. Eles reconheceram que o desaparecimento total da iluminação, ou a iluminação perfeitamente uniforme das superfícies de um objeto, tornaria esse objeto inarticulado; faria, de fato, com que ele desaparecesse. Mas eles também reconheceram que a relação de brilho não é exatamente a mesma que o efeito de iluminação. O controle arbitrário de luz e sombra pode explicar o objeto sem prendê-lo no tempo. O pintor, portanto, concebeu um método gráfico de fusão do primeiro plano e do fundo por meio de uma extensão arbitrária de luz e sombra. Por valores sutilmente graduados e conscientemente controlados, os planos são feitos para inclinar-se para frente ou para trás sem, em última instância, definir o volume, de modo que as formas pareçam dissolver-se no espaço do fundo. Essas pequenas facetas de sombra são como sinais de direção dinâmicos, guiando o olho para toda e qualquer extensão possível no espaço. É possível, no entanto, decompor o sólido de tal forma que o observador não consiga encontrar nenhuma unidade espacial em termos da modelagem ilusória por sombreamento. A unidade espacial só pode então ser alcançada criando-se uma tensão espacial viva entre os movimentos virtuais dos valores que avançam e recuam. Buscando encontrar ordem, mantêm-se as forças centrífugas dos valores tonais em equilíbrio espacial como se suspensas por forças invisíveis. Cada valor de brilho possui um significado estrutural claro na organização desse espaço.

Exatamente o mesmo aconteceu quando esses mesmos pintores romperam com a perspectiva linear. Libertados das amarras da perspectiva absoluta e da modelagem por sombreamento, os valores de brilho revelaram um poder intrínseco de criar experiências espaciais na superfície da imagem sem sugerir o mundo tridimensional dos objetos. Gradações de valor, em definições nítidas ou borradas, foram reconhecidas como forças plásticas genuínas, e a superfície da imagem alcançou uma clareza estrutural e uma nova intensidade sensória.

\vspace*{\fill}
\centering
146

\newpage
\section*{Page 149}

\noindent\textit{Picasso. Pierrot. Cortesia do Museu Guggenheim de Arte Não Objetiva}

\newpage
\section*{Page 150}

\subsection*{Influência da fotografia}
Enquanto os pintores trabalhavam para a quebra do hábito de representação por modelagem através do sombreamento, a fotografia alcançou uma perfeição até então insondável na renderização de formas visíveis por luz e sombra. A representação fotográfica trouxe para o foco coisas e eventos em suas aparências reais, revelando muito do que antes passava despercebido ou embaçado em nossa observação. Pela primeira vez, os homens foram capazes de congelar os processos em movimento da natureza em padrões de luz e sombra. O que o olho nunca foi capaz de fazer, o sistema óptico da câmera e a emulsão fotossensível puderam fazer. Pôde registrar com objetividade e precisão a variedade infinita de diferenças de brilho refletidas das superfícies.

Este avanço na gravação fotográfica tornou certas reavaliações necessárias nos hábitos visuais. Pela sua própria perfeição mecânica, tornou obsoleto o objetivo herdado do pintor: a representação das aparências ilusórias de coisas familiares. No entanto, quanto mais precisa a gravação fotográfica, mais óbvia se tornava a limitação inerente de uma perspectiva absoluta. Uma forma pode interceptar a luz e projetar uma sombra sobre outra de tal forma que o caráter espacial do objeto na sombra se tornará ininteligível. A fotografia, dentro de sua própria esfera, lutava para encontrar uma solução para este problema, desvinculando as fontes de luz e organizando arbitrariamente a distribuição de luz e sombra. Os melhores fotógrafos conseguiram alcançar um tratamento plástico flexível de luz e sombras.

\subsection*{Nitidez e falta de definição}
Objetos vistos de longe tornam-se gradualmente embaçados e indefinidos. Os artistas do Renascimento haviam observado isso e o introduziram como um dispositivo de representação. Através de seu trabalho, isso se tornou uma declaração estereotipada. A fotografia, no entanto, invalidou em muitos aspectos esses padrões aceitos de perspectiva aérea, assim como havia invalidado os da perspectiva linear.

O olho é um instrumento óptico construído de tal forma que só pode focar em um plano. Não somos capazes de ver objetos próximos e distantes nitidamente ao mesmo tempo. Nunca percebemos isso completamente até que outro instrumento óptico, a câmera, o trouxesse forçosamente à nossa atenção, congelando a relação de imagens borradas e nítidas na superfície da fotografia. Então, pudemos ver e estudar uma imagem em todas as suas sutilezas de modulação tonal. Tornamo-nos sensíveis ao significado espacial da nitidez e da falta de definição.

A representação do espaço foi ampliada por este novo idioma plástico. Os pintores descobriram que os princípios assim descobertos poderiam ser usados na criação visual, independentemente de haver ou não representação real de objetos, pois seu poder reside na capacidade de proporcionar uma experiência espacial legítima, em vez de qualquer função de auxiliar o olho no reconhecimento de objetos.

148

\newpage
\section*{Page 151}

\textbf{Claude Lorrain. Paisagem}\par
Cortesia do Museu de Arte de Cleveland\par
\vspace{5cm}

\begin{center}
\textbf{L. Moholy Nagy. Padrão de Marselha 1929}
\end{center}
\vspace{6cm}

\raggedleft 149

\newpage
\section*{Page 152}

\par\noindent
\textit{Contraste de Textura \textbullet}

\vspace{2em} 

\par\noindent
\large\textbf{Textura}

\par
O progresso tecnológico contribuiu grandemente para a introdução de outro idioma visual: a textura. Um conhecimento mais amplo e um uso mais extensivo de materiais e estruturas, a descoberta de materiais sintéticos e a cultura da máquina com sua nova riqueza de superfícies, tornaram imperativa a familiaridade com a nova paisagem. O olho desassistido não conseguiria acompanhar, nenhuma habilidade manual teria a precisão de coordenação para representar, todas as intrincadas qualidades de superfície dos novos materiais e objetos criados pelo homem --- por exemplo, um disco fonográfico com sua inumerável variedade de trilhas sonoras, ou um objeto de metal polido feito à máquina com sua perfeita qualidade de superfície. Somente a câmera poderia lidar adequadamente com a domesticação visual da nova riqueza do mundo-objeto. Somente a câmera poderia acompanhar as propriedades visuais que se desdobram rapidamente das formas e estruturas recém-criadas.

\par
A fotografia, no entanto, fez mais do que aguçar as sensibilidades à riqueza textural do ambiente. Dois pontos de um ou dois objetos, se próximos o suficiente, se fundirão em um quando estiverem além do limiar visual de discriminação. A fotografia deu um novo e mais amplo significado a esse fenômeno. Explorações com macro-, micro- e aerofotografia abriram campos visuais até então além do alcance humano. Na observação visual ordinária, a escala das coisas é clara em referência ao espectador. A representação manual tradicionalmente também se baseava em uma escala relacionada ao espectador. A imagem fotográfica, no entanto, é recortada do quadro espacial familiar de referência e frequentemente não há indício para decifrar a escala espacial. Uma microfotografia e uma aerofotografia podem ser facilmente confundidas. O espaço é condensado ou expandido de acordo com os acessórios ópticos usados em sua gravação.

\vspace{3em} 
\noindent\rule{\textwidth}{0.4pt} 

\vspace{0.5em} 

\par\noindent
\small{\textbullet Trabalho realizado para o curso do autor em Fundamentos Visuais.}

\par\noindent
\small{150}

\newpage
\section*{Page 153}

Devido à relatividade da escala espacial, as variadas qualidades de textura e valores tornaram-se os únicos sinais visíveis capazes de indicar relações espaciais. A forma modelada por sombreamento não podia mais ser considerada o único agente de ordenação espacial das diferenças de valor de brilho. A forma visual tornou-se apenas um caso limite em um novo e mais extenso contexto visual --- superfície da textura.

Walter Peterhans. \textit{Ofélia---Homenagem a Rimbaud}

\newpage
\section*{Page 154}

\begin{flushleft}
Picasso. \textit{Natureza Morta Vire La} \\
... 1914-15 \\
\textit{Cortesia do Sr. Sidney Janis}
\end{flushleft}

Seguindo os fotógrafos pioneiros, os pintores começaram a assimilar na imagem-quadro qualidades de textura inerentes a cada material. Esta nova propriedade sensorial enriqueceu a imagem. Pois a textura possui uma dimensão única. O ritmo particular de luz e sombra que compõe a textura visível está além da nossa capacidade de distinguir em qualquer forma de organização visual em termos de modelagem por sombreamento. Possui um grão fino de impacto sensorial que pode ser compreendido apenas em sua correspondência estrutural com outras sensações sensoriais. A textura superficial da grama, concreto, metal, juta, seda, jornal ou pele, fortemente sugestiva das qualidades do tato, experimentamos visualmente em uma espécie de mistura intersensorial. Vemos, não luz e sombra, mas qualidades de maciez, frieza, rugosidade, tranquilidade---visão e tato estão fundidos em um único todo.

\par
152

\newpage
\section*{Page 155}

\raggedright

Henry Kann \\
\textit{Exercício com Planos Transparentes} \\
\textit{Escola de Design em Chicago} \\
\textit{R. B. Tague, Instrutor}

\vspace{2em} % Spacing to match visual separation

Paul Rand. Design Publicitário

\vspace{3em} % Significant vertical space before the column content, as it's a new visual grouping

\textbf{lógica-6} \\
\textbf{PLATÃO} \\
\small
Este livro é um exame crítico de alguma teoria do pensamento e sua operação para um sistema geral de lógica baseado na proposição e no silogismo, e sua relação com o problema do conhecimento.

Ele tenta apresentar um relato dos métodos usados pelos cientistas em seus estudos e a maneira como vários conceitos estão relacionados entre si, para possibilitar uma compreensão mais profunda do processo de aquisição do conhecimento científico.

Ele trata da lógica em relação à retórica, com particular ênfase no uso de métodos lógicos para a análise de argumentos e para sua construção, de acordo com os fatos da prática científica.
\normalsize

\vspace{1.5em} % Consistent spacing between course blocks

\textbf{economia-2} \\
\textbf{ADAM SMITH} \\
\small
Uma pesquisa das principais teorias e doutrinas que fundamentam a estrutura e operação dos sistemas econômicos contemporâneos, incluindo um exame de alguns dos problemas significativos que nossa sociedade enfrenta atualmente.

Inclui uma consideração do desenvolvimento histórico do pensamento econômico, com o objetivo de mostrar como as teorias econômicas atuais evoluíram das anteriores e como elas refletem as condições sociais e políticas em mudança.

Um estudo dos princípios de produção e distribuição de riqueza, com particular ênfase no papel do governo na economia.

Também inclui uma análise da teoria do valor, com o objetivo de mostrar como ela pode ser aplicada aos problemas de precificação e alocação de recursos.
\normalsize

\vspace{1.5em}

\textbf{literatura-5} \\
\textbf{HOMERO} \\
\small
Este é um estudo dos princípios da crítica literária e sua aplicação à análise de várias formas literárias.

Ele tenta mostrar como as obras literárias podem ser compreendidas e apreciadas examinando seu contexto histórico, suas características estilísticas e suas implicações filosóficas.

Também inclui uma discussão sobre a relação entre literatura e outras formas de arte, e do papel da literatura na sociedade.

Ele trata da teoria da linguagem como uma forma de comunicação humana, com particular ênfase nos aspectos semânticos e pragmáticos da linguagem.
\normalsize

\vspace{1.5em}

\textbf{matemática-4} \\
\small
Este curso é uma introdução aos conceitos e métodos fundamentais da matemática, com o objetivo de mostrar como eles podem ser aplicados à solução de problemas em várias áreas do conhecimento.

Inclui uma consideração do desenvolvimento histórico da matemática, com o objetivo de mostrar como as teorias matemáticas atuais evoluíram das anteriores e como elas refletem as condições sociais e intelectuais em mudança.

Ele trata da teoria dos números, com particular ênfase nas propriedades dos números inteiros e sua relação com a teoria das equações.
\normalsize

\vspace{1.5em}

\textbf{psicologia-3} \\
\small
Uma introdução ao estudo da mente, sua estrutura e funções, métodos de pesquisa psicológica e a aplicação de princípios psicológicos aos problemas do comportamento humano.

Inclui uma consideração do desenvolvimento histórico da psicologia, com o objetivo de mostrar como as teorias psicológicas atuais evoluíram das anteriores e como elas refletem as condições sociais e científicas em mudança.

Ele trata da teoria da personalidade, com particular ênfase nas várias abordagens ao estudo da personalidade e sua relação com os problemas de saúde mental.

Também inclui uma discussão sobre a relação entre psicologia e outras formas de conhecimento, e do papel da psicologia na sociedade.
\normalsize

\vfill % Pushes content to the top
\centering
153

\newpage
\section*{Page 156}

\begin{center}\textit{\large Influência de fontes de luz artificiais}\end{center}

O homem contemporâneo vive em um ambiente urbano que oferece, através de cada fonte de luz artificial, uma cena óptica noturna incomparável a qualquer experiência visual anterior. Edifícios que foram modelados sob o sol em uma forma escultural clara, sob as fontes de luz artificiais que agem simultaneamente, perdem sua qualidade tridimensional. Os contornos ficam obscurecidos. Os pontos de luz, vindos de dentro e de fora simultaneamente, e a fusão de luminosidade e claro-escuro, quebram a forma sólida como unidade de medida do espaço. Os padrões de luz flutuantes e vibrantes não podem ser encaixados em uma forma oticamente modelada. Uma interpretação espacial só pode ser alcançada assumindo uma unidade espacial mais dinâmica do que as formas ilusórias esculpidas pela luz e sombra. Diferenças de brilho, as definições nítidas ou embaçadas, a textura da luz abrangem o espaço por seus valores intrínsecos de avanço ou retrocesso. Aqui também houve uma forte influência ambiental forçando o pintor a reconsiderar e descartar seu antigo hábito de modelar por sombreamento.

A iluminação artificial não apenas introduziu uma nova abordagem para a representação espacial, mas contribuiu para uma ampliação e uma redireção das experiências visuais e, consequentemente, para um reajuste radical das sensibilidades visuais do homem.

``Com nossos métodos atuais de produzir brilhos tremendos em áreas muito pequenas, e controlando-os quase completamente tanto em intensidade quanto em posição espacial, o triunfo do homem da noite para o dia tornou-se praticamente completo. Ao mesmo tempo, todo um novo campo de investigação se separou da óptica tradicional, a saber, o da engenharia de iluminação, um estudo que está em estreita relação com a óptica fisiológica e psicológica. Os principais engenheiros de iluminação estão agora dispostos a reconhecer que sua ciência não pode mais ser considerada simplesmente um ramo da física aplicada, como era o caso nos dias em que se acreditava que a determinação de valores fotométricos esgotava os problemas do campo. Percebe-se agora que um estudo dos efeitos da luz no organismo humano é igualmente importante, tão importante, de fato, que constitui um ramo separado da engenharia de iluminação.''

\vspace{1em}
\noindent \textbullet \ David Katz, \textit{The Word of Color}, Londres, 1935

\vspace*{\fill}
\centering 154

\newpage
\section*{Page 157}

\vspace*{9cm} % Ajuste o espaçamento vertical para posicionar a legenda abaixo de onde a imagem principal estaria.

\raggedright
Bernice Abbott. Vista Noturna

\vspace*{3cm} % Ajuste o espaçamento vertical para posicionar as legendas abaixo de onde a imagem secundária estaria.

\raggedright
R. J. Wolff. Pintura 1937\\
Cortesia de T. B. Foley

\vfill % Empurra o número da página para o fundo da página.

\raggedleft
155

\newpage
\section*{Page 158}

Gyorgy Kepes. Experimento Com Luz 1940

\newpage
\section*{Page 159}

\textbf{R. J. Wolff. Construção em Luz 1938}

\textit{"A principal motivação por trás deste trabalho reside no novo ambiente interno que foi criado pela arquitetura contemporânea. As chamadas artes livres não podem mais ignorar os fatores fisiológicos que tornaram o quadro emoldurado e o objeto sobre pedestal tão arquitetonicamente perturbadores quanto a parede revestida e o lustre."}

\textit{"A Construção em Luz é uma tentativa de criar imagens a partir da luz e do espaço reais. A forma atinge a identidade sem materialidade. As técnicas de pintura, escultura e luz são combinadas para servir a fins arquitetônicos, bem como a um meio de livre expressão." Robert J. Wolff.}

\textbf{L. Moholy Nagy. Fotograma 1923}

157

\newpage
\section*{Page 160}

\small
Man Ray. Raiografia 1922\\
Cortesia do Museu de Arte Moderna
\par

\vfill
\centering 158

\newpage
\section*{Page 161}

L. Moholy Nagy. Fotograma 1939

\vspace*{\fill}
159

\newpage
\section*{Page 162}

L. Moholy Nagy. Fotograma 1930

\newpage
\section*{Page 163}

N. Lerner. \textit{Estudo do Volume da Luz 1939} \\
\textit{Workshop da Luz, Escola de Design em Chicago}

Fotógrafos, pintores e outros experimentadores com a luz são importantes pioneiros na testagem dos efeitos psicofisiológicos da organização plástica da luz. Helmholtz lembrou aos cientistas há muito tempo que "Um estudo cuidadoso das pinturas dos grandes mestres... é de grande importância para a ótica fisiológica." O estudo das expressões dos mestres contemporâneos não é menos significativo. Não só poderiam auxiliar a pesquisa em ótica psicológica, mas também nos ajudar a nos reeducar para uma melhor compreensão da conformação do nosso ambiente físico.

"Estímulos excitantes de brilho têm um efeito, não só da mesma maneira em todos os órgãos dos sentidos, não só dos órgãos dos sentidos sobre os músculos, e vice-versa dos músculos sobre o olho, mas é induzida por tais estímulos uma modificação de todo o organismo... Minhas observações clínicas também provam que o estudo das reações do organismo a estímulos excitantes de brilho e escuridão visa esclarecer problemas, que têm um alcance muito além da psicologia da percepção e que aqui temos que lidar com processos biológicos fundamentais de significado multifacetado para a prática clínica também."\textbullet

\textbullet Walter Barnstein, \textit{Jornal de Psicologia Geral}, 1936

\vspace*{\fill}
\hfill 161

\newpage
\section*{Page 164}

\hspace*{0.3\textwidth}\textit{Lester Beall. Design Publicitário, Fotograma}

\vspace{8em} 

\noindent\textit{Harold Walter. Cartaz com Fotograma}

\vspace{8em} 

\raggedleft\textit{Gyorgy Kepes. Design Publicitário, Fotograma 1937}\par

\vspace{3em} 
\noindent 52

\newpage
\section*{Page 165}

Cahiers d'Art

12\textsuperscript{o} ANO

163

\newpage
\section*{Page 166}

\subsection*{Representação da relação das cores}

As formas primitivas de representação invariavelmente mostravam a cor local do objeto, apesar das mudanças de iluminação, assim como mantinham o valor de brilho constante, o tamanho e a forma reais do objeto, independentemente da distorção de perspectiva. Na pintura do homem paleolítico, artistas assírios e egípcios, pintores de vasos gregos, primeiros pintores europeus, crianças e primitivos, a cor tinha um papel simbólico claro para os objetos como um todo e não era sobrecarregada com detalhes de observações minuciosas. As áreas de cor não eram prejudicadas em suas qualidades sensoriais elementares.

Quanto mais precisa a observação da aparência do objeto e mais aguçada a capacidade do homem de distinguir efeitos ópticos, menos as áreas de cor na superfície da imagem podiam funcionar em plena intensidade sensorial. Assim como o tamanho e a forma do objeto foram alterados pela perspectiva linear, a cor local perdeu sua referência absoluta ao objeto. Desde o Renascimento, as tendências representacionais têm se concentrado na representação exata da modificação das cores locais pelos efeitos da iluminação. Luz e sombra, o reflexo de uma cor em outra, a cor da fonte de luz e outras modificações ópticas foram cuidadosamente registradas à medida que os pintores se esforçavam para uma representação precisa das aparências ópticas dos objetos a partir de um ponto de vista fixo. A entrega incondicional às aparências da coisa foi uma consequência inevitável, e através dessa submissão a um naturalismo superficial, a qualidade sensorial das cores foi gradualmente esvaecida.

É uma experiência familiar que a qualidade sensorial inata em um signo, em uma palavra, em um evento, com o tempo, é absorvida pela coisa para a qual ela representa. Somente repetindo uma palavra familiar repetidamente, por exemplo, pode-se resgatar nela a qualidade sensorial de seu som, torná-la independente de contexto e restaurar sua intensidade sensorial original. Deve-se olhar para uma paisagem familiar de uma posição que oferece uma imagem retiniana incomum das relações familiares de seus objetos componentes, se se quiser sentir a intensidade original das cores dessa paisagem. Somente então as cores, antes incorporadas em seus objetos, se libertarão para falar em sua própria língua pura, a linguagem dos sentidos. A busca consistente para representar todos os aspectos aparentes do ambiente visível levou, paradoxalmente, à libertação da qualidade sensorial da cor. No controle do jogo de luz sobre os objetos, os pintores incluíram o espaço entre os objetos no mundo-objeto. Eles ampliaram o alcance representacional dos objetos, incluindo a atmosfera, o ar, como uma substância moduladora de luz. Eles estavam tentando representar com pigmento a luz que preenche o espaço.

\vspace*{\fill}
\centering 164

\newpage
\section*{Page 167}

T. M. W. Turner. \emph{Incêndio das Casas do Parlamento, 1834}
Cortesia do Museu de Arte de Cleveland

\vspace{5em} % Approximate vertical space to match the image layout

Georges Seurat.
Detalhe de
\emph{Domingo na Grande Jatte 1884-86}
Cortesia do Art Institute de Chicago

\par
Essa tentativa de encontrar uma articulação adequada da luz refletida em pigmentos que pudessem igualar a riqueza sensorial vibrante e viva da luz atmosférica transmitida — o novo tema — levou, em última instância, a uma nova base de representação. Os pintores descobriram cientificamente as leis da mistura de cores. Eles perceberam que não podiam representar a luminosidade da luz transmitida da atmosfera misturando cores da maneira comum. Nas pesquisas científicas de Chevreul e Helmholtz, eles encontraram sua resposta: a cor poderia ser misturada na retina. O dispositivo que inventaram para criar essa mistura óptica de luz colorida na retina foi a fragmentação da superfície de cor, anteriormente modelada suavemente, em pequenos pontos ou linhas de cor. Estes foram combinados de uma forma que os fundia na retina em uma qualidade luminosa. Essa inovação abriu o caminho em duas direções. Uma foi a redescoberta do plano de cor como o elemento construtivo básico da imagem plástica — em sua forma embrionária, uma mancha de cor das pinturas impressionistas. A outra foi a aplicação consciente da mistura aditiva de cores, que deixou claro que a imagem é feita pelo homem, que o fator humano é um elemento integral na imagem, e que a representação de experiências espaciais não pode ser o fac-símile da realidade espacial, mas deve ser uma estrutura correspondente baseada no aparelho receptor humano.

\par
O resultado real na pintura, entretanto, foi na maioria dos casos meramente um registro abreviado do efeito de cor vibrante. A experiência visual foi formulada apenas em termos do olho como um aparelho fisiológico, e a imagem era simplesmente uma réplica dos pontos de cor na retina ampliados na escala do plano da imagem. Aqui a organização plástica atingiu o ponto zero.

\par
\raggedleft
165

\newpage
\section*{Page 168}

\noindent Henri Matisse. \textit{Harmonia em Amarelo} \\
\noindent \textit{Reprodução Cortesia} \\
\noindent \textit{The Art Institute of Chicago}

\par
\noindent Picasso. \textit{Pierrot}

\par
\noindent \textit{A dimensão espacial da cor}

\par
Mas aqui também novas direções de organização plástica
da mancha de cor agora viviam sua própria e verdadeira existência. Poderia ser testada em várias
funções na superfície da imagem. Dissociadas da iluminação fixa, as
superfícies de cor manifestaram primeiramente seu valor espacial intrínseco. Esta
dimensão espacial inerente da cor foi controlada e explorada para construir
sólidos na superfície. Após longas aberrações quando a cor era usada
apenas como coloração, e indicava a cor de um objeto, a cor agora consti-
tuía o objeto. O peso e o volume da massa foram modelados por
e através do avanço e recuo das cores. As formas poderiam agora crescer a partir
das cores, e o uso estrutural da cor poderia dar um sentido de realidade
espacial que nenhum método representacional anterior havia alcançado. Para fazer
as cores funcionarem neste sentido estrutural, é necessário compreender todas
as suas ações espaciais. O interesse dos pintores desde o impressionismo
tem se concentrado no teste das cores em suas ações de avanço, recuo,
expansão e contração.

Nova significância agora atribuída ao fato de que as cores são percebidas na
superfície da imagem em uma interação dinâmica umas com as outras, de modo que cada
qualidade de cor tem apenas um valor relativo, porque sofre modificações
devido à sua inter-relação com outras superfícies de cor. Esta pesquisa neces-
sariamente incluiu aprender como uma cor modifica a outra ao ser colocada
ao lado dela. Buscou também o conhecimento de características ativas e passivas ---
quando e como uma cor prende o olho com um poder maior do que outra.
Quanto mais conhecimento ele adquiria, mais audacioso o pintor se tornava ao des-
associar as superfícies de cor da pertença ao objeto. Os pontos de cor dos
pintores impressionistas, através das facetas de cor de Cézanne; o padrão
decorativo de cor de Gauguin e Matisse, alcançaram pleno desenvolvimento na obra
do cubista e atingiram uma pureza e poder finais na obra de Malevich
e Mondrian.

\par
\raggedleft 166

\newpage
\section*{Page 169}

\large\textbf{As dimensões de sensação da cor}

\vspace{1em} % Adiciona um pouco de espaço vertical após o título

A emancipação do ponto de cor da subordinação servil ao objeto e a iluminação fixa trouxeram outro resultado importante. Deu aos pintores o impulso inicial para ir além das fronteiras de outra tradição de expressão da cor.

Foi apontado anteriormente que, através de uma intrincada interação da estrutura do conjunto de recepção sensorial e de fatores de memória, as sensações de cor são dotadas de qualidades de sensação únicas. Somos capazes de ver, ou parece que somos capazes de ver, o que o olho estruturalmente não é capaz de ver. Vemos cores quentes e frias, quietas e ruidosas, nítidas e opacas, leves e pesadas, tristes e alegres, estáticas e dinâmicas, selvagens e mansas. Por outro lado, temos sensações acústicas e olfativas que possuem cores, brilho, saturação e qualidades espaciais de altura, largura, comprimento, peso, direção e movimento.

Existe uma base estrutural comum para todos os tipos de sensações. Temos a capacidade de perceber qualidades estruturais comuns na visão, audição, tato e paladar. A visão e a audição, em particular, mostram um reservatório inesgotável de estruturas de sensações intercambiáveis. As sensações podem evocar uma resposta emocional intensa, sem que haja ascensão à consciência. Pintores, músicos, poetas e cientistas, cientes da importância e das potencialidades criativas inerentes a esta correspondência estrutural, procuraram e trabalharam para um controle criativo para a sincronização dos sentidos. Goethe fez importantes contribuições. A. W. Schlegel, o romântico alemão do início do século XIX, inventou uma escala de cores correspondente às vogais humanas, e atribuiu um significado especial a cada conjunção particular da cor da vogal. ``A'' representa o vermelho claro luminoso, e significa juventude, amizade, radiância. ``I'' significa azul celeste, simbolizando o amor. ``O'' é roxo, ``V'' significa violeta, etc. Pesquisas científicas recentes ofereceram novos dados importantes. Von Hornbostal realizou um estudo extensivo do fator comum dos diferentes dados sensoriais. Ele afirma sua descoberta em um dos experimentos da seguinte forma: ``Para um cheiro particular, digamos benzol, o cinza brilhante correspondente é escolhido no disco de cores, e para o mesmo cheiro, da série de diapasões, o tom brilhante correspondente.'' Franz Boas traz observações de outro campo. Ele escreve:

\begin{quote}
``A maioria de nós sentirá que um tom alto e um comprimento exagerado, talvez também a vogal i (inglês ee) indicam pequenez, enquanto um tom baixo e um comprimento e as vogais a, o, u (inglês oo) indicam tamanho grande.\ldots Tamanho grande ou pequeno, ou intensidade, pode ser expresso por variações de som. Assim, Nez Perce, uma língua indígena falada em Idaho, muda n para l para indicar pequenez; Dakota tem muitas palavras em que s muda para sh, ou z para j, indicando maior intensidade.\ldots Sem dúvida, o tipo particular de sinestesia entre som, visão e tato desempenhou seu papel no crescimento da linguagem.''
\end{quote}

\small\textbullet\ \textit{Franz Boas, General Anthropology, 1938}

\newpage
\section*{Page 170}

Na representação tradicional, essas qualidades intersensoriais das cores eram fixadas aos objetos, da mesma forma que os valores espaciais intrínsecos de brilho e matiz eram embutidos nos objetos e, assim, prejudicados pela perspectiva linear e modelagem por sombreamento. Pintores contemporâneos romperam esses velhos laços. As cores tiradas do contexto do objeto podem evocar respostas emocionais que vêm de um nível mais profundo do que do estado consciente. Pintores que empregavam essas qualidades intersensoriais da cor foram capazes de despertar reações emocionais de grande intensidade e variedade. Essa abordagem teve uma grande influência rejuvenescedora. Os pintores expressionistas em busca de uma estrutura de cor que pudesse induzir fortes respostas emocionais avançaram audaciosamente e moldaram cores e formas com uma flexibilidade até então inédita.

É interessante notar que essa abordagem foi, em seus primeiros passos, o equivalente psicológico da visão em perspectiva que se aproxima do mundo visível de um ponto de vista fixo. O ponto de vista fixo, nesta última fase do individualismo, era psicológico. Via-se o que se era capaz de ver a partir de sua posição psicológica. Assim como a perspectiva linear distorcia, moldava e modificava as características espaciais reais de acordo com um ângulo de visão particular, o equivalente psicológico da perspectiva linear distorceu, modelou e modificou a aparência da cor e as características de cor dos objetos de acordo com a atitude, emoção, desejos e anseios atuais do indivíduo; o ângulo de visão psicológica. O desenvolvimento consistente neste tratamento subjetivo da aparência da cor das coisas levou a uma negação da cor como realidade objetiva fixa, assim como a busca consistente pela representação da aparência em perspectiva levou à negação da ideia de perspectiva, a realidade óptica fixa. Mas após uma avaliação totalmente subjetiva dos primeiros pintores expressionistas, pintando um rosto azul, vermelho ou amarelo, e um animal verde, preto ou qualquer outra cor que não fosse sua cor natural, mas o que se pode chamar de um ângulo de visão psicológico, o conteúdo do objeto desaparece por completo. A cor permanece como um teclado universal de sentimentos. A representação da cor atinge um nível mais alto de objetividade. E, mais uma vez, a desintegração abriu caminho para uma integração mais completa. A cor foi recuperada como um material básico da criação plástica, um elemento espacial que pode ser organizado estruturalmente, bem como usado como uma qualidade sensorial elementar com um efeito emocional no observador.

\newpage
\section*{Page 171}

\centering
W. Kandinsky. Quadro com Três Manchas. Cortesia do Museu Guggenheim de Arte Não-Objetiva

\raggedleft
169

\newpage
\section*{Page 172}

\section*{Representação do movimento}
A matéria, base física de toda experiência espacial e, portanto, o material-fonte da representação, é cinética em sua própria essência. Desde ocorrências atômicas até ações cósmicas, todos os elementos na natureza estão em perpétua interação---em um fluxo completo. Estamos vivendo uma existência móvel. A terra está girando; o sol está se movendo; as árvores estão crescendo; as flores estão abrindo e fechando; as nuvens estão se unindo, dissolvendo-se, vindo e indo; luz e sombra estão caçando-se mutuamente em um jogo infatigável; as formas estão aparecendo e desaparecendo; e o homem, que está experimentando tudo isso, está ele próprio sujeito a toda mudança cinética. A percepção da realidade física não pode escapar à qualidade do movimento. A própria compreensão dos fatos espaciais, o significado de extensão ou distância, envolve a noção de tempo---uma fusão de espaço-tempo que é movimento. ``Ninguém jamais notou um lugar exceto em um tempo ou um tempo exceto em um lugar'', disse Minkowski em seus \textit{Princípios da Relatividade}.

\section*{As fontes da percepção do movimento}
Como em uma selva selvagem, onde se abrem novos caminhos para progredir, o homem constrói estradas de percepção nas quais é capaz de se aproximar do mundo móvel, para descobrir ordem em suas relações. Para construir essas avenidas de apreensão perceptual, ele se apoia em certos fatores naturais. Um deles é a natureza da retina, a superfície sensível na qual o panorama móvel é projetado. O segundo é o sentido de movimento de seu corpo---as sensações cinestésicas de seus músculos oculares, membros, cabeça, que têm uma correspondência direta com os acontecimentos ao seu redor. O terceiro é a associação de memória de experiências passadas, visuais e não visuais; seu conhecimento sobre as leis da natureza física do mundo-objeto circundante.

\section*{A mudança da imagem retiniana}
Percebemos qualquer estimulação sucessiva dos receptores retinianos como movimento, porque tais estimulações progressivas estão em interação dinâmica com estimulações fixas, e, portanto, os dois tipos diferentes de estimulação podem ser percebidos em um todo unificado apenas como um processo dinâmico, o movimento. Se a retina é estimulada com impactos estacionários que se sucedem rapidamente, a mesma sensação de movimento óptico é induzida. Letreiros luminosos com suas lâmpadas elétricas piscando rapidamente são percebidos em continuidade através da persistência da visão e, portanto, produzem a sensação de movimento, embora a posição espacial das lâmpadas seja estacionária. O movimento no cinema é baseado na mesma fonte da percepção visual.

As mudanças de quaisquer dados ópticos indicando relações espaciais, como tamanho, forma, direção, intervalo, brilho, clareza, cor, implicam movimento. Se a imagem retiniana de qualquer um desses sinais sofre mudança contínua e regular, expansão ou contração, progressão ou gradação, uma per-

\newpage
\section*{Page 173}

percebe um movimento de aproximação ou afastamento, de expansão ou contração. Se alguém vê uma distância crescente ou decrescente entre esses sinais, percebe um movimento horizontal ou vertical.
``Suponha, por exemplo, que uma pessoa esteja parada em uma floresta densa, onde é impossível para ela distinguir, exceto vagamente e grosseiramente em uma massa de folhagem e galhos ao seu redor, o que pertence a uma árvore e o que pertence a outra, e quão separadas as árvores estão. No momento em que ele começa a se mover para frente, no entanto, tudo se desvenda e imediatamente ele tem uma apercepção do conteúdo da floresta e das relações dos objetos entre si no espaço.''

De um trem em movimento, quanto mais próximo o objeto, mais rápido ele parece se mover. Um objeto distante move-se lentamente e um muito remoto parece estar estacionário. O mesmo fenômeno, com uma velocidade relativa menor, pode ser observado na caminhada, e com uma velocidade ainda maior em um avião em aterrissagem ou em um elevador em movimento.

\noindent\textbf{O papel da velocidade relativa}

A velocidade do movimento tem um efeito condicionante importante. O movimento pode ser muito rápido ou muito lento para ser percebido como tal por nosso limitado aparelho sensorial. O crescimento de árvores ou do homem, a abertura de flores, a evaporação da água são movimentos além do limiar da percepção visual comum. Não se vê o movimento do ponteiro de um relógio, de um navio em um horizonte distante. Um avião no céu mais alto parece pairar imóvel. Ninguém consegue ver movimentos menos rápidos além do ponto em que se observa a transformação óptica do movimento na ilusão de um sólido. Uma tocha girada rapidamente perde sua extensão física característica, mas se submerge em outro sólido de aparência tridimensional -- no volume virtual de um cone ou uma esfera. Nossa incapacidade de distinguir nitidamente além de um certo intervalo de impactos ópticos torna as impressões visuais um borrão que serve como ponte para uma nova forma óptica. O grau de velocidade de seu movimento determinará a densidade aparente dessa nova forma. A densidade óptica do mundo visível é em grande parte condicionada pela nossa capacidade visual, que possui suas limitações particulares.

\noindent\textbf{A sensação cinestésica}

Quando um objeto em movimento entra no campo visual, persegue-se ele com um movimento correspondente dos olhos, mantendo-o em uma posição estacionária ou quase estacionária na retina. A estimulação retiniana, então, não pode sozinha explicar a sensação de movimento. A experiência do movimento, que é inegavelmente presente em tal caso, é induzida pela sensação de movimentos musculares. Cada fibra muscular individual contém uma terminação nervosa, que registra cada movimento que o músculo faz. Que somos capazes de sentir o espaço no escuro, avaliar distâncias-direções na ausência de corpos contatados, é devido a essa sensação muscular---a sensação cinestésica.

\noindent\quad\textbullet\ Helmholtz, Óptica Fisiológica

\vspace*{\fill}
\raggedleft 171
\par

\newpage
\section*{Page 174}

\raggedright

\noindent E. G. Lukacs. \textit{Ação} \\
\small Da turma de design de Herbert Bayer \par
\vspace{1cm}

\noindent H. L. Carpenter. \textit{Movimento} \textbullet \par
\small \textbullet Trabalho feito para o curso do autor em Fundamentos Visuais. \par
\vspace{1cm}

\noindent Paul Rand. \textit{Design de Capa} \par
\vspace{1.5cm}

\noindent \textit{Fontes de memória} \par
\vspace{0.5cm}

A experiência ensina o homem a distinguir as coisas e a avaliar suas propriedades físicas. Ele sabe que os corpos têm peso; sem suporte, eles cairão por necessidade. Quando, portanto, ele vê um corpo no ar que ele sabe ser pesado, ele associa automaticamente a direção e a velocidade de sua queda. Também se acostuma a ver objetos pequenos como mais móveis do que os grandes. Um homem é mais móvel que uma montanha; um pássaro está mais frequentemente em movimento do que uma árvore, o céu ou outras unidades visíveis em seu plano de fundo. Tudo o que se experimenta é percebido em uma unidade polar na qual um polo é aceito como um plano de fundo estacionário e o outro como uma figura móvel e mutável. \par

\vspace*{\fill}
\noindent 172

\newpage
\section*{Page 175}

\section*{Dispositivos de representação tradicionais}

Ao longo de toda a história, os pintores tentaram sugerir movimento na superfície estática da imagem, para traduzir alguns dos sinais ópticos da experiência de movimento em termos da imagem pictórica. Seus esforços, no entanto, foram tentativas isoladas nas quais uma ou outra fonte de experiência de movimento foi utilizada; o deslocamento da imagem retiniana, a experiência cinestésica ou a memória de experiências passadas foram sugeridos em termos bidimensionais.

Essas tentativas foram condicionadas principalmente pelo hábito de usar as coisas como unidade básica de medida para cada evento na natureza. As características constantes das coisas e objetos, em primeiro lugar o corpo humano, animais, sol, lua, nuvens ou árvores, foram usadas como os primeiros pontos fixos de referência na busca de relações no turbilhão óptico dos acontecimentos.

Portanto, os pintores tentaram primeiro representar o movimento sugerindo as modificações visíveis dos objetos em movimento. Eles conheciam as características visuais dos objetos estacionários e, portanto, toda mudança observável servia para sugerir movimento. O artista pré-histórico conhecia seus animais, sabia, por exemplo, quantas pernas eles tinham. Mas quando via um animal em movimento realmente rápido, não conseguia evitar ver a modificação visual das características espaciais conhecidas. O pintor das cavernas de Altamira que retrata uma rena correndo com várias pernas, ou o cartunista do século XX que retrata um rosto em movimento com muitos perfis sobrepostos, está estabelecendo uma relação entre o que ele sabe e o que vê.

Outros pintores, procurando indicar movimento, utilizaram a distorção expressiva dos corpos em movimento. Michelangelo, Goya, e também Tintoretto, ao alongar e esticar a figura, mostraram distorção do rosto sob a expressão das tensões da ação e mobilizaram inúmeras outras referências psicológicas para sugerir ação.

\noindent Ch. D. Gibson. \\
O Dilema do Cavalheiro c. 1900

O menor movimento é mais possuidor da atenção do que a maior riqueza de objetos relativamente estacionários. Pintores de muitos períodos diferentes observaram isso bem e o exploraram criativamente. A vitalidade óptica das unidades em movimento que enfatizavam por contornos dinâmicos, por uma interação veemente de contraste vigoroso de luz e escuridão, e por contraste extremo de cores. Em várias pinturas de Tintoretto, Maffei, Veronese e Goya, a riqueza óptica e a intensidade das figuras em movimento são justapostas contra o padrão visual submisso, neutro, do fundo estacionário.

173

\newpage
\section*{Page 176}

\begin{flushleft}
\textit{Harunobu. Dia Ventoso Sob Salgueiro}
\end{flushleft}
\begin{flushleft}
\textit{Cortesia do Instituto de Arte de Chicago}
\end{flushleft}

\vspace{5.5cm} % Espaço para a imagem superior

\begin{flushleft}
\textit{Maffei. Pintura}
\end{flushleft}

\vspace{5.5cm} % Espaço para a imagem inferior

A exploração criativa das estimulações sucessivas dos receptores
retinianos em termos da superfície da imagem foi outro artifício que
muitos pintores acharam útil. A continuidade linear prende a atenção e
força o olho a um movimento de perseguição. O olho, seguindo a linha,
age como se estivesse no caminho de algo em movimento e atribui à linha
a qualidade do movimento. Quando os escultores gregos organizaram a
draperia de suas figuras que representavam em movimento, as linhas
foram concebidas como forças ópticas que faziam o olho seguir sua
direção.

Sabemos que um objeto pesado em um fundo que não oferece
resistência substancial cairá. Ao ver tal objeto, o interpretamos como
ação. Fazemos uma espécie de qualificação psicológica. Todo objeto
visto e interpretado em um referencial de gravitação é dotado de ação
potencial e pode aparecer como caindo, rolando, movendo-se. Porque
costumamos assumir uma identidade entre as direções horizontal e
vertical na superfície da imagem e as direções principais do espaço
como as percebemos em nossas experiências cotidianas, qualquer
colocação de uma representação de objeto na superfície da imagem que
contradiga o centro de gravidade, a direção principal do espaço – o eixo
horizontal ou vertical – faz com que esse objeto pareça estar em ação.
A parte superior e inferior da superfície da imagem têm um
significado a esse respeito.

\vspace{\fill}
174

\newpage
\section*{Page 177}

G. McVicker. \textit{Estudo do Movimento Linear}\\
\textit{Trabalho feito para o curso do autor}\\
\textit{em Fundamentos Visuais}\\
\textit{Patrocinado pelo The Art Director's Club}\\
\textit{de Chicago. 1938}

Lee King. \textit{Estudo da Representação do Movimento}\\
\textit{Trabalho feito para o curso do autor}\\
\textit{em Fundamentos Visuais}\\
\textit{Escola de Design em Chicago}

\newpage
\section*{Page 178}

\subsection*{Tentativas contemporâneas de representação do movimento}

Enquanto a representação visual da profundidade havia encontrado vários sistemas completos, como a perspectiva linear, a modelagem por sombreamento, um desenvolvimento paralelo nunca havia ocorrido na representação visual do movimento. Possivelmente, isso se deu porque o ritmo de vida era comparativamente lento; portanto, a ordenação e representação dos eventos podiam ser comprimidas sem sérias repercussões em formulações estáticas. Os eventos eram medidos por coisas, formas estáticas idênticas a si mesmas, em uma fixidez perpétua. Mas esse ponto de vista estático perdeu toda a semelhança de validade quando as experiências diárias bombardearam o homem com uma velocidade de impactos visuais em que a fixidez das coisas, sua autoidentidade, parecia se dissolver. Quanto mais complexa a vida se tornava, mais as relações dinâmicas confrontavam o homem, em geral e em particular, como experiências visuais, mais necessário se tornou reavaliar as antigas concepções sobre a fixidez das coisas e buscar uma nova forma de ver que pudesse interpretar o ambiente do homem em sua mudança. Não foi por acaso que nossa época fez a primeira busca séria por uma reformulação dos eventos na natureza em termos dinâmicos. Essa reformulação de nossas ideias sobre o mundo incluía quase todos os aspectos que se percebem. A interpretação do mundo objetivo em termos de física, a compreensão do organismo vivo, a leitura do movimento interno dos processos sociais e a interpretação visual dos eventos eram, e ainda são, uma luta por uma nova medida suficientemente elástica para se expandir e contrair ao seguir as mudanças dinâmicas dos eventos.

\subsection*{A influência das condições tecnológicas}

O ambiente do homem que vive hoje possui uma complexidade que não pode ser comparada com nenhum ambiente de qualquer época anterior. Os arranha-céus, a rua com sua vibração caleidoscópica de cores, as vitrines com suas múltiplas imagens espelhadas, os bondes e automóveis, produzem uma simultaneidade dinâmica de impressão visual que não pode ser percebida nos termos dos hábitos visuais herdados. Neste turbilhão óptico, os objetos fixos parecem totalmente insuficientes como a fita métrica dos eventos. A luz artificial, o piscar das lâmpadas elétricas e o jogo móvel dos muitos novos tipos de fontes de luz bombardeiam o homem com sensações cinéticas de cor, com um teclado nunca antes experimentado. O homem, o espectador, é ele próprio mais móvel do que nunca. Ele viaja em bondes, automóveis e aviões e seu próprio movimento confere aos impactos ópticos um ritmo muito além do limiar de uma percepção clara de objetos. A máquina que o homem opera adiciona sua própria demanda por uma nova forma de ver. As interações complicadas de suas partes mecânicas não podem ser concebidas de forma estática; elas devem ser percebidas pela compreensão de seus movimentos. O cinema, a televisão e, em grande parte, o rádio, exigem um novo pensamento, ou seja, uma forma de ver que leva em conta qualidades de mudança, interpenetração e simultaneidade.

176

\newpage
\section*{Page 179}

O homem só pode enfrentar com sucesso este intrincado padrão de eventos ópticos se conseguir desenvolver uma velocidade em sua percepção para igualar a velocidade de seu ambiente. Ele só pode agir com confiança se aprender a se orientar na nova paisagem móvel. Ele precisa ser mais rápido do que o evento que pretende dominar. A origem da palavra "velocidade" tem um significado revelador. Em sua forma original na maioria das línguas, a velocidade está intimamente conectada ao sucesso. Espaço e velocidade são, além disso, em algumas formas iniciais de linguagem, intercambiáveis em significado. A orientação, que é a base da sobrevivência, é garantida pela disseminação da compreensão das relações dos eventos com os quais o homem se confronta.

\subsection*{Motivações sociais e psicológicas}

Significativamente, as tentativas contemporâneas de representar o movimento foram feitas nos países onde a vitalidade da vida era mais prejudicada por condições sociais ultrapassadas. Na Itália, os avanços tecnológicos e suas consequências econômico-sociais estavam ligados às relíquias de ideias e instituições passadas. Os defensores da mudança não conseguiam ver uma direção clara e positiva. A mudança, como a concebiam, significava expansão desenfreada, política de poder imperialista. A vanguarda do imperialismo em expansão identificava o passado com os monumentos do passado, e com os guardiões desses monumentos; e tentaram quebrar, com um vandalismo desinibido, tudo o que lhes parecia prender o progresso em direção aos seus objetivos. "Queremos libertar nosso país da gangrena fétida de professores, arqueólogos, guias e antiquários", proclamava o manifesto futurista de 1909. A violência da expansão imperialista era identificada com a vitalidade; com o próprio fluxo da vida. Tudo o que impedia o desejo da besta de alcançar sua presa deveria ser destruído. Movimento, velocidade, celeridade tornaram-se seus ídolos. Implementos mecânicos destrutivos, o trem blindado, a metralhadora, uma bomba explosiva, o avião, o carro, o boxe, eram símbolos adorados da nova virilidade que buscavam. Na Rússia, onde o presente também estava ligado ao passado e o povo lutava por um novo fôlego de ação, o interesse também se concentrava nas qualidades dinâmicas da experiência. A motivação básica de reorientação para uma expressão cinética era bastante semelhante à dos futuristas italianos. Era um desgosto total com um presente mantido cativo pelo passado. Os pintores, escritores da Rússia, assim como as massas russas, ansiavam por escapar para um futuro livre dos laços de instituições e hábitos ultrapassados. Museus, gramática, autoridade, eram concebidos como inimigos; força, massas em movimento, máquinas em movimento eram amigos. Mas esta revolta contra tradições estagnadas, este ridicularizar selvagem de todas as formas ultrapassadas, abriu o caminho para a construção de um mundo mais amplo. A velha linguagem, que como Mayakovsky disse "era fraca demais para acompanhar a vida", foi reorganizada em idiomas cinéticos de propaganda revolucionária. A linguagem visual do passado, de cujos mestres Mayakovsky perguntou com justo desprezo: "Pintores, vocês tentarão capturar cavalaria veloz com a pequena rede de contornos?" foi infundida com novo sangue vivo de visão cinematográfica.

\vspace*{\fill}
\raggedleft
177

\newpage
\section*{Page 180}

\subsection*{Dispositivos representacionais}

Em sua busca por encontrar uma projeção óptica que se conformasse à realidade dinâmica como a sentiam e compreendiam, os pintores repetiam inconscientemente o caminho traçado pela ciência física em avanço. Seu primeiro passo foi representar, no mesmo plano pictórico, uma sequência de posições de um corpo em movimento. Isso era basicamente nada mais que uma catalogação de localizações espaciais estacionárias. A ideia correspondia ao conceito da física clássica, que descreve objetos existentes no espaço tridimensional e mudando de localizações em sequência de tempo absoluto. O conceito do objeto foi mantido. A sequência de eventos congelados no plano pictórico apenas amplificou a contradição entre a realidade dinâmica e a fixidez do conceito de objeto tridimensional.

Seu segundo passo foi fundir as diferentes posições do objeto, preenchendo o caminho de seu movimento. Os objetos não eram mais considerados unidades isoladas e fixas. Energias potencial e cinética foram incluídas como características ópticas. O objeto era considerado em movimento ativo, indicando sua direção por ``linhas de força'', ou em movimento potencial, prenhe de linhas de força, que apontavam a direção para onde o objeto iria se fosse libertado. Os pintores, assim, procuraram retratar o ponto de vista mecânico da natureza, concebendo equivalentes ópticos para massa, força e gravitação. Essa inovação significou um progresso importante, pois as linhas de força indicadas poderiam funcionar como as forças plásticas do plano pictórico bidimensional.

O terceiro passo foi guiado pelo desejo de integrar o labirinto cada vez mais complicado de direções de movimento. A confusa mistura caótica de linhas de força centrífugas precisava ser unificada. A representação simultânea dos numerosos aspectos visíveis que compõem um evento foi a nova técnica representacional aqui introduzida. A análise espacial cubista foi sincronizada com a linha de forças. O corpo do objeto em movimento, o caminho de seu movimento e seu fundo foram retratados na mesma imagem, fundindo todos esses elementos em um padrão cinético. A linguagem romântica dos manifestos futuristas descreve o método assim: ``A simultaneidade da alma em uma obra de arte; tal é o objetivo excitante de nossa arte. Ao pintar uma figura em uma varanda, vista de dentro, não nos limitaremos àquilo que pode ser visto através da moldura da janela; daremos a soma total da sensação visual da rua, a dupla fileira de casas estendendo-se à direita e à esquerda as varandas floridas, etc. \ldots em outras palavras, uma simultaneidade de ambiente e, portanto, um desmembramento e deslocamento de objetos, uma dispersão e confusão de detalhes independentes uns dos outros e sem referência à lógica aceita'', disse Marinetti. Este conceito mostra grande semelhança com a ideia expressa por Einstein, que expôs, como físico, a interpretação espaço-tempo da teoria geral da relatividade. ``O mundo dos eventos pode ser descrito por uma imagem estática projetada no fundo do contínuo espaço-tempo quadridimensional. No passado, a ciência descrevia o movimento como ocorrências no tempo; a teoria geral da relatividade interpreta os eventos existentes no espaço-tempo.''

178

\newpage
\section*{Page 181}

Giacomo Balla. \textit{Cão na Coleira} 1912. Cortesia do Museu de Arte Moderna.

\bigskip

Giacomo Balla. \textit{Automóvel e Ruído}. Cortesia da Arte Deste Século.

\vfill
\raggedleft 179

\newpage
\section*{Page 182}

Marcel Duchamp.\\
Nu Descendo a Escada 1912\\
Reprodução Cortesia\\
O Instituto de Arte de Chicago

\bigskip

Marcel Duchamp. Jovem Triste em um Trem.\\
Cortesia de Art of This Century

\vfill
\begin{center}
180
\end{center}

\newpage
\section*{Page 183}

\raggedright

Gyorgy Kepes. Design Publicitário 1938
Cortesia da Container Corporation of America

\par % Adicionar uma linha em branco para separar os parágrafos.

\textbf{cartões dobráveis}
construção
design
produção
serviço
CONTAINER CORPORATION OF AMERICA

\par % Adicionar uma linha em branco para separar os parágrafos.

\textbf{MERCEARIAS DE 4 ANDARES?}
Aumentos enormes no espaço de loja seriam necessários para lidar com o comércio atual---sem embalagens modernas.
CONTAINER CORPORATION OF AMERICA

\par % Adicionar uma linha em branco para separar os parágrafos.

Herbert Matter. Design Publicitário
Cortesia da Container Corporation of America

\vfill
\null\hfill 181

\newpage
\section*{Page 184}

Harold E. Edgerton. Golfista

Cartaz Soviético
Batemos
Contra os Falsos Trabalhadores de Choque

182

\newpage
\section*{Page 185}

E. McKnight Kauffer. O Pássaro Madrugador 1919

\textit{Cortesia do Museu de Arte Moderna}

183

\newpage
\section*{Page 186}

Delauney. \textit{Ritmo Circular} \textit{Cortesia do Museu Guggenheim de Arte Não Objetiva}

184

\newpage
\section*{Page 187}

A aproximação mais próxima da representação do movimento nos termos genuínos do plano da imagem foi alcançada pela utilização de planos de cor como fator organizador. A origem da cor é a luz, e as cores na superfície da imagem têm uma tendência intrínseca a retornar à sua origem. O movimento, portanto, é inerente às cores. Pintores com a intenção de realizar todas as potencialidades de movimento da cor acreditavam que a imagem se torna uma forma apenas nas inter-relações progressivas de cores opostas. Superfícies de cor adjacentes exibem efeitos de contraste. Elas reforçam uma à outra em matiz, saturação e intensidade.

Quanto maior a intensidade das superfícies de cor alcançada por um uso cuidadosamente organizado de contraste simultâneo e sucessivo, maior o seu movimento espacial de cor em relação ao plano da imagem. O seu movimento de avanço, recuo, contração e circulação na superfície cria uma rica variedade, circular, espiral, pendular, etc., no processo de moldá-las numa única forma que é clara ou, em termos práticos, cinzenta. ``A forma é movimento'', declarou Delaunay. O contorno clássico contínuo dos objetos foi, portanto, eliminado e uma descontinuidade rítmica criada pelo agrupamento de cores no maior contraste possível. O plano da imagem, dividido em várias superfícies de cor contrastantes de diferente matiz, saturação e intensidade, só podia ser percebido como uma forma, como um todo unificado na sequência dinâmica da percepção visual. A animação da imagem que eles alcançaram baseia-se nos passos progressivos para equilibrar as cores opostas.

As forças centrífugas e centrípetas dos planos de cor contrastantes movem-se para frente e para trás, para cima e para baixo, para a esquerda e para a direita, compelindo o espectador a uma participação cinética à medida que ele segue a direção espacial intrínseca das cores. A qualidade dinâmica baseia-se no movimento genuíno das forças plásticas em sua tendência ao equilíbrio. Assim como um pião ou a roda de uma bicicleta, que só encontram seu equilíbrio em movimento, a imagem plástica alcança unidade em movimento, em relações perpétuas de cores contrastantes.

\vspace{\baselineskip}
\textit{A. M. Cassandre. Cartaz}

\vfill
185

\newpage
\section*{Page 188}

\begin{center}
\textit{O processo de fazer}
\end{center}

\vspace{1em}

O mundo espacial não consiste em unidades criadas instantaneamente, mas em processos de vir a ser, transformações incansáveis de configurações espaciais. A natureza forma; flores, árvores, rochas, montanhas, formações de nuvens, corpos animais ou humanos, bem como formas feitas pelo homem; edifícios ou implementos, são apenas configurações temporárias no fluxo perpétuo de vir a ser e desaparecer. Toda forma, portanto, é um registro visível inevitável de origem. As configurações espaciais dos galhos das árvores, as formas de metais derretidos transmitem sua história de surgimento, assim como uma pegada na neve ou na areia, a forma da tinta derramada da garrafa, ou o padrão de linhas desenhado por um lápis em um papel. O passado espaço-tempo --- movimento --- é inerente a toda forma.

Mas o pano de fundo espaço-tempo que resultou em uma configuração pode ser tão vasto que está além do limiar da nossa capacidade de compreendê-lo. Existem inúmeras formas na natureza cuja história nativa está inteiramente oculta porque a escala de complexidade em sua origem é muito vasta. Não se pode perceber instantaneamente em uma folha, em uma rocha, o fundo cinético de seu vir a ser.

\vspace*{\fill}

\noindent 186 \hfill \textit{F. Levstik. Fotografias}

\newpage
\section*{Page 189}

\begingroup % Localize settings for this page content
\raggedright % Ensure left alignment for all paragraphs/lines

\noindent % Prevent initial paragraph indent

\begin{minipage}[t]{0.35\textwidth} % Left column content block
\textbf{L. Levstik.} \\
\textit{Estudo de Tratamento de Superfície}

\vspace{2.5cm} % Vertical space between the two left text blocks

\textbf{T. Hauge. Tratamento de Superfície} \\
\textit{Trabalho realizado para o curso do autor} \\
\textit{em Fundamentos Visuais} \\
\textit{Patrocinado por} \\
\textit{O Clube de Diretores de Arte de Chicago}
\end{minipage}
\hfill % Horizontal fill to push the next minipage to the right
\begin{minipage}[t]{0.6\textwidth} % Right column content block
\vspace*{2cm} % Vertical space to position the top of this block relative to the left column's top

\textit{O processo material de criação; tratamento de superfície}

\vspace{0.5cm} % Small space between heading and paragraph

Fazer é uma atividade espacial, e o caminho visível do movimento de uma ferramenta sobre um meio é uma mensagem espacial. Não se pode deixar de ver, além de toda configuração espacial, a força, a velocidade e a direção do movimento que a criou. Cada imagem criada serve, portanto, como uma expressão ótica de movimento. A ação e o poder da ferramenta, a estrutura e a textura da superfície resistindo à ferramenta, determinam a estrutura do caminho visível. Consequentemente, cada ferramenta e cada material tem seu próprio idioma de movimento. A imagem criada é feita pelo homem, e aqui o fundo espacial dinâmico adquire um novo significado. Fazer implica movimento corporal. Os movimentos corporais, por sua vez, evocam um prazer cinético, satisfação nervosa. Ao ver qualquer sinal visual criado pelo homem, identifica-se inevitavelmente com o criador. Acompanha-se o traço visível dos movimentos e revive-se novamente todos os passos de uma coordenação neuromuscular da criação original. Uma linha ou uma superfície sugere o grau de controle em sua criação. Pode ter ousadia e fluência ditadas pela autoconfiança da habilidade, ou pode ser hesitante, não controlada. Assim, uma linha ou uma superfície possui uma qualidade cinética inata independente do que representa e de sua relação plástica na superfície.
\end{minipage}
\endgroup

\vfill % Push content to top of the page, and the page number to the bottom

\noindent\hfill 187

\newpage
\section*{Page 190}

O tratamento de superfície, isto é, o caminho visível do ato criativo, determina a autenticidade da expressão. Na cultura ocidental, a representação, com sua representação servil e obediente do mundo dos objetos, dificultou o uso honesto do tratamento de superfície. Com cuidado meticuloso, os pintores eliminaram todos os sinais da feitura. Eles se esforçaram para camuflar o fato de que a imagem criada é uma realidade diferente do assunto real. Nesta falsificação, uma importante qualidade orgânica da imagem foi perdida. Na arte clássica, no entanto, há muitos exemplos para provar que seus grandes mestres entenderam a qualidade cinética implícita no tratamento de superfície. A maioria dessas obras foi produzida no auge do desenvolvimento dos pintores e nunca se tornou para eles padrões de expressão pictórica. Somente pintores contemporâneos, através de uma longa e consistente luta, fizeram do tratamento de superfície um fator integral da expressão visual. O mesmo trabalho pioneiro que resultou na libertação dos elementos plásticos básicos, cor, planos e linhas, também recuperou o tratamento de superfície.

A demanda consciente por um respeito genuíno ao processo de criação, no entanto, surgiu de uma necessidade mais geral do que a da linguagem da visão. A caça ao lucro míope --- o canibalismo do século XIX --- destruiu quase todos os aspectos vivos do processo de trabalho, da atividade criativa. A devoção cega à quantidade levou à escravidão do homem à máquina. A crescente mecanização da produção, com toda a sua compulsão de uniformidade, rapidamente levou ao desaparecimento do verdadeiro artesanato baseado no respeito pela verdade do material, da ferramenta e do criador. Esta vergonha da verdadeira natureza da criação, este desrespeito às qualidades inerentes de ferramentas e materiais, tornou-se uma epidemia perigosa em todos os campos do esforço humano. Desde a criação do objeto mais simples do dia a dia até as mais amplas dimensões da expressão, uma atitude falsa foi dominante. Não só eliminou todo o prazer rítmico na criação, o prazer do trabalho, mas também eclipsou a compreensão dos materiais e ferramentas. No início do século passado, Carlyle, e um pouco depois Ruskin e Morris, reconheceram as consequências devastadoras da licença da produção em massa na atividade criativa e, assim, na vida do homem. Eles discerniram claramente que o desajuste de cada material e ferramenta que o homem usa implica o desajuste do próprio homem. Eles perceberam que o homem deve redescobrir em cada processo de trabalho o prazer do trabalho, a experiência de formar a arte, para chegar a uma existência integrada. ``Aquilo que entendo por arte real é a expressão do prazer do homem no trabalho. Não acredito que ele possa ser feliz no trabalho sem expressar essa felicidade; \dots Um presente muito gentil é este da natureza, pois todo homem, sim, parece que todas as coisas, também devem trabalhar; \dots'' escreveu William Morris. Esta geração enfrenta a tarefa de realizar esta visão e estendê-la no plano social mais amplo. É uma tarefa importante, pois envolve não apenas a revitalização da arte visual como tal, mas ainda mais verdadeiramente o desenvolvimento de sensibilidades aguçadas treinadas para a missão de eliminar a falsidade e a farsa das relações humanas.

188

\newpage
\section*{Page 191}

\begin{raggedright}
Rudolph Bauer. Presto 1929\\
Cortesia de\\
O Museu Guggenheim\\
de Arte Não-Objetiva
\end{raggedright}

\begin{center}
189
\end{center}

\newpage
\section*{Page 192}

W. Hogarth.

\itshape Diagrama de dança \par
\itshape Análise da Beleza \par

\bigskip

``Nova Arte de Escrever'' Por Pedro Diaz Morante Madrid 1626 \par
Cortesia da Biblioteca Newberry \par

\bigskip

O que faço, no que me ocupo, no que me encanto? Ali me assombrará a longa lista das visões horrendas infernais. A memória terrível, tão amarga falha que condena, e outros males. Pois como (cego) com tão grave carga de angústias e tormentos desiguais. Não tremo, não me emendo, não me espanto; louco devo desejar, pois não sou. Morante o escreveu.

\newpage
\section*{Page 193}

\noindent Gyorgy Kepes. Fotograma Com Linhas 1939

\newpage
\section*{Page 194}

F. Levstik.

\textit{Tratamentos de Superfície Contrastantes}



Bobri. \textit{Desenho}



\begin{center}
192
\end{center}

\newpage
\section*{Page 195}

Assim como cada registro individual possui sua própria qualidade intrínseca de movimento---traços lineares evocam uma experiência de diferentes velocidades e ritmo; uma superfície pulverizada ou impressa a sensação de um surgimento quase instantâneo---a combinação, na mesma superfície, de uma variedade de tratamentos cria uma experiência visual qualificada pela tensão.

Paul Rand. Design Publicitário

história-1

\newpage
\section*{Page 196}

{\centering\itshape O processo psicológico do fazer\par}
\vspace{\baselineskip} % Add some vertical space after the heading

O processo físico de fazer, a execução da imagem, é apenas parte do devir. Ferramentas físicas e o meio condicionam este crescimento; eles não definem sua direção final. A mente do homem cria a imagem; seu sistema nervoso é a ferramenta básica.

A imagem cresce no sentido de que o homem vê o que ele quer ver. Assim como cada ferramenta tem sua própria maneira única de viver na superfície da imagem, cada indivíduo tem sua própria maneira de ligar sinais ópticos em formas e imagens que ele gostaria de ver. Como Hans Arp disse de forma tão convincente: ``A arte é um fruto que nasce do homem, como o fruto de uma planta, como um filho de sua mãe.''

Se alguém olha para uma formação de nuvens, ou a mancha de tinta, e encontra nelas rostos, montanhas, animais, cria-se imagens que são modeladas por processos mentais inconscientes. A imagem criada, uma pintura, tem gênese semelhante; é ditada por necessidades emocionais, advindo, portanto, de reinos inconscientes.

Não se pode suportar o caos no espaço psicológico de alguém, assim como não se pode suportar o caos dos impactos ópticos do espaço geográfico. O homem organiza o caos óptico formando totalidades espaciais significativas. Assim também ele organiza o caos de seu espaço psicológico, formando imagens visuais de seus desejos, equilíbrios temporários nos conflitos perpétuos de prazer e realidade; impulsos e tabus sociais. Os resultados de sua imaginação criativa são aceitos por ele como formas reais de sua existência. Como afirmou Freud: ``Somente em um campo a onipotência do pensamento foi retida em nossa civilização, a saber, na arte. Somente na arte ainda acontece que o homem, consumido por seus desejos, produz algo semelhante à gratificação desses desejos, e esse brincar, graças à ilusão artística, evoca efeitos como se fossem algo real.''

Os mesmos eventos sociais que causaram a exaustão do ritmo --- o prazer de fazer --- do processo físico de fazer também esgotaram o processo interno de fazer de seu alimento mais essencial. O conceito de objeto estático, a perspectiva fixa do espaço psicológico, congelou o ritmo biomórfico da imagética visual. O fetiche de nosso tempo --- a mercadoria manufaturada mecanicamente --- imprimiu seu padrão no processo criativo. Não foi por acaso que um dos primeiros rebeldes contra os erros da revolução industrial, Carlyle, escreveu em 1831: ``O artificial é o mecânico consciente; o natural é o inconsciente, dinâmico. A inconsciência é o sinal da criação; a consciência, na melhor das hipóteses, é a da manufatura.'' Hoje, o protesto tomou forma veemente. Artistas contemporâneos, revoltando-se contra as amarras do conceito estático, descartam todo controle consciente. O empreendimento artístico foi reduzido a uma mera assistência de acontecimentos aleatórios. O artista atua como parteira. Ele apenas auxilia no nascimento de uma forma viva que cresce de estratos mais profundos do que seus esforços conscientes poderiam alcançar. Ele inventa técnicas que oferecem os mínimos obstáculos ao fluxo livre da formação orgânica.

\vspace*{\fill}
\centering
194

\newpage
\section*{Page 197}

Hans Arp recortou pedaços de papel colorido e, com deliberado abandono, jogou-os sobre um pedaço de papelão, virou-os e finalmente colou no papelão o padrão que formaram por acaso. Tal acaso contém, no entanto, mais razão do que nós, com nossas vendas e sentidos confusos, podemos ver. A ordem resultante mostra uma compreensão orgânica muito mais abrangente do que o conceito formal, aguçado pela lógica, de objeto estático. É natural que essas expressões automáticas se assemelhem aos reinos biomórficos da natureza. Elas possuem a mesma ordem das formas visíveis de mutações, transformações, o ritmo assimétrico perpétuo dos processos ainda não fossilizados em termos de coisas.

Hans Arp. \textit{Montanha, Mesa, Âncoras, Umbigo} 1925\\
Cortesia do Museu de Arte Moderna

Microfotografia\\
de Espículas de Esponja

195

\newpage
\section*{Page 198}

\noindent \textit{Manchas de Tinta}\par

\vspace{1em} % Espaçamento vertical para separar visualmente as seções.

\noindent O pensamento plástico, o pensamento com os sentidos, afirmava os desejos e a vontade dos homens opostos ao controle da máquina. Tendo alcançado o domínio científico de um novo e vasto território da natureza e sua ordenação em uma dimensão tecnológica unilateral, o homem buscava um contato renovado com a pulsação das forças dinâmicas dos processos naturais. Ele reconheceu que o progresso tecnológico científico precisava ser reavaliado em dimensões biológicas. Em vez do antigo ponto fixo de perspectiva, ele desenvolveu, para atender à sua necessidade, a perspectiva do crescimento em vez da ordem estática, o ritmo dinâmico. O artista redescobriu a natureza. Mas ele se afastou da representação naturalista das formas das árvores, flores e animais, e tomou como seu novo tema os processos visíveis do crescimento.\par

\vspace{1em} % Espaçamento vertical para separar visualmente as seções.

\noindent Juan Miró.\\
\textit{Interior Holandês 1928}\\
\textit{Cortesia de Art of This Century}\par

\vfill
\noindent 196

\newpage
\section*{Page 199}

\noindent Paul Klee. \textit{Planta Masculina e Feminina} 1921.
\par
\noindent Cortesia de Art of This Century.
\par
\noindent Detalhes adicionais (manuscritos na imagem):
\begin{itemize}
    \item Assinatura: Klee
    \item Número de catálogo/inventário: 1921/76
    \item Título original em alemão: \textit{weibliche und männliche Pflanze} (planta feminina e masculina)
\end{itemize}
\par
\noindent Página: 197

\newpage
\section*{Page 200}

Herbert Bayer. Mensagens Através da Atmosfera 1942

\vspace*{0.15\textheight}

\noindent\hspace*{0.55\textwidth} Herbert Bayer. Pintura 1942

\vspace*{0.12\textheight}

\noindent\hspace*{0.55\textwidth} Joseph Feher. Design Publicitário

\vfill

\raggedleft 198

\newpage
\section*{Page 201}

Paul Rand. Design de Publicidade

VERÃO...

\newpage
\section*{Page 202}

\centering\textit{Rumo a uma iconografia dinâmica}\par

\bigskip

A experiência visual é mais do que a experiência de qualidades puramente sensoriais. As sensações visuais se entrelaçam com superposições de memória. Cada configuração visual contém um texto significativo, evoca associações de coisas, eventos; cria respostas emocionais e conscientes.

A imitação literária da natureza ligada a um ponto de observação fixo havia matado a imagem como um organismo plástico. Era bastante natural, portanto, que o significado associativo fosse identificado com o conteúdo literário e, consequentemente, descartado como desnecessário. A arte não representacional esclareceu as leis estruturais da imagem plástica. Ela restabeleceu a imagem em seu papel original como uma experiência dinâmica baseada nas propriedades dos sentidos e em sua organização plástica. Mas ela descartou os sinais significativos das relações visuais.

A imagem foi "purificada". Mas essa purificação ignorou o fato de que a distorção e a desintegração da imagem como experiência plástica não se deviam a sinais significativos representados como tal, mas sim ao conceito de representação predominante que era estático e limitado, e consequentemente em contradição com a natureza plástica dinâmica da experiência visual. A estrutura de significado baseou-se na mesma concepção que gerou o ponto de vista fixo da representação espacial, a perspectiva linear e a modelagem por sombreamento. As coisas foram representadas juntas em um sistema fixo de ordem empírica e seu significado também adquiriu a característica dessa fixidez.

Juan Gris, um dos mais proeminentes pintores que trabalham em direção à nova linguagem da visão, deixou claro que uma nova e saudável estrutura plástica não é um objetivo final, mas apenas um novo começo para a compreensão dos valores inerentes às relações dos elementos significativos da natureza visível. ``Procuro dar forma concreta ao que é abstrato. Passo do geral ao particular, com o que quero dizer que tomo a abstração como meu ponto de partida e o fato real como meu ponto de chegada. \ldots Considero a matemática o lado arquitetônico da pintura, o lado abstrato, e quero humanizá-la; Cézanne faz um cilindro de uma garrafa. Comecei com o cilindro para criar uma unidade individual de tipo especial. De um cilindro faço uma garrafa, uma garrafa particular. Cézanne trabalha em direção à arquitetura. Eu me afasto dela: é por isso que componpo com abstrações, (em cores) e eu organizo quando essas cores se tornam objetos; por exemplo, eu componho com um branco e um preto e eu organizo quando o branco se torna um papel e o preto uma sombra! Quero dizer que eu organizo o branco para que se torne um papel e o preto para que se torne uma sombra.''

Whitehead, um dos principais estudiosos de hoje, entende isso quando escreve:

\vfill
\centering 200

\newpage
\section*{Page 203}

Assim, 'arte', no sentido geral que exijo, é qualquer seleção pela qual os fatos concretos são organizados de modo a suscitar atenção para valores particulares que são realizáveis por eles. Por exemplo, a mera disposição do corpo humano e da visão, de modo a obter uma boa vista de um pôr do sol, é uma forma simples de seleção artística. O hábito da arte é o hábito de desfrutar valores vívidos.

Quais são esses valores? Quais são os fatos concretos? Valor é, de modo muito geral, aquilo que torna algo útil. Valores são, em termos humanos, as diretrizes reconhecidas para uma vida humana mais satisfatória. Eles são a "ordem" potencial compreendida na relação do homem com a natureza e com seus semelhantes. A ordem só faz sentido como uma ordem em um campo definido. Os valores são condicionados por eventos concretos dos reinos físico, psicológico e social. Os valores ainda não foram formulados para o nosso tempo. Estamos vivendo em uma era sem forma de transição, de caos, incomparável a tudo o que o homem já experimentou antes. Nessa confusão, a arte plástica, a experiência mais direta da ordem, a atividade formadora por excelência, ganha significado. A ordem para o nosso tempo só pode ser formulada nos termos concretos do campo dinâmico das forças sociais presentes. Somente se abrangermos no pensamento e na visão as forças dinâmicas, as contradições presentes nos processos biológicos e sociais, seremos capazes de resolvê-las. Somente se pudermos guiar os eventos do nosso tempo em direção a uma organização social "planejada" e integrada, poderemos alcançar um novo equilíbrio temporário; uma vida humana mais satisfatória.

Pensar e ver, em termos de coisas estáticas e isoladas idênticas apenas consigo mesmas, possuem uma inércia inicial que não consegue acompanhar o ritmo da vida, portanto não podem sugerir valores---ordem plástica---intrínseca a este campo dinâmico da existência social. O senso comum considera o repouso e o movimento como processos inteiramente diferentes. No entanto, o repouso é, na realidade, um tipo especial de movimento, e o movimento é, em certo sentido, um tipo de repouso. A imagem plástica pode cumprir sua presente missão social apenas abrangendo essa identidade de direções opostas e referindo-a a experiências sociais concretas. A linguagem visual herdada fossilizou os eventos em um sistema de sinais estático. A revolução nas artes plásticas trouxe de volta uma abordagem dinâmica no nível sensorial. As estruturas plásticas devem se expandir para absorver, sem perder sua força plástica, as imagens significativas das experiências sociais concretas atuais. A tarefa do artista contemporâneo é liberar e trazer para a ação social as forças dinâmicas da imagética visual. Assim como os cientistas contemporâneos estão lutando para liberar a energia aprisionada do átomo, os pintores de nossos dias devem liberar o reservatório de energia inesgotável das associações visuais. Para realizar isso, eles precisam de uma compreensão clara do campo social, honestidade intelectual e poder criativo capaz de integrar experiências em uma forma plástica. Este objetivo só será alcançado quando a arte, mais uma vez, viver em unidade inseparável com a vida humana.

\vspace{1em}
\noindent $\bullet$ Whitehead, \textit{Science and the Modern World}

\newpage
\section*{Page 204}

\centering \textit{Leis de organização de signos visuais significativos}
\par
\vspace{1em}

\noindent Cada representação de um objeto ou de uma coisa age na superfície da imagem e descarrega sua própria direção única de associações como um ponto, uma linha, uma forma, age no plano da imagem, e força o olho em direções espaciais virtuais. Essas representações possuem posições, direção, forma, tamanho, distância e peso. Elas podem avançar até que se esteja propenso a segui-las, ou podem recuar para que se esteja disposto a perdê-las. Elas possuem texturas com calor sensorial; ou são frias, com exatidão geométrica ou teórica. Elas possuem brilho e cor e podem mover-se com várias velocidades. À medida que se busca a ordem espacial, e através das inter-relações das forças plásticas cria-se um todo espacial unificado, também se busca uma ordem de significado e constrói-se, a partir das diferentes direções de associação, o todo comum e significativo.

Observamos uma fotografia de dois homens sentados em um banco, e cada unidade da imagem evoca associações. Um homem está mais bem vestido que o outro. Eles estão sentados de costas um para o outro. Seus corpos, suas posturas estão cheios de sugestões associativas. Nós os comparamos e contrastamos, descobrindo diferenças e similaridades. Tentamos preencher as diferenças usando as similaridades. A imagem se torna uma experiência dinâmica. Ela possui um auto-movimento devido à oposição descoberta. A experiência atinge uma unidade à medida que preenchemos, com uma história viva, o latente pano de fundo humano da situação visível. Não vemos coisas, unidades estáticas fixas, mas percebemos, em vez disso, relações vivas. Olhamos para a fotografia de um olho preso na lama e vemos na mesma imagem arame farpado. A contradição inerente nas associações dos respectivos elementos mantém nossa mente em movimento até que a contradição seja resolvida em um significado; até que esse significado, por sua vez, se torne uma atitude em relação às coisas ao nosso redor e sirva de fermento para protesto contra a vida em condições desumanas. A contradição é, então, a base da organização dinâmica das qualidades associativas da imagem. Quando unidades representacionais dentro da mesma imagem contêm afirmações que parecem contrárias à lógica aceita dos eventos, a atenção do espectador é forçada a buscar as possíveis relações até que uma ideia central seja encontrada que tece os signos significativos juntos em um todo significativo. Os campos de associação da representação de um homem, uma árvore, uma máquina, e assim por diante, em sua combinação na superfície de uma imagem, podem reforçar uns aos outros ou colidir uns com os outros, criando tensões, pressões, atritos. Cada tensão é resolvida em uma configuração de significado. Essas configurações, por sua vez, servem de base para tensões adicionais; consequentemente, para configurações adicionais. Os signos significativos possuem, assim, suas leis de organização semelhantes à organização plástica. Mas enquanto a relação das qualidades plásticas emerge através da organização dinâmica do espectador em um todo espacial, no caso da organização dos signos significativos, o todo unificador possui as dimensões de atitudes, sentimentos e pensamentos humanos.

\vspace*{\fill}
\centering 202

\newpage
\section*{Page 205}

\centering
F. Levstik. Fotografia

\vspace*{25\baselineskip}

\centering
N. Lerner. Olho e Arame Farpado

\vfill

\raggedleft 203

\newpage
\section*{Page 206}

Foi apontado antes que a linha de nossa comunicação com o pano de fundo dinâmico da associação foi cortada por um curto-circuito quando elementos pictóricos eram apresentados em correspondência estática com as coisas que representavam, quando a representação das coisas e as próprias coisas eram consideradas idênticas. O caminho levou então, não à descoberta progressiva e evolutiva das relações entre as coisas, mas apenas às próprias coisas e seu significado associativo. Sobre uma fraca base plástica, tornou-se progressivamente perigoso construir uma estrutura de associações referentes a sentimentos e valores concretos. Mas depois que os pintores testaram e reconstruíram a base plástica, a expressão visual avançou para usar essa estrutura para nos redirecionar à ordem plástica --- a ordem significativa --- na vida que o homem faz para si mesmo.

Os estágios de desenvolvimento pelos quais o uso estrutural das associações passou correspondem àqueles na busca pelas leis da organização plástica. A unidade de significado foi primeiramente desintegrada em facetas de significado. Mais tarde, essas facetas de significado foram compreendidas em suas interconexões, avaliadas como forças e campos, testadas em suas tensões, equilíbrio dinâmico, e reorganizadas em um novo todo significativo.

\textbf{Desintegração do sistema fixo de organização do significado}

A desintegração foi provocada pelas contradições sociais em que o homem vivia. No final da última guerra, em revolta contra uma aparente bagunça sem esperança de frustração política e cultural, contra a pirataria da mediocridade, valores falsos e autoridade falsa, os homens descartaram avidamente todo valor, e assim todo significado. Artistas, não menos que outros homens, perderam sua crença na significatividade de suas próprias vidas e da vida em geral. Em ódio cego eles se propuseram a destruir tudo que continha o menor vestígio de coerência significativa. Instituições sociais, costumes, valores éticos ou morais, sentimentos e obras de arte eram, declararam, uma perpetuação do velho disparate --- um câncer na existência humana. Eles viam ao seu redor evidências de um esforço tremendo e meticuloso na criação de obras de arte cujos resultados finais eram desprovidos de estrutura de significado social da arte. Com amarga ironia, eles tomaram fragmentos como sua matéria-prima plástica: o lixo da lixeira, jornais, bilhetes de bonde, papel mata-borrão, botões antigos, fotografias rasgadas, cartões-postais. Mas eles não conseguiam se livrar do desejo instintivo de formar, de moldar uma ordem plástica. Eles uniram todos esses fragmentos que foram tirados de seu contexto e não tinham conexão lógica alguma, e essa aglomeração aleatória de sinais fragmentados e não relacionados de imagens significativas revelou um poder de expressão inesperado.

204

\newpage
\section*{Page 207}

\begin{center}
Kurt Schwitters. Relevo Merzbild 1915 \\
\textit{Cortesia de Art of This Century}
\end{center}

Cada material, cada forma, cada fotografia carregava em si características do mundo do qual foi tirada. O observador foi forçado a encontrar ordem nos fragmentos não relacionados, a rastrear algumas conexões significativas latentes nas obras Dada e Merz, colagens ou fotomontagens, basicamente sem sentido e aleatórias. Quanto mais distantes os elementos estavam em significado e mais impossível parecia encontrar integração para eles, maior se tornava a tensão do espectador enquanto ele lutava para encontrar uma fonte de integração. Essa tensão foi um ponto zero da organização-significado. Serviu como base para a redireção.

\vfill
\raggedleft 205

\newpage
\section*{Page 208}

Harold Walter. Colagem

M. Martin Johnson. Colagem

\noindent\rule{\linewidth}{0.4pt}
\vspace{6pt}
\textsuperscript{*} \textit{Trabalho realizado para o curso do autor em Fundamentos Visuais}

\textit{Patrocinado pelo Art Director's Club de Chicago, 1938}

\newpage
\section*{Page 209}

\section*{Reintegração}

Assim como, após a desintegração da perspectiva renascentista fixa, linhas e planos de cores revelaram uma qualidade dinâmica e se moveram em todas as direções espaciais, assim também, após a desintegração da unidade de significado fixa da lógica tradicional, energias associativas inerentes em cada fragmento visível da realidade foram subitamente liberadas. O próximo passo foi dado em direção à reintegração dessas facetas de significado liberadas.

A nova tendência---e, em grande parte, seus resultados---não era tão nova quanto parecia. Como as inovações revolucionárias na representação espacial haviam redescoberto a base original da imagem, as pesquisas no manuseio da ``matéria'' e a ordenação dinâmica do elemento de significado restabeleceram um antigo princípio básico da expressão criativa, a liberdade de expressão de um naturalismo unilateral. Reafirmou este princípio com uma consistência até então inigualável. Reforçou a iconografia simbólica primitiva sem sangue com uma base dinâmica sensorial de organização plástica. Avaliou a alegoria, o equilíbrio estático de signos significativos, em um equilíbrio dinâmico.

Havia muitas direções convergentes nessas tentativas de ligar as facetas de significado liberadas em um novo todo dinâmico. A pintura, enriquecida com novos idiomas, colagem e fotomontagem, contribuiu para a compreensão estrutural da relação dos signos representacionais e abriu caminho para essa redireção. O cinema fez a primeira análise aprofundada da conexão estrutural de imagens representacionais em sequência temporal real. A arte publicitária foi pioneira no teste de imagens representacionais em combinação com unidades plásticas puras e elementos verbais.

Cartão-postal de imagem mexicana.

Giuseppe Arcimboldo. Verão

\small{Cortesia de Reprodução}

\small{O Instituto de Arte de Chicago}

Xanti Schawinsky. Guerra

\newpage
\section*{Page 210}

G. Apollinaire. Ideograma

todo meu elemento
Guillaume Apollinaire

György Kepes. Estalingrado 1942

ESTALINGRADO RESISTE

\vfill
\hspace*{\fill}208

\newpage
\section*{Page 211}

\vspace*{4in} % Espaço para a imagem
\begin{flushright}
György Kepes. Fotografia Aérea China 1942 Colagem \\
Cortesia do Sr. H. Ziebolz
\end{flushright}

Apollinaire em seus ideogramas, Miro em seu poema-pintura incorporam a palavra escrita no conjunto plástico com uma interação dinâmica do significado verbal e das qualidades sensoriais dos elementos pictóricos. Esses pintores estão fundindo os dois em uma expressão que evoca associações de grande profundidade devido à intensidade sensorial dos valores plásticos, e de grande amplitude devido às associações descarregadas pela base linguística. Cor, forma e textura, linha e símbolo atingem uma unidade orgânica e, assim, treinam o espectador para formar em um todo orgânico suas próprias experiências das qualidades divergentes.

Vimos que a imagem se torna uma experiência viva no nível sensorial apenas através da participação dinâmica do observador. Vimos que a experiência plástica é baseada na tendência dinâmica do observador, que não suporta o caos, não suporta a contradição e, consequentemente, busca a ordem, por um todo unificado que pode ligar as direções espaciais visuais aparentemente opostas ou contraditórias das unidades visuais em uma unidade espacial. Uma participação dinâmica similar também promove a integração de sinais visuais significativos.

A fibra viva de nossas respostas inconscientes é dada pelas imagens concretas dos eventos circundantes. A expressão visual, baseada na compreensão da estrutura dinâmica da imagética visual, pode ser inestimável para reajustar nosso pensamento como um processo dinâmico. Quando a organização plástica e a organização dos sinais significativos são sincronizadas em uma estrutura dinâmica comum, temos um implemento significativo de progresso. Tais imagens sugerem um novo hábito de pensamento, reforçado com a força elementar da experiência sensorial. A partir delas, o sistema nervoso pode adquirir a nova disciplina necessária para a dinâmica da vida contemporânea.

\vspace{\fill}
\begin{flushright}
209
\end{flushright}

\newpage
\section*{Page 212}

\begin{flushleft}
\textit{Motivações psicológicas da reintegração}
\end{flushleft}

As consequências do caos social finalmente penetraram nas regiões mais abrigadas da vida individual. O homem desajustado perdeu o controle ao manter em segundo plano os eventos dinâmicos em seu espaço psicológico. Impulsos sexuais destrutivos, o medo da morte e o desconhecido que foram domesticados por sonhos e mitologia em culturas mais integradas, perderam seus mestres e, correndo descontroladamente, invadiram os domínios conscientes. Não se podia mais negar que o subconsciente é o verdadeiro pano de fundo dos eventos psicológicos e que o consciente, usando a comparação de Freud, é como a pequena fração visível de um iceberg, da qual a maior parte tem sua existência ameaçadora submersa sob o mar.

O avanço científico e o domínio óptico da realidade, em progresso, novamente seguiram avenidas convergentes. O processo de compreensão do espaço físico se repetiu no espaço psicológico. Na ciência, a geometria euclidiana foi reconhecida apenas como a primeira aproximação do espaço e, na pintura, a perspectiva fixa como uma representação insuficiente das experiências espaciais. Na psicologia, a região consciente foi compreendida apenas como um complexo limitado de eventos psicológicos, e sua representação na arte como apenas o primeiro passo de sua expressão criativa. E assim como os cientistas pioneiros encontraram uma representação "mais real" do espaço físico ao fundir espaço e tempo em uma unidade indivisível, e os pintores pioneiros uma representação "mais real" ao soldar objetos e fundo em uma unidade plástica dinâmica pela interpenetração de planos de cores e linhas, também os pioneiros do espaço psicológico buscaram "um mundo mais real do que o real por trás do real", pela fusão das experiências conscientes e subconscientes; nas palavras de André Breton: "a futura resolução dos dois estados (aparentemente contraditórios), sonho e realidade, em uma espécie de realidade absoluta." O objeto que havia sido analisado pelos pintores cubistas no pano de fundo do campo espaço-tempo, estava agora sendo analisado pelos pintores surrealistas no campo das associações subconscientes. "É essencialmente sobre os objetos que o surrealismo lançou mais luz nos últimos anos. Somente o exame muito atento das muitas especulações recentes às quais o objeto deu origem publicamente (o objeto onírico, o objeto que funciona simbolicamente, o objeto fantasma, o objeto descoberto, etc.), pode-nos dar uma compreensão adequada dos experimentos em que o surrealismo está engajado agora", disse André Breton em "O que é o Surrealismo".

O subconsciente, manifestado em sonhos e associações livres, possui outra lógica que não a do espaço-tempo derivada de fatos empíricos, e a nova imagem-pintura, em um automatismo soberano, reuniu a representação de objetos não relacionados nas experiências cotidianas. Mas a importância de um ou outro objeto na esfera subconsciente frustrada ditou seu tamanho e posição na superfície da imagem. A estrutura do conteúdo associativo dominou a organização da imagem. Consequentemente, a ordem plástica foi novamente restringida.

\vspace*{\fill}
210

\newpage
\section*{Page 213}

\textbf{PORTRAIT DE L'ACTEUR A B} \\
\textbf{RETRATO DO ATOR A B}

\textbf{DANS SON ROLE MEMORABLE} \\
\textbf{EM SEU PAPEL MEMORÁVEL}

\textbf{L AN DE GRACE 1713} \\
\textbf{O ANO DA GRAÇA 1713}

D'un judas de Port-Royal \\
De um judas de Port-Royal \\
détruite \\
destruído \\
mais invulnérable \\
mas invulnerável

je te vise \\
eu te miro \\
pape Clément XI \\
papa Clemente XI \\
vieux chien \\
velho cão

\textit{Andre Breton. Portrait of An Actor A. B. 1941} \\
\textit{André Breton. Retrato de Um Ator A. B. 1941}

\textit{Courtesy of Art of This Century} \\
\textit{Cortesia de Art of This Century}

\textit{Herbert Bayer. Deposition 1940} \\
\textit{Herbert Bayer. Deposição 1940}

\newpage
\section*{Page 214}

Pintura de Max Ernst.
Cortesia de Art of This Century

212

\newpage
\section*{Page 215}

\begin{flushleft}
Marc Chagall, Paris Pela Janela 1912 \\
Cortesia do Museu de Arte Moderna
\end{flushleft}
\vspace*{\fill}
\raggedleft 213

\newpage
\section*{Page 216}

Francis Picabia. \textit{Imagem Muito Rara Sobre a Terra}

Cortesia de \textit{Arte Deste Século}

TRÈS RARE TABLEAU SUR LA TERRE

Quadro Muito Raro Sobre a Terra

\newpage
\section*{Page 217}

Gyorgy Kepes. Passado Húngaro. Fotograma.

215

\newpage
\section*{Page 218}

Paul Klee. \textit{Máscara do Medo}\\
Cortesia do Museu de Arte Moderna

216

\newpage
\section*{Page 219}

\noindent Algumas expressões visuais, no entanto, se aproximaram do objetivo de sincronizar a estrutura plástica com imagens de eventos sociais concretos. Embora sua abrangência e importância estejam em níveis diferentes, elas indicam um novo caminho para a expressão visual. Picasso, impulsionado a uma fúria de indignação por um drama humano causado pelas forças sociais regressivas de seu tempo, apresenta em Guernica imagens de dor e sofrimento. Boca e lábios, nariz e narinas, são moldados à dor em posições distantes da realidade empírica, mas próximas o suficiente para serem reconhecíveis em termos familiares. As lágrimas estão em ação como uma bomba que explode. Os planos plásticos e as superfícies texturizadas agem como indivíduos que sofrem. As linhas tentam escapar da superfície da imagem, correndo com uma fúria óptica incrível; os planos seguem uns aos outros em uma sequência rítmica que se assemelha aos gritos de uma sirene de perigo. As texturas desmembram o corpo vivo. Mas todas essas forças plásticas violentas são organizadas como uma progressão visual na qual cada parte exige a outra, e só pode viver com a ajuda da outra. Uma forma assume a direção de outra forma; um valor tonal repete ou contrasta um valor anterior no todo. Dois sistemas contraditórios, organização plástica---a mensagem de ordem---e a organização de um todo significativo---a mensagem do caos---são fundidos em um todo indivisível.

\noindent \textbf{Picasso. Guernica}\\
Reprodução Cortesia do Instituto de Arte de Chicago

\newpage
\section*{Page 220}

El Lissitzky. Ilustração 1923

Moholy Nagy. Leda. Fotomontagem 1926

218

\newpage
\section*{Page 221}

\normalfont\itshape Motivações tecnológicas. A invenção da fotomontagem\par
\vspace{\baselineskip} % Add some space after the heading

A complexidade da cultura da máquina confrontou a visão do homem com sérios obstáculos. Máquinas e muitos produtos de máquinas não podem ser compreendidos apenas pela sua imagem exterior. Uma máquina é uma unidade funcional e em movimento. A única maneira adequada de dominá-la visualmente é percebê-la em sua qualidade dinâmica, nas interconexões funcionais de suas partes visíveis. A fotografia naturalista, com seu ponto de vista fixo tradicional, não conseguia representá-la. A maioria das novas unidades técnicas eram mercadorias para venda; a retórica óptica, portanto, precisava fazer uso das projeções simultâneas. E muito antes que os pintores começassem a atacar o problema, a publicidade neste país já estava utilizando a fotomontagem. A solução se assemelhava à análise cubista do espaço, mas tinha esta diferença; enquanto nas pinturas cubistas a conectividade dos elementos era ditada pelo objetivo de tornar o objeto claro em todos os aspectos espaciais visíveis, na fotomontagem essa conectividade era ditada pelas relações funcionais e significativas dos elementos do objeto representado.

\par
A ideia de dissecar e rearranjar elementos fotográficos e combiná-los com desenhos foi levada adiante nas formas experimentais da fotomontagem. Assim como uma interação de engrenagens, o espaço era representado pela interação de linhas e formas sem referências naturalistas e por unidades fotográficas e desenhadas com fragmentos dos aspectos espaciais familiares. A imagem resultante produz uma experiência espacial dinâmica pela coordenação de representações de unidades tridimensionais reais e elementos plásticos puros de linhas e formas.

\vspace{1.5em} % Small vertical space for visual separation

\begin{flushright}
\itshape\small
Ruth Robbins. Montagem\\
Trabalho realizado para o curso do Autor\\
em Fundamentos Visuais\\
Escola de Design em Chicago
\end{flushright}

\vspace*{\fill} % Push content to the top, allowing page number at the bottom
\raggedleft 219

\newpage
\section*{Page 222}

Clifford Eitel.\\
Fotomontagem.

Elsa Kula Pratt.\\
Fotomontagem.\\
Trabalho feito para o curso do Autor\\
em Fundamentos Visuais\\
Escola de Design em Chicago

\vfill
\centering 220

\newpage
\section*{Page 223}

Todas essas descobertas vieram a focar nas tarefas práticas da arte publicitária contemporânea. A publicidade poderia utilizá-las porque não era prejudicada por formas tradicionais. A publicidade foi feita para utilizá-las porque pertencia à sua própria natureza ser contemporânea e vigorosa, e só poderia ser assim através do uso dos novos idiomas visuais dinâmicos. Uma mera ilustração de um fato ou de uma ideia não era vital o suficiente para induzir fortes respostas no espectador. Para transmitir uma mensagem publicitária de forma eficaz, os elementos mais heterogêneos---mensagem verbal, desenho, fotografia e formas abstratas---eram empregados. Essa variedade de signos e símbolos significativos só poderia ser integrada por uma organização de significado dinâmica. A publicidade visual, no entanto, tem o olho como seu cliente. Para satisfazer esse cliente, ela deve ser vital como uma experiência visual e deve oferecer conforto ao olho. Cada unidade significativa tem uma base óptica. Ela possui cor, valor, textura, forma, direção, tamanho e intervalo. A publicidade, para seu interesse bem concebido, aprendeu a usar a organização plástica dinâmica dessas qualidades ópticas; isto é, tornou-se uma arte. Aqui reside um grande desafio para a publicidade hoje. O ambiente artificial contemporâneo constitui uma parte muito grande do entorno visível do homem. Cartazes nas ruas, revistas ilustradas, livros ilustrados, rótulos de embalagens, vitrines e inúmeras outras formas existentes ou potenciais de publicidade visual poderiam então servir a um propósito duplo. Eles poderiam disseminar mensagens socialmente úteis, e poderiam treinar o olho, e assim a mente, com a disciplina necessária de ver além da superfície das coisas visíveis, para reconhecer e desfrutar de valores necessários para uma vida integrada. Se as condições sociais permitirem que a publicidade sirva a mensagens que se justifiquem no sentido social mais profundo e amplo, a arte publicitária poderia contribuir eficazmente para preparar o caminho para uma arte popular positiva, uma arte que atinge a todos e é compreendida por todos.

Morton Goldsholl. Design Publicitário 1943

É preciso EXPERIÊNCIA
\ldots um registro
contínuo
de
produção de produtos

A experiência é uma grande professora --- e o velho ditado tem duplo significado em relação à Produção de Guerra. Como há pouco tempo disponível agora para adquirir experiência, aqueles que já a possuem oferecem, de fato, uma vantagem inestimável.

Nas próximas duas páginas, são ilustrados alguns dos produtos fabricados por esta organização durante os anos intermediários, até, e agora incluindo, o conflito atual.

Dentro das limitações de espaço, sugere o escopo de nossa formação --- um registro contínuo de produção que manteve nossa organização treinada e uma fábrica e instalações adaptadas à velocidade e aos padrões peculiares da Produção de Guerra.

Desde projetos até produtos acabados, a experiência que adquirimos faz-se sentir e é a vantagem distinta para a solução de seus problemas de Produção de Guerra.

\newpage
\section*{Page 224}

\noindent\textbf{ELETRÔNICA - UMA NOVA CIÊNCIA PARA UM NOVO MUNDO} \\
Herbert Bayer. \textit{Design de Publicidade} 1943

\vspace{1.5em}

\noindent\textbf{ELÉTRONS NA MEDICINA} \\
Herbert Bayer. \textit{Design de Publicidade} 1943

\vspace{1.5em}

\noindent\textbf{cinema} \\
Harold Walter. \textit{Colagem} 1938

\vspace{\fill}
\raggedleft 222

\newpage
\section*{Page 225}

\raggedright
\textbf{ILUSTRAÇÃO}

\vspace{1em}

Collins Miller \& Hutchings Inc. \\
907 North Michigan Avenue \\
Franklin 5854

\vspace{1em}

\textit{Gyorgy Kepes. Design de Capa} \\
\textit{Cortesia de Collins, Miller \& Hutchings Inc.}

\vspace{4em}

\textit{Gyorgy Kepes. Design Publicitário} \\
\textit{Cortesia de Collins, Miller \& Hutchings Inc.}

\vfill
\raggedleft
223

\newpage
\section*{Page 226}

\raggedright

ARTES DO VESTUÁRIO

Paul Rand. Design de Capa

\textit{apresentando autênticas modas da Esquire}

ESCOPO

W. Burtin. \textit{Design Publicitário} 1941

224

\newpage
\section*{Page 227}

Alexei Brodovich. Cartaz 1942

liberdade
de palavra
UMA DAS QUATRO LIBERDADES PELAS QUAIS OS ALIADOS LUTAM

Joseph Feher, Design de Publicidade
Cortesia de Collins, Miller \& Hutchings Inc.

Collins, Miller \& Hutchings Gravadores de Fotos de Chicago

``VOC\^E PAGA SEU DINHEIRO E FAZ SUA ESCOLHA''

M. Lafitte n\~ao era pirata de forma alguma, mas um patriota leal, um contrabandista cavalheiro, um homem muito incompreendido. Como dizem os negros da Louisiana, ``Voc\^e paga seu dinheiro e faz sua escolha.''

Em sua Hist\'oria Informal do Submundo de Nova Orleans, Herbert Asbury aponta que in\'umeros livros e artigos provaram conclusivamente que Jean Lafitte foi um pirata, um assassino e um grande vil\~ao; outros livros e artigos, igualmente numerosos, provaram de forma igualmente conclusiva que

Ao comprar gravuras, voc\^e tamb\'em -- ``paga seu dinheiro e faz sua escolha.'' No entanto, ao comprar suas gravuras da Collins, Miller \& Hutchings, n\~ao h\'a escolha. Voc\^e sempre recebe as melhores gravuras que sabemos fazer a um pre\c co que \'e o mesmo para todos.

225

\newpage
\section*{Page 228}

El Lissitzky, Autorretrato 1924

\par\vspace*{15em} % Espaço para a primeira imagem

Herbert Bayer. Design Publicitário 1939

\par\vspace*{\fill} % Espaço para a segunda imagem e empurra o número da página para o final

\begin{center}
226
\end{center}

\newpage
\section*{Page 229}

A. M. Cassandre. Pôster 1925

Cortesia do Museu de Arte Moderna

O MAIS FORTE

O INTRAN

227

\newpage
\section*{Page 230}

\raggedright
A resposta da América!
PRODUÇÃO

Jean Carlu

DIVISÃO DE INFORMAÇÕES
ESCRITÓRIO DE GERENCIAMENTO DE EMERGÊNCIA

\bigskip

Jean Carlu. Cartaz 1940

\vfill
\centering
228

\newpage
\section*{Page 231}

\section*{SOBRE O AUTOR}

Gyorgy Kepes dividiu seu tempo entre pintura, ensino, experimentação com filmes e design comercial. Seu interesse dominante é o ensino e ele acredita na chegada de uma nova arte que combinará ciência e estética para o benefício de ambos. Nativo da Hungria, ele trabalhou na Áustria, Alemanha, Inglaterra e, nos últimos dez anos, neste país. Desde seus primeiros dias de estudante em Budapeste, Kepes tem experimentado e sido pioneiro constantemente, na esperança de descobrir um vocabulário novo e mais funcional para as artes visuais. Ao deixar de lado tradições desgastadas e convenções pesadas, ele ajudou a desenvolver novas possibilidades não apenas para as artes publicitárias, mas também para o cinema, fotografia e pintura.

Kepes exibiu seu trabalho amplamente na Europa e nos Estados Unidos. Ele ministrou cursos de design publicitário para o Art Directors Club em Chicago, e por seis anos esteve associado à Escola de Design como chefe dos departamentos de Luz e Cor. Recentemente, ele lecionou no North Texas State Teachers College em Denton, no Brooklyn College, e atualmente faz parte do corpo docente do M.I.T.

\vfill
\centering
paul theobald and company \\
5 North Wabash Ave., Chicago, Ill. 60602

\newpage
\section*{Page 232}

\textbf{ARQUITETURA} \\
\textbf{CONTEMPORÂNEA: SUAS RAÍZES} \\
\textbf{E TENDÊNCIAS} \\
por L. Hilberseimer

Um levantamento abrangente, apresentado em três seções, dos esforços criativos durante os últimos cem anos, nos diversos países, enfatizando claramente os aspectos significativos da arquitetura.

A primeira seção discute os muitos movimentos que surgiram como resultado da busca dos arquitetos pelo estilo da época.

A segunda seção aborda os períodos entre as duas Guerras Mundiais -- uma época de experimentação, descoberta e problemas de habitação.

O autor conclui com uma breve análise das tendências arquitetônicas do tempo presente, desenvolvimentos e consequências finais. 228 ilustrações. 248 páginas. 8\textonehalf x 11.
\hfill \$12,75

\bigskip

\textbf{A NATUREZA DAS CIDADES: SUA} \\
\textbf{ORIGEM, CRESCIMENTO E DECLÍNIO} \\
por L. Hilberseimer

Uma discussão completa e objetiva sobre todos os problemas de planejamento urbano, desde os primeiros assentamentos até a cidade industrial atual. 255 ilustrações. 285 páginas. 8\textonehalf x 11.
\hfill \$8,75

\bigskip

\textbf{O NOVO PADRÃO REGIONAL} \\
por L. Hilberseimer

Uma análise extensiva de princípios de planejamento sólidos e práticos em larga escala, integrando agricultura e indústria e unindo um vasto território em uma entidade orgânica autossuficiente. 125 ilustrações. 200 páginas. 8\textonehalf x 11.
\hfill \$5,50

\bigskip

\textbf{MIES VAN DER ROHE} \\
por L. Hilberseimer

Um relato lúcido da obra de Mies Van der Rohe, que descobriu, desenvolveu e aperfeiçoou as potencialidades arquitetônicas do aço. 189 ilustrações. 200 páginas. 8\textonehalf x 11.
\hfill \$9,75

\bigskip

\textbf{O QUE É DESIGN?} \\
por Paul-Jacques Grillo

Uma apresentação completa e honesta da descoberta e apreciação do bom design. Embora o autor dê ênfase à arquitetura, sua teoria de Design abrange todos os outros campos. 420 ilustrações. 200 páginas. 8\textonehalf x 11.
\hfill \$14,75

\bigskip\bigskip\bigskip

\textbf{O MUNDO NÃO-OBJETIVO} \\
por Casimir Malevich

A primeira e única tradução para o inglês, e sem dúvida, uma das mais profundas declarações de teoria estética do século vinte. 79 ilustrações. 112 páginas. 8\textonehalf x 11.
\hfill \$4,50

\bigskip

\textbf{VISÃO EM MOVIMENTO} \\
por L. Moholy-Nagy

Esta análise definitiva da educação através da arte tem sido aclamada no mundo inteiro como a mais forte influência em nossa geração. 476 ilustrações -- 12 em cores. 376 páginas. 8\textonehalf x 11.
\hfill \$16,50

\bigskip

\textbf{A NOVA PAISAGEM EM} \\
\textbf{ARTE E CIÊNCIA} \\
por G. Kepes

452 ilustrações cuidadosamente organizadas, acompanhadas de ensaios e documentos para ajudar o leitor a desenvolver sua capacidade de ver e compreender conexões significativas para si mesmo, do aspecto recém-surgido da natureza, da nova paisagem, até então, invisível, mas agora revelada pela ciência e tecnologia. 384 páginas. 8\textonehalf x 11.
\hfill \$17,50

\bigskip

\textbf{COR BÁSICA:} \\
\textbf{UMA INTERPRETAÇÃO DA} \\
\textbf{TEORIA DE OSTWALD} \\
por E. Jacobson

Devido às suas características práticas, este livro, o único que trata dos princípios básicos de organização e harmonia das cores, tornou-se a referência padrão para designers, arquitetos, professores e estudantes. 460 ilustrações em cores. 212 páginas. 8\textonehalf x 11.
\hfill \$14,75

\bigskip

\textbf{DESIGN DE MARCA REGISTRADA} \\
ed. por E. Jacobson

Uma revisão histórica e exame da utilidade da marca registrada através de sólidos princípios de design. 400 ilustrações em cores. 200 páginas. 8\textonehalf x 11.
\hfill \$8,75

\bigskip\bigskip\bigskip

\emph{Esgotado}

\textbf{A NOVA CIDADE} \\
por L. Hilberseimer

\textbf{ROUPAS SÃO MODERNAS?} \\
por B. Rudofsky

\textbf{DESIGN PARA NEGÓCIOS} \\
por J. Gordon Lippincott

\textbf{RECONSTRUINDO NOSSAS} \\
\textbf{COMUNIDADES} \\
por W. Gropius

\textbf{ARTE MODERNA NA PUBLICIDADE} \\
ed. por E. Jacobson

\textbf{TRAJE MEXICANO} \\
por C. Merida

\bigskip

\emph{Em Preparação}

\textbf{DENTRO DA BAUHAUS} \\
por Howard Dearstyne

\vspace{2cm}

\raggedright
\emph{paul theobald e companhia} \\
5 North Wabash Avenue, Chicago 60602

\end{document}